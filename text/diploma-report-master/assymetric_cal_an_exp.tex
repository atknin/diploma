
  Весьма наглядной иллюстрацией являются собственные кривые отражения от Si(440) рассчитанные при
  трех разных углах падения и соответсвенно имеющие разный коэффициент асимметрии. Угол
  Брэгга для такой плоскости отражения составляет $\theta_B = 21.68^o$, угол наклона поверхности
  составляет $\varphi = 20^o 53^{'}$.

  \begin{figure}[H]
  \centering
  \includegraphics[width=0.99\textwidth]{images/rocking_curve_assym_3.png}
  \caption{Кривые отражения 440 $MoK_{\alpha 1}$ от Si, полученные при разных углах падения (для разных b)}
  \label{ris:rocking_curve_assym_3}
  \end{figure}
  Сдвиг центра кривой происходит из-за наличия преломления на величину 0.5 и 16.5 угловых секунд.


  На рисунке \ref{ris:assymetric_exp_50}
  приведены результаты двухкристального эксперимента, где в качестве
  кристалла образца и монохроматора использовался кристалл кремния Si(440). Образец был взят таким
  образом, что плоскость отражения располагалась под углом $\phi = 20^o 53^{'}$ к поверхности.

  \begin{figure}[H]
    \centering
    \subfloat[]{\includegraphics[width=0.45\textwidth]{images/assym-blue-50.png}\label{ris:assymetric_exp_a}}
    \hfill
    \subfloat[]{\includegraphics[width=0.45\textwidth]{images/assym-red-50.png}\label{ris:assymetric_exp_b}}
    \caption{Двухкристальная КДО для схемы с установленным кристаллом монохроматором Si(440) и асимметричным образцом Si(440),
    угол разориентации поверхности $\varphi = 20^o53^{'}$ для разных углов падения (a) $b = 33.52$, $\varphi$ > 0, (b) $b = 0.03$, $\varphi$ < 0.
     Размер щелевых устройств $S_1 = S_2 = 50$ мкм}
    \label{ris:assymetric_exp_50}
  \end{figure}

Результаты полученные в ходе эксперимента находятся в согласии с теорией и при использовании резко
асимметричных рефлексом возможно получать кривые близкие к собственным кривым образца.

  % Как было показано в разделе \ref{sec:rocking_curve_section}, чтобы получить
  % рентгеновский пучок с очень малой угловой расходимостью необходимо выбирать
  % скользящий угол падения к поверхности кристалла \ref{ris:assymetric_exp_a}.
