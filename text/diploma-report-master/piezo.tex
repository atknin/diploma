Согласно сказанному в (\ref{sec:piezo_theor}), в определенных кристаллографических направлениях в
условия воздействия внешнего электрического поля, будет возникать деформация
сжатия или растяжения. Этим деформациям соответсвует изменение межплоскостных
 расстояний, которое может быть измерено с помощью дифракции рентгеновского
 излучения по изменению угла Брегговского пика \cite{marchenkov2014}.

 \begin{figure}[H]
   \centering
   \includegraphics[width=.4\textwidth]{images/piezo.png}
   \caption{Приложенное электрическое поле к $x$ - срезу образца}
   \label{ris:x_cut}
 \end{figure}

Исходя из закона Вульфа - Брегга, если межплоскостное расстояние получило приращение
$\Delta d$, тогда изменение угла отражения $\Delta \theta$ составит:

$$ \Delta d = \frac{\lambda}{2}\left( \frac{1}{\sin(\theta_B + \Delta \theta) } - \frac{1}{\sin \theta_B } \right) $$

\begin{equation}
   \Delta \theta =-  \frac{\tan \theta_B}{\frac{d}{\Delta d}+1}  = -  \frac{\Delta d }{d}  \tan \theta_B
\end{equation}
где $\Delta d/d = r$ - изменение межплоскостного расстояния. Таким образом, учитывая связь с
(\ref{eq:piezomodule}),  в кристалле толщиной $L$ и разностью потенциалов на его
гранях $V$ пьезоэлектрический
напряженность электрического поля составляет $E = \frac{V}{L}$, а модуль
 рассчитывается исходя из следующего выражения.

 \begin{equation}
    \frac{\Delta d}{d}  = -\frac{\Delta \theta \cdot L}{V \tan \theta_B}
    \label{eq:piezomodule_l}
 \end{equation}

Выражение (\ref{eq:piezomodule_l}) было использовано в следующих работах [,?,?,?,] для
пересчета отстройки брегговского максимума в величину пьезоэлектрического модуля.
Следует отметить, такой подход не является общим и имеет существенные ограничения при
измерении, например, сдвиговых пьезомодулей, а также в том случае если параметр решетки
в направлении деформации определяется не величиной одного пьезомодуля $d_{ij}$.

На рисунке \ref{ris:piezo_deformation_general} изображена элементарная ячейка
на которой обозначены кристаллографические направления по отношению к декартовой
системе координат.

\begin{figure}[H]
  \centering
  \includegraphics[width=.6\textwidth]{images/piezo_deformation_general.png}
  \caption{Триклинная ячейка в декартовой система координат}
  \label{ris:piezo_deformation_general}
\label{ris:}
\end{figure}



Рассмотрим деформационное поведение элементарной ячейки при работе пьезоэлектрического
эффект в частном случае кристалла LGT, в котором поле приложенное вдоль направления X
вызывает деформации в направлении действия модулей $d_{11}$, $d_{12}$ и $d_{14}$ (\ref{sec:piezo_matrix}).

\begin{figure}[H]
  \centering
  \includegraphics[width=.5\textwidth]{images/d11.png}
  \caption{К объяснению действия модуля $d_{11}$ на деформацию ячейки}
  \label{ris:d11}
\end{figure}

Для модуля $d_{11}=dx/x$ имеется самая простейшая ситуация, деформация вдоль оси $x$
соответсвует изменению параметра решетки $a$ на величину $da = d_{11}\cdot a$ в расчете
на метр/вольт, если перпендикулярно оси $х$ вырезать пластинку толщиной $L$
(рисунок \ref{eq:x_cut}), и на обкладки такого конденсатора подать напряжение
$V$, то "новый" параметр решетки $a{'}$ будет равен:
\begin{equation}
   a{'}  = a \left(1+\frac{d_{11}\cdot V }{L}\right)
   \label{eq:a_deformed}
\end{equation}

Для модуля $d_{12} = dy/y$, вследствие приложенного электрического поля вдоль направления $x$,
ячейка деформируется по нескольким параметрам одновременно. Происходит не только увеличение
параметра $b$, но и изменение угла $\gamma$ (рисунок \ref{ris:d12}).

\begin{figure}[H]
  \centering
  \includegraphics[width=.6\textwidth]{images/d12.png}
  \caption{К объяснению действия модуля $d_{12}$ на деформацию ячейки, вид сверху ($XY$)}
  \label{ris:d12}
\end{figure}

Исходя из теоремы косинусов получим измененный параметр $b$
\begin{equation}
   b'^2=dy^2+b^2-2 \cdot dy \cdot b \cdot \cos(270-\gamma) \nonumber
   \label{eq:b_formed_1}
\end{equation}
\begin{equation}
   b' = b \sqrt{ 2\sin^2 \gamma \cdot d_{12}+1}
   \label{eq:b_formed_2}
\end{equation}
из теоремы синусов следует изменение параметра $\gamma$
\begin{equation}
   \frac{\sin(270-\gamma)}{b^{'}} = \frac{\sin (\Delta \gamma)}{dy} \nonumber
   \label{eq:b_formed_3}
\end{equation}
\begin{equation}
   \gamma^{'} = \gamma + \frac{\cos\gamma \cdot \sin \gamma \cdot d_{12}}{ \sqrt{ \sin^2 \gamma  (d_{12}^2 + 2d_{12})+1}}
   \label{eq:b_formed_4}
\end{equation}


Модуль $d_{14} = dy/z$ является сдвиговым, его действие обуславливается не только
изменением угла $\beta$, но параметра $c$ (рисунок \ref{ris:d14}).
\begin{figure}[H]
  \centering
  \includegraphics[width=.6\textwidth]{images/d14.png}
  \caption{К объяснению действия модуля $d_{14}$ на деформацию ячейки, вид ($YZ$)}
  \label{ris:d14}
\end{figure}
из теоремы косинусов получим зависимость величины сдвига от параметра решетки $с$,
который находится под наклоном к плоскости $YZ$ на угол $\alpha$.
$$
    dy^2 = A^2 + A{'}^2 - 2 A{'}A \cdot \cos \Delta \beta
$$
сделаем приближение по малому углу $\Delta \beta$, тогда величина сдвига будет
определятся разностью длин сторон треугольника
$$
  dy = A^{'} - A
$$
из теоремы синусов приращение угла $\beta$ задается выражением,
$$
\Delta \beta = \frac{dy \cdot\sin \beta}{ A+dy }
$$
где
$$
dy = d_{14} \cdot c \cdot \sin \alpha \cdot \sin \beta
$$
тогда конечные выражения для измененных параметров выглядят следующим образом
\begin{equation}
   \Delta \beta = \frac{d_{14} \cdot \sin^2\beta}{1+d_{14}\sin \beta}
   \label{eq:b_formed_5}
\end{equation}
\begin{equation}
   c{'} = c(1+d_{14}\sin\beta)
   \label{eq:b_formed_6}
\end{equation}

Для того, чтобы получить значение угла смещения брегговского максимума,
необходимо рассчитать межплоскостное расстояние для выбранных индексов отражения
до (недеформированная ячейка) и после (деформированная) приложения электрического
поля.
