Согласно сказанному в (\ref{sec:piezo_theor}), в определенных кристаллографических направлениях в
условия воздействия внешнего электрического поля, будет возникать деформация
сжатия или растяжения. Этим деформациям соответсвует изменение межплоскостных
 расстояний, которое может быть измерено с помощью дифракции рентгеновского
 излучения по изменению угла Брегговского пика \cite{marchenkov2014}.

Исходя из закона Вульфа - Брегга, если межплоскостное расстояние получило приращение
$\Delta d$, тогда изменение угла отражения $\Delta \theta$ составит:

$$ \Delta d = \frac{\lambda}{2}\left( \frac{1}{\sin(\theta_B + \Delta \theta) } - \frac{1}{\sin \theta_B } \right) $$

\begin{equation}
   \Delta \theta =-  \frac{\tan \theta_B}{\frac{d}{\Delta d}+1}  = -  \frac{\Delta d }{d}  \tan \theta_B
\end{equation}
где $\Delta d/d = r$ - изменение межплоскостного расстояния. Таким образом, учитывая связь с
(\ref{eq:piezomodule}), для напряженности электрического поля $E = \frac{V}{L}$ для
 кристалла толщиной $L$ и разностью потенциалов на его гранях $V$ пьезоэлектрический модуль
 рассчитывается исходя из следующего выражения

 \begin{equation}
    d_{ij} = -\frac{\Delta \theta \cdot L_i}{V \tan \theta_B^i}
 \end{equation}
