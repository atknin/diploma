\subsection{Методика получения двухкристальных кривых дифракционного отражения}

Измерение кривой дифракционного отражения в двухкристальной схеме представляет
собой измерение зависимости отраженного образцом рентгеновского излучения при
пошаговом повороте исследуемого кристалла относительно падающего на него
излучения в окрестности точного значения угла Брэгга.
Существует несколько схем измерения кривых отражения рентгеновского излучения.

\subsubsection*{$\omega$ - сканирование}
В данном типе сканирования кривая отражения измеряется путем поворота образца
относительно падающего пучка в плоскости дифракции. При таком сканировании
угол между падающим и дифрагированным пучками (угол рассеяния) остается постоянным
(рисунок ~\ref{ris:omega_scan}). Получаемая в результате кривая носит название кривой качания.


\begin{figure}[H]
  \centering
  \includegraphics[width=1\textwidth]{images/omega_scan.png}
  \caption{Схема реализации $\omega $ - сканирования}
  \label{ris:omega_scan}
\end{figure}

\subsubsection*{$\vartheta - 2\vartheta$ - сканирование}
В отличие от предыдущего, данный метод сканирования соответствует изменению
 модуля вектора рассеяния при неизменном его угловом положении
 (рисунок ~\ref{ris:theta_2theta_scan}). Угловое положение падающего пучка и
 детектора изменяется синхронно и симметрично относительно используемой системы
 атомных плоскостей, а установленная перед детектором апертурная щель вырезает
  только зеркально отраженную часть пучка. Именно поэтому при построении карт
   пространственного распределения спектра полосы щелей на этих картах остаются
   неподвижными (т.к. несмотря на движение щели  $S_2$ в процессе
    $\vartheta - 2\vartheta$ -  сканирования ее отстройка от зеркального
    положения всегда равна 0).

 \begin{figure}[H]
   \centering
   \includegraphics[width=1\textwidth]{images/theta_2theta_scan.png}
   \caption{Схема реализации $\vartheta - 2\vartheta$ - сканирования}
   \label{ris:theta_2theta_scan}
 \end{figure}
Кроме того, используемый подход позволяет наглядно продемонстрировать интересный эффект.
 Независимо от ширины входной и приемной щелей характеристическая линия спектра
 трубки $k_{\alpha 2}$ всегда вносит вклад в КДО, проявляясь в виде дополнительного
 пика на ее хвосте (рис. 2.21). Данный слабый пик возникает за счет того, что
 даже при очень малой входной щели $S_1$ линия $k_{\alpha 2}$ будет, отражаясь на «хвосте»
 кривой монохроматора, пролетать через входную щель и, при определённом угле
 поворота образца, интенсивно дифрагировать в максимуме его собственной кривой
 отражения и давать весомый ($~10^{-6}$) вклад в общую интенсивность КДО.


\subsubsection{Выражение для расчета двухкристальных КДО}

Для того, чтобы разобраться в том, как формируются экспериментальные
двухкристальные КДО, нам необходимо построить спектрально-угловое распределение
в соответствии со схемой эксперимента (рисунок \ref{ris:double_crystal_schem_lamtet_a}).

\begin{eqnarray} \label{eq:doudle_spectra_angle_map}
  P(\theta,\vartheta,\lambda) = g_{\lambda}(\lambda)g_{\vartheta}(\vartheta) P_M \left(\vartheta - \frac{\lambda - \lambda_1}{\lambda_1}\tan(\theta_B) \right) \cdot \nonumber \\
   P_S \left(\theta + \vartheta - \frac{\lambda - \lambda_1}{\lambda_1}\tan(\theta_B)\right)
 \end{eqnarray}

Выражение (\ref{eq:doudle_spectra_angle_map}) определяет спектрально угловое распределение после прохождения двух кристаллов с
коэффициентами отражения  $P_M$ (монохроматор) и $P_S$ (образец), причем последний принимает во внимание положение угла отстройки $\theta$ относительно
точного Брегга (рисунок \ref{ris:double_crystal_schem_lamtet_b}).

\begin{figure}[H]
  \centering
  \subfloat[Схема двухкристального эксперимента]{\includegraphics[width=0.5\textwidth]{images/double_crystal_schem.png}\label{ris:double_crystal_schem_lamtet_a}}
  \hfill
  \subfloat[Спектрально угловое распределение. Положение щелевых устройств обозначено синей и белой линиями вблизи $\vartheta = 0$ угл.сек. Кристалл-образец
  выведен из точного Брегговского положения на 300 угл. сек.
   ]{\includegraphics[width=0.45\textwidth]{images/double_crystal_schem_lamtet.png} \label{ris:double_crystal_schem_lamtet_b}}

  \caption{Схема и спектрально-угловое распределение после отражения расходящегося, полихромотического пучка. Несмотря на то, что в
  экспериментальной схеме детектор со щелью не стоят на месте, на карте обе щели $S_1$ и $S_2$ - неподвижны. }
  \label{ris:double_crystal_schem_lamtet}
\end{figure}

Выражение (\ref{eq:doudle_spectra_angle_map}) не учитывает особенности особенности влияния щелевых коллиматоров, о которым мы говорили в
(раздел \ref{sec:slits_section}), а так же тот факт, что детектор не разделят энергетическую составляющую пучка.


\begin{eqnarray} \label{eq:doudle_spectra_angle_map_on_detector}
  P_{double}(\theta) = \sum_{\lambda = -\infty}^{\infty}g_{\lambda}(\lambda)\cdot
  \sum_{\vartheta = \vartheta_{s1}}^{\vartheta_{s2}} g_{\vartheta}(\vartheta) g_{S}(\vartheta) \cdot \nonumber \\
   P_M \left(\vartheta - \frac{\lambda - \lambda_1}{\lambda_1}\tan(\theta_B) \right) \cdot \nonumber \\
   P_S \left(\theta + \vartheta - \frac{\lambda - \lambda_1}{\lambda_1}\tan(\theta_B)\right)
 \end{eqnarray}
 где пределы суммирования определяются как $\vartheta_{s2} = - \vartheta_{s1} = \frac{\delta+S_1}{2L_1}$.

 \begin{figure}[H]
   \centering
   \includegraphics[width=1\textwidth]{images/double_crystal_form_kdo.png}
   \caption{Формирование двухкристальной кривой дифракционного отражения ($\theta - 2\theta$ - cканирование),
   $\theta_B^M = \theta_B^S = 10.6^o$  }
   \label{ris:double_crystal_form_kdo}
 \end{figure}

В том случае, если схема дисперсионная т.е. углол Брегга кристалла - образца отличен от угла Брега кристалла-монохроматора,
наблюдается уширения двухкристальных кривых (рисунок \ref{ris:double_crystal_form_kdo_dissp}).
 \begin{figure}[H]
   \centering
   \includegraphics[width=1\textwidth]{images/double_crystal_form_kdo_dissp.png}
   \caption{Формирование двухкристальной кривой дифракционного отражения ($\theta - 2\theta$ - cканирование) в случае
   наличия дисперсии, $\theta_B^M = 10.6^o$, $\theta_B^S = 21.6^o$ }
   \label{ris:double_crystal_form_kdo_dissp}
 \end{figure}
