Цель, поставленная в рамках настоящей работы, достигнута в полном объеме.
В частности, были разработаны универсальные алгоритмы расчета,
позволяющие моделировать картину рентгеновской дифракции для различных типов
источников излучения и широкого набора оптических элементов и учитывающие большое
количество эффектов, возникающих в реальной экспериментальной схеме. Данные 
алгоритмы реализованы в виде web-интерфейса, позволяющего широкому кругу
исследователей проводить моделирование дифракционной картины для экспериментальных схем,
собираемых из оптических элементов по модульному принципу. В то же время,
 результаты работы создают предпосылки для дальнейших исследований и расширения
 функционала алгоритмов моделирования, в частности, в направлении учета дефектов
 структуры кристалла, восстановления полной матрицы пьезомодулей по данным
 рентгеновской дифракции, моделирования вклада диффузного рассеяния в картину ТРД.
