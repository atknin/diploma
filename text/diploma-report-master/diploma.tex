\documentclass[pdftex,a4paper,14pt,english,russian]{extarticle}

\usepackage[top=2cm,bottom=27mm,left=3cm,right=18mm]{geometry}
\usepackage[T2A]{fontenc}
\usepackage[utf8]{inputenc}
\usepackage[english,russian]{babel}
\usepackage[pdftex]{hyperref}
\usepackage{indentfirst}
\usepackage{titlesec}
\usepackage[pdftex]{graphicx}
\usepackage{amsmath}
\usepackage{amssymb}
\newcommand*\Laplace{\mathop{}\!\mathbin\bigtriangleup}
\usepackage[makeroom]{cancel}
\usepackage{ragged2e}

\usepackage{multirow}
\usepackage{algorithmic}
\usepackage[boxed]{algorithm}
\usepackage{tabularx}
\usepackage{xtab}
\usepackage{subfig}
\usepackage{hyphenat}
\usepackage{setspace}
\usepackage{fancyhdr}




% код программы - Листинг
\usepackage[usenames, dvipsnames]{color}
\definecolor{mygreen}{rgb}{0,0.6,0}
\usepackage{listings}
 \lstset{commentstyle=\color{mygreen},
 literate=
 	{~}{{\textasciitilde}}1
 	{а}{{\selectfont\char224}}1
 	{б}{{\selectfont\char225}}1
 	{в}{{\selectfont\char226}}1
 	{г}{{\selectfont\char227}}1
 	{д}{{\selectfont\char228}}1
 	{е}{{\selectfont\char229}}1
 	{ё}{{\"e}}1
 	{ж}{{\selectfont\char230}}1
 	{з}{{\selectfont\char231}}1
 	{и}{{\selectfont\char232}}1
 	{й}{{\selectfont\char233}}1
 	{к}{{\selectfont\char234}}1
 	{л}{{\selectfont\char235}}1
 	{м}{{\selectfont\char236}}1
 	{н}{{\selectfont\char237}}1
 	{о}{{\selectfont\char238}}1
 	{п}{{\selectfont\char239}}1
 	{р}{{\selectfont\char240}}1
 	{с}{{\selectfont\char241}}1
 	{т}{{\selectfont\char242}}1
 	{у}{{\selectfont\char243}}1
 	{ф}{{\selectfont\char244}}1
 	{х}{{\selectfont\char245}}1
 	{ц}{{\selectfont\char246}}1
 	{ч}{{\selectfont\char247}}1
 	{ш}{{\selectfont\char248}}1
 	{щ}{{\selectfont\char249}}1
 	{ъ}{{\selectfont\char250}}1
 	{ы}{{\selectfont\char251}}1
 	{ь}{{\selectfont\char252}}1
 	{э}{{\selectfont\char253}}1
 	{ю}{{\selectfont\char254}}1
 	{я}{{\selectfont\char255}}1
 	{А}{{\selectfont\char192}}1
 	{Б}{{\selectfont\char193}}1
 	{В}{{\selectfont\char194}}1
 	{Г}{{\selectfont\char195}}1
 	{Д}{{\selectfont\char196}}1
 	{Е}{{\selectfont\char197}}1
 	{Ё}{{\"E}}1
 	{Ж}{{\selectfont\char198}}1
 	{З}{{\selectfont\char199}}1
 	{И}{{\selectfont\char200}}1
 	{Й}{{\selectfont\char201}}1
 	{К}{{\selectfont\char202}}1
 	{Л}{{\selectfont\char203}}1
 	{М}{{\selectfont\char204}}1
 	{Н}{{\selectfont\char205}}1
 	{О}{{\selectfont\char206}}1
 	{П}{{\selectfont\char207}}1
 	{Р}{{\selectfont\char208}}1
 	{С}{{\selectfont\char209}}1
 	{Т}{{\selectfont\char210}}1
 	{У}{{\selectfont\char211}}1
 	{Ф}{{\selectfont\char212}}1
 	{Х}{{\selectfont\char213}}1
 	{Ц}{{\selectfont\char214}}1
 	{Ч}{{\selectfont\char215}}1
 	{Ш}{{\selectfont\char216}}1
 	{Щ}{{\selectfont\char217}}1
 	{Ъ}{{\selectfont\char218}}1
 	{Ы}{{\selectfont\char219}}1
 	{Ь}{{\selectfont\char220}}1
 	{Э}{{\selectfont\char221}}1
 	{Ю}{{\selectfont\char222}}1
 	{Я}{{\selectfont\char223}}1
 	{і}{{\selectfont\char105}}1
 	{ї}{{\selectfont\char168}}1
 	{є}{{\selectfont\char185}}1
 	{ґ}{{\selectfont\char160}}1
 	{І}{{\selectfont\char73}}1
 	{Ї}{{\selectfont\char136}}1
 	{Є}{{\selectfont\char153}}1
 	{Ґ}{{\selectfont\char128}}1
}
\newcommand{\angstrom}{\buildrel _{\circ} \over {\mathrm{A}}}
\numberwithin{equation}{subsection}
\pagestyle{fancyplain}
\fancyhf{}

\renewcommand{\headrulewidth}{0pt}
\rfoot{\fancyplain{}{\centering \thepage}} %нумерация страниц

\linespread{1.3}
\floatname{algorithm}{Алгоритм}

\addto\captionsrussian{ \renewcommand{\contentsname}{ СОДЕРЖАНИЕ}}
\addto\captionsrussian{ \renewcommand\refname{СПИСОК ИСПОЛЬЗОВАННЫХ ИСТОЧНИКОВ}}

% прилоение
\addto\captionsrussian{ \renewcommand\appendixname{Приложение}}
\makeatletter
\def\redeflsection{\def\l@section{\@dottedtocline{1}{1.5em}{7.8em}}}
\renewcommand\appendix{\par
\setcounter{section}{0}%
\setcounter{subsection}{0}%
\def\@chapapp{\appendixname}%
\addtocontents{toc}{\protect\redeflsection}
\def\thesection{\appendixname\hspace{0.2cm}\@arabic\c@section}}
\makeatother
% /прилоение
% new commands
\newcommand\sign[1]{sign(#1)}



% подрисуночная подпись

\addto\captionsrussian{\renewcommand{\figurename}{Рисунок}}
\makeatletter
\long\def\@makecaption#1#2{%
  \vskip\abovecaptionskip
  \hbox to\textwidth{\hfill\parbox{1\textwidth}{\centering #1 - #2}\hfill}
  \vskip\belowcaptionskip}
\makeatother

% новый уровень
\titleclass{\subsubsubsection}{straight}[\subsection]

\newcounter{subsubsubsection}[subsubsection]

\renewcommand\thesubsubsubsection{\thesubsubsection.\arabic{subsubsubsection}}
\renewcommand\theparagraph{\thesubsubsubsection.\arabic{paragraph}}
\renewcommand\thesubparagraph{\theparagraph.\arabic{subparagraph}}

\titleformat{\subsubsubsection}
  {\normalfont\normalsize\bfseries}{\thesubsubsubsection}{1em}{}
\titlespacing*{\subsubsubsection}
{0pt}{3.25ex plus 1ex minus .2ex}{1.5ex plus .2ex}

\makeatletter
\renewcommand\paragraph{\@startsection{paragraph}{5}{\z@}%
  {3.25ex \@plus1ex \@minus.2ex}%
  {-1em}%
  {\normalfont\normalsize\bfseries}}
\renewcommand\subparagraph{\@startsection{subparagraph}{6}{\parindent}
  {3.25ex \@plus1ex \@minus .2ex}%
  {-1em}%
  {\normalfont\normalsize\bfseries}}
\def\toclevel@subsubsubsection{4}
\def\toclevel@paragraph{5}
\def\toclevel@paragraph{6}
\def\l@subsubsubsection{\@dottedtocline{4}{7em}{4em}}
\def\l@paragraph{\@dottedtocline{5}{10em}{5em}}
\def\l@subparagraph{\@dottedtocline{6}{14em}{6em}}
\@addtoreset{subsubsubsection}{section}
\@addtoreset{subsubsubsection}{subsection}
\@addtoreset{paragraph}{subsubsubsection}
\makeatother
\setcounter{secnumdepth}{6}
\setcounter{tocdepth}{6}

% названия с красной строки
\titlespacing*{\section}{\parindent}{1ex}{1em}
\titlespacing*{\subsection}{\parindent}{1ex}{1em}
\titlespacing*{\subsubsection}{\parindent}{1ex}{1em}
\titlespacing*{\subsubsubsection}{\parindent}{1ex}{1em}
%добавить точку после номера
\makeatletter
\renewcommand{\@seccntformat}[1]{\indent \csname the#1\endcsname.\quad}
\makeatother

\begin{document}
\begin{titlepage}
  \begin{center}
    % \textsc{\small Министерство образования Республики Беларусь}\\[0.5cm]
    \textsc{\small \large федеральное государственное бюджетное образовательное
    учреждение высшего образования "московский государственный университет им. Ломоносова"}\\[0.3cm]
    \textsc{\large Физический факультет}\\[0.3cm]
    \textsc{\large кафедра оптики, спектроскопии и физики наносистем}\\[0.5cm]

    \begin{minipage}{\textwidth}
      \begin{flushleft}

      \end{flushleft}
    \end{minipage}\\[0.5cm]

    \textsc{\large  магистерская диссертация}\\[0.5cm]
    \textbf{\large Двух- и трехкристальная дифрактометрия в исследовании
    пьезоэлектрических кристаллов в условиях воздействия электрического поля}\\[1.5cm]


    \begin{minipage}{\textwidth}
      \begin{flushright}
        Выполнил студент группы 241М \hspace*{2.1cm} \\
        Аткнин И. И.\underline{\hspace*{4.6cm}}\\[0.5cm]
        Научный руководитель: к.ф.-м. н. \hspace*{1.8cm} \\
        Марченков Н. В. \underline{\hspace*{3.7cm}}\\[0.5cm]
        Научный руководитель: к.ф.-м. н., доцент\\
        Стремоухов С. Ю. \underline{\hspace*{3.7cm}}\\[0.5cm]
      \end{flushright}
    \end{minipage}\\[1.5cm]


    \begin{minipage}{\textwidth}
      \begin{flushleft}
        \textit{Допущена к защите 31.05.2017}\\
        Зав. кафедрой \underline{\hspace*{4.5cm}}\\
      \end{flushleft}
    \end{minipage}\\[1.5cm]

    \vfill
    \textsc{\small Москва\\ 2017}
  \end{center}
\end{titlepage}

\setcounter{page}{2}

\titleformat*{\section}{\centering\bfseries\Large}
\tableofcontents
\titleformat*{\section}{\bfseries\Large}


\newpage
% ---------------------section 1 -----------------------
  \section*{\centering ВВЕДЕНИЕ}
  \addcontentsline{toc}{section}{\protect\numberline{}ВВЕДЕНИЕ}%
  
Здесь написать, помимо актуальности, обзор разделов диплома. Зачем нужен тот или иной раздел.
К примеру, доказать наличие кинематической теории.

  \newpage
\section{Литературный обзор}
  \subsection{Атомный фактор рассеяния}
    
Рентгеновское излучение, взаимодействуя с электронами атомов вещества рассеивается.
Протоны (ядра атомов) в рассеянии рентгеновских лучей практически не участвуют, т.к.
амплитуда электромагнитной волны, рассеянной заряженной частицей,
 обратно пропорциональна ее массе - формула Томсона \cite{iveronova1972}. Величина такого рассеяния
 зависит от количества электронов в атоме.  Тяжелые металлы,
 например свинец, Pb (Z = 82), рассеивают рентгеновское излучение сильнее легких,
 таких как Ni (Z = 28) или  Co (Z = 27), а такие атомы, как He или H – прозрачны
 для рентгеновского излучения. Определим атомный множитель $f$ (атомный фактор рассеяния)
как отношение амплитуды волны, рассеянной одним атомом, к амплитуде волны, рассеянной
одним свободным электроном. Действительно, если в какой либо точке пространства сосредоточено
$Z$ электронов, то заряд этой группы равен $Q = Z\cdot e$, а масса $M = Z \cdot m_e$.

На рисунке ~\ref{ris:atom_factor} представлена диаграмма направленности атомного
фактора лантана в зависимости от угла. Размеры атома соизмеримы с длиной волны
рентгеновских лучей, поэтому между волнами рассеянными отдельными электронами, возникает
разность фаз. Это разность фаз равна нулю только при $2 \theta = 0$, поэтому структурный
фактор зависит от $\theta$ и $\lambda$. Максимальная величина, которая равна $Z$,
 наблюдается в случае рассеяния вперед и рассеяния назад.

\begin{figure}[H]
  \centering
  \includegraphics[width=0.9\textwidth]{images/atom_factor.png}
  \caption{ (Слева) фактор рассеяния для атома лантана (La, N = 57), (справа)
  схема расположения векторов для падающей и рассеянной волн}
  \label{ris:atom_factor}
\end{figure}

Приближенное выражение для расчета атомного фактора рассеяния
представляется \cite{International_Tables} в виде выражения:

\begin{equation}
  f_0 = \sum_{i=1}^{4} \cdot a_i e^{ -b_i (\frac{sin \vartheta_B}{\lambda})^2} + C
 \end{equation}
где $a_i$, $b_i$ и $c$ - коэффициенты Кромер-Манна для бездисперсионного канала рассеяния атомами решетки,
ограничением является $0<\frac{sin\vartheta}{\lambda}<2.0 \angstrom ^{-1}$.
 Характерная зависимость структурного фактора от угла рассеяния и длины волны
для атомов входящих в состав кристалла LGT (La, Ga, Ta, O) представлена на рисунке ~\ref{ris:atom_factor_GaLaTa}.

\begin{figure}[H]
  \centering
  \includegraphics[width=0.6\textwidth]{images/atom_factor_GaLaTa.png}
  \caption{ Атомный фактор рассеяния для атомов: галлия (Ga), лантана (La), тантала (Ta) и  кислорода (O)}
  \label{ris:atom_factor_GaLaTa}
\end{figure}

При расчете интенсивности рассеяния атомом необходимо учитывать факт,
что все электроны связаны между собой, таким образом необходимо записывать
уравнение движение связанного электрона по действием падающего излучения \cite{iveronova1972}.
Если атом многоэлектронный, то амплитуда рассеянной волны равна сумме амплитуд волн,
рассеянных всеми электронами атома:

\begin{equation}
  f = f_0 + f^{'} + i f^{''}
 \end{equation}

 где, $f_0$ - атомный фактор рассеяния, рассчитанный без учета сил связи электронов
 с ядром, а $f^{'}$ и $f^{''}$ - дисперсионные поправки \cite{f0f1f12},
 первая из которых учитывает дополнительное рассеяние,
а вторая - дополнительное поглощение вблизи собственных частот колебаний электронов в атоме.

 \begin{figure}[H]
   \centering
   \includegraphics[width=0.4\textwidth]{images/dispers_f.png}
   \caption{ Схематичная зависимость атомного фактора $f^2 = (f_0 + f^{'})^2 + (f^{''})^2 $ от
   длины волны $\lambda$ вблизи края поглощения}
   \label{ris:dispers_f}
 \end{figure}

Дисперсионные поправки зависят от длины волны и практически не зависят
от $\theta$. А так как $f_0$ уменьшается с ростом угла рассеяния,
 дисперсионные поправки начинают играть роль при больших углах рассеяния.

  \subsection{Структурный фактор рассеяния}
    \label{sec:structure_factor}
Атомы решетки, взаимодействуя с рентгеновским излучением, рассеивают его.
Если в элементарной ячейке более одного атома, волны рассеянные разными атомами,
 интерферируя между собой, вносят вклад в общую картину рассеяния,
 ослабляя или усиливая ее.

 \begin{figure}[H]
   \centering
   \subfloat[]{\includegraphics[width=0.45\textwidth]{images/interference_construct.png}}
   \hfill
   \subfloat[]{\includegraphics[width=0.45\textwidth]{images/interference_destruct.png}}
   \caption{Примеры интерференции двух волн, отраженных различными системами атомных плоскостей для случая
   конструктивной (a) и деструктивной (b) интерференции}
   \label{ris:interference_by_plate}
 \end{figure}

Рассеяние от набора атомов характеризуется структурным фактором, определяемым векторным
 сложением фаз по всем N атомам элементарной ячейки:

 \begin{equation}
   F = \sum_{n} f_n e^{ i\vec{h}\vec{r}_n} =   \sum_{n} f_n \cdot e^{-i\phi_n},
   \label{eq:F_factor}
  \end{equation}
\noindent
где $\phi_n = 2 \pi (hx_n+ky_n+lz_n)$;  $h, k, l$ - индексы Миллера; $x, y, z$ - относительные координаты
атомов в элементарной ячейке.

В соответсвии с \ref{eq:F_factor} в качестве примера был произведен расчет трехмерной ($hkl$) -
карты струкутрного фактора (рис. \ref{ris:hkl_LGT_SI}).
Цветом изображена величина структурного фактора для разных
 индексов плоскостей отражения для кристаллов LGT и Si.
 В таком представлении просматривается периодичность образования запрещенных
 рефлексов в кубическом кремнии. В кристалле LGT запрещенных (синий цвет)
  индексов для отражения на порядок меньше, связанно это с более низкой
  симметрией кристалла.

  \begin{figure}[H]
    \centering
    \subfloat[]{\includegraphics[width=0.49\textwidth]{images/hkl_Si.png} \label{ris:hkl_LGT_SI_a}}
    \hfill
    \subfloat[]{\includegraphics[width=0.49\textwidth]{images/hkl_LGT.png} \label{ris:hkl_LGT_SI_b}}
    \caption{Карта распределения величины структурного фактора
    (цвет соответствует его величине) в координатах индексов Миллера для кристалла Si (a) и LGT (b)}
    \label{ris:hkl_LGT_SI}
  \end{figure}

  \subsection{Влияние температуры. Тепловой фактор Дебая - Валлера}
    \subsection{Температурный фактор рассеяния}
При определении положений атомов следует также
учитывать их тепловые колебания около равновесных
 положений, нарушающих «совершенность» решетки.
 Мерой смещения атомов при тепловых колебаниях служит
 их среднеквадратичная амплитуда $u^2$.
 Структурный фактор рассеяния для колеблющихся атомов имеет вид:

 \begin{equation}
   F_T = F\cdot e^{-B(\frac{sin\vartheta_B}{\lambda})^2}
  \end{equation}
где $B = 8 \pi^2 u^2$ - температурный фактор (фактор Дебая - Валлера).
Величина $B$ может варьироваться в диапазоне от $1 \angstrom $ до $ 100\angstrom $.

  % \input{kinematic_theory.tex}
  \subsection{Динамическая теория рассеяния}
    \subsubsection{Симметричная схема дифракции}
      \subsubsection{Симметричная схема дифракции}
  При рассмотрении большинства физических процессов, задействованных
  в методах исследования структуры веществ с помощью рентгеновских лучей,
  используется математический аппарат волновой оптики.
  Плоская монохроматическая волна, распространяющая в вакууме, изображена на рисунке ~\ref{ris:plane_wave_vacuum},
  амплитуда плоских волн в вакууме $E_0$ не меняется с удалением от
  источника (в отличии от сферических или цилиндрических). В приближении плоской волны,
  плотность потока энергии, переносимой волной через единицу площади неизменна на любом расстоянии от источника.

  \begin{figure}[H]
    \centering
    \includegraphics[width=0.8\textwidth]{images/plane_wave_vacuum.png}
    \caption{Схема к описанию динамической теории дифракции рентгеновского излучения с кристаллом.
      $\vec {k}$ - волновой вектор в вакууме; $\vec {q}$ - волновой вектор в среде;
     $0$ - коэффициент для обозначения падающей волны, а $h$ -  дифрагированной волны; $\vec{h}$ - вектор
     обратной решетки ($|h|=2\pi/d$); $\vec{n}$ - вектор нормали к поверхности, направленный внутрь объема;
     $\lambda$ - длина волны волнового вектора; $\vec{\varepsilon}$ - вектор аккомодации, характеризующий изменение
     волнового вектора в среде из-за преломления; $\vartheta$ - угол падения излучения на кристалл, для данного
     случая угол совпадает с углом Брегга $\vartheta = \theta_B$, т.к.  $\vec {k}_0 + \vec{h} = \vec {k}_h $}
    \label{ris:plane_wave_vacuum}
  \end{figure}

Рентгеновские лучи, как и видимы свет, распространяются параллельно и преломляются при
прохождении через границу раздела двух сред с разной оптической плотностью.
 Преломление рентгеновских лучей намного слабее, чем у видимого света, причем
 абсолютный показатель преломления рентгеновских лучей практически во всех средах
 практический одинаков и настолько близок к единице, что их преломление не удавалось обнаружить
 в течение тридцати лет после открытия рентгеновских лучей \cite{fetisov2007}, более того
 для рентгеновских лучей вакуум оказывается оптически наиболее плотной средой и луч
 при переходе в конденсированную среду увеличивает угол с нормалью к поверхности раздела сред ($n_{refr} \approx 1-10^{-5}$ ).
 Таким образом, волновой вектор, распространяющийся в вакууме отличается от своего продолжения в
 среде, но тангенциальная составляющая при переходе из одной среды в другу, в соответствии с теорией о циркуляции, сохраняется \cite{landau_8_1992}.

 \begin{equation}
   \vec{q}_0 = \vec{k}_0 + \varepsilon k_0 \cdot \vec{n}
  \end{equation}

Квадрат вектора,

\begin{equation}
   q_0^2 = k_0^2 + 2k_0^2 \varepsilon \cdot \gamma_0 + \cancelto{0}{k_0^4  \varepsilon^2}
   \label{eq:k_0_squred}
 \end{equation}

  где, $\gamma_0 = \cos(\vec{k}_0 \textasciicircum \vec{n})$ - косинус угла между вектором $\vec{k}_0$ и нормалью к поверхности кристалла,
  последним слагаемым можно пренебречь в силу его малости ($\sim 10^{-6}$).
  Волновой вектор дифрагированной волны, в соответствии с условием Брегга,

  $$\vec{k}_h = \vec{k}_0+\vec{h}$$

  \begin{equation}
     k_h^2 = \vec{k}_0^2+2k_0^2 \varepsilon \cdot \gamma_h
     \label{eq:k_h_squred}
   \end{equation}
   где, $\gamma_h = \cos(\vec{k}_0+ \vec{h} \textasciicircum \vec{n})$ - косинус угла между вектором $\vec{k}_h$ и нормалью к поверхности кристалла.

Для дальнейшего рассмотрения уравнения связывающего амплитуду падающей и дифрагированной волн в рамках
динамической теории рассеяния введем следующий параметр $\alpha$, характеризующий степень отклонения от условия Брегга.
\begin{equation}
   \alpha = \frac{k_0^2-k_h^2}{k_0^2}
   \label{eq:alpha}
\end{equation}


$$  \alpha = 1 - \frac{|\vec{k}_0|^2+2|\vec{k}_0||\vec{h}|\cos(\vec{k}_0 \textasciicircum \vec{h})+|\vec{h}|^2}{k_0^2}$$

учитывая, что $ |h| = 2|k_0| \sin(\theta_B) $, а $\vec{k}_0 \textasciicircum \vec{h} = 90-\theta_B+\vartheta$, получим:

\begin{equation}
   \alpha = -4\sin(\theta_B)(\sin(\theta_B+\vartheta)-\sin(\theta_B))
\end{equation}

    \subsubsection{Асимметричная схема дифракции}
      В том случае если рентгеновское излучение отражается от атомных плоскостей не
 параллельных поверхности, в таком случае говорят об асимметрии отражения (рисунок ~\ref{ris:assymetric_brag}).

\begin{figure}[h]
  \centering
  \subfloat[$b >> 1$, $\varphi$ > 0]{\includegraphics[width=0.45\textwidth]{images/assym2.png}\label{fig:f1}}
  \hfill
  \subfloat[$b << 1$, $\varphi$ < 0]{\includegraphics[width=0.45\textwidth]{images/assym1.png}\label{fig:f2}}
  \caption{Схема Брегговской дифракции для асимметричного отражения}
  \label{ris:assymetric_brag}
\end{figure}

Для того чтобы охарактеризовать степень асимметрии, введем коэффициент $b$:

\begin{equation}
  b = \frac {\gamma_0}{|\gamma_h|}
  \label{eq:koef_b}
 \end{equation}
где, $\gamma_0 = cos \psi_0 = sin ( \varphi + \theta_B)$, $\gamma_h = cos \psi_h = sin ( \varphi - \theta_B)$,
$\varphi$ - угол между плоскостью отражения и поверхностью образца.

    \subsubsection{Рентгеновская поляризуемость в среде}
      
Вне кристалла, падающая волна описывается в виде совокупности плоских волн с волновым вектором  $\vec{k_0}$.
\begin{equation}
  \vec{E}_0(\vec{r},t) = E_0 e^{i(\vec{k}_0\vec{r}-\omega t)}
 \end{equation}

Падающая волна $\vec{E}(\vec{r},t)_0$ порождает волновое поле внутри кристалла, которое характеризуется вектором
электромагнитной индукции, параллельным вектору напряженности электрического поля $\vec{D}||\vec{E}$, тогда
имеет место:
\begin{equation}
 \vec{D}_0(\vec{r},t) = (1+\chi(\vec{r})) E_0 e^{i(\vec{k}_0\vec{r}-\omega t)} = A(r) e^{i(\vec{k}_0\vec{r}-\omega t)}
\end{equation}
 \noindent
где, $\chi$ - поляризуемость среды. Амплитуда волны $A(\vec{r})$ - не зависит
от времени, но зависит от координат, связано
это с тем что электроны колеблются под действие распространяющейся волны, и испускаемые ими
электромагнитные волны интерферируют между собой и с исходной волной. Устанавливается некоторое стабильное
электромагнитное поле с переодически изменяющейся в пространстве амплитудой. Периодичность эта должна быть
той же, что и периодичность решетки. Таким образом, в силу трехмерной периодичности $\chi(\vec{r}+\vec{h}) = \chi(\vec{r})$
, функцию $\chi(\vec{r})$ можно разложить в ряд Фурье и представить в виде

\begin{equation}
\chi(\vec{r}) = \sum_{h}\chi_h e^{i\vec{h}\vec{r}}
\end{equation}

Подробный вывод выражений для Фурье компонент $\chi_h$ для рентгеновской поляризуемости
 в среде представлен в (\ref{sec:polarizability}), из которого следует:
\begin{equation}
\chi_h = -\frac{e^2 \lambda^2}{m \pi c^2} \frac{1}{V} \sum_{n} f_n \cdot e^{-2\pi i\vec{h}\cdot \vec{r}_n}
\end{equation}
\noindent
где, $h$ - соответсвует какому - либо направлению вектора обратной решетки, для конкретных {hkl}.

На данном этапе мы не рассматриваем возможность распространение в кристалле большого количества
волн (многоволновый случай), а рассмотрим только два узла обратной решетки h - [000] и h - [hkl].
Тогда поляризуемость примет конечный вид

\begin{equation}
\chi(\vec{r}) = \chi_0 + \chi_h e^{i\vec{h}\vec{r}} + \chi_{-h} e^{-i\vec{h}\vec{r}}
\end{equation}

\textcolor{mygreen}{Если будет время, нарисовать распределение волнового поля для всех волн и их суммы}

    \subsubsection{Собственная кривая отражения (КДО)}
      
 \subsubsection{Коэффициент брегговского отражения от кристалла}
 Из системы уравнение Максвелла получим (~\ref{sec:wave_equation}) следующее волновое уравнение

\begin{equation}
 \Laplace \vec{E} - k_0^2 \vec{D} = \Laplace \vec{E} - k_0^2 (1+\chi)\vec{E} = 0
 \label{eq:wave_maxwel}
\end{equation}

как было упомянуто выше, в кристалле распространяются две волны
\begin{equation}
 \begin{cases}
   \vec{E}_0 = \vec{e}_0 E_0 e^{i\vec{q}_0\vec{r}}
   \\
   \vec{E}_h = \vec{e}_h E_h e^{i\vec{q}_h\vec{r}}
 \end{cases}
 \label{eq:E_0_E_h}
\end{equation}

где,

\begin{equation}
\vec{e}_0 \cdot \vec{e}_h = C
 \begin{cases}
   1, \quad \quad \quad \quad  \sigma    - \text{поляризация}\\
   \cos(2\theta_B), \quad   \pi - \text{поляризация}
 \end{cases}
\end{equation}

\begin{figure}[H]
  \centering
  \includegraphics[width=0.7\textwidth]{images/polarize_E.png}
  \caption{ Колебание вектора напряженности электрического поля для разных типов линейной поляризации рентгеновского излучения}
  \label{ris:polarize_E}
\end{figure}

Подставим  ~\ref{eq:E_0_E_h} в уравнение ~\ref{eq:wave_maxwel} и получим
систему динамических уравнений
\begin{equation}
 \begin{cases}
   \delta_0 E_0 - C\chi_{-h}E_h=0
   \\
   \delta_h E_h - C\chi_{h}E_0=0
 \end{cases}
\end{equation}
где,
\begin{equation}
   \delta_{(0,h)} = \frac{q_{(0,h)}^2}{k_0^2}-1-\chi_0
\end{equation}

Прировняем детерминант системы к 0, получим дисперсионное уравнение

\begin{equation}
   \delta_0 \delta_h -C^2 \chi_h \chi_{-h} = 0
\end{equation}

Воспользуемся равенством тангенциальных компонент волнового вектора при переходе между средами (\ref{eq:k_h_squred}, \ref{eq:k_0_squred}),
необходимо отметить $k_h == q_h$ - т.к при выходе излучения из среды происходит лишь преломление, суммарная интенсивность останется прежней.
\begin{equation}
 \begin{cases}
   \delta_0 = \frac{q_0^2 - k_0^2}{k_0^2} - chi_0 = 2\varepsilon\gamma_0 - \chi_0
   \\
   \delta_h = \frac{q_h^2 - k_0^2}{k_0^2} - chi_0 = 2\varepsilon\gamma_h - \alpha \chi_0
 \end{cases}
\end{equation}
где $\alpha$ соответствует выражению (\ref{eq:alpha}). Дисперсионное уравнение с учетом граничных условий примет вид

\begin{equation}
   (2\varepsilon \gamma_0 - \chi_0)(2\varepsilon \gamma_h - \alpha - \chi_0) - C^2 \chi_h\chi_{-h} = 0
\end{equation}

Решив уравнение относительно параметра аккомодации $\varepsilon$ получим два корня

\begin{equation}
   \varepsilon_{1,2} = \frac{1}{4\gamma_0} \left( \chi_0 (1-b) - b\alpha \pm \left( [\chi_0(1+b)+b\alpha]^2 - 4bC^2 \cdot \chi_{h}\chi_{-h} \right)^{1/2} \right)
\end{equation}
где b - соответсвует (\ref{eq:koef_b}),  а произведение коэффициентов поляризуемости
 $$\chi_{h} \cdot \chi_{-h} = Re(\chi_{h})^2-Im(\chi_{h})^2 - 2i \quad Re(\chi_{h}) \cdot Im(\chi_{h})$$

Наличие двух решений говорит о том, что в кристалле имеется две проходящие и две дифрагированные волны, но
анализ полученного решения $ \varepsilon_{1,2}$ показывает что один корень имеет положительную мнимую часть, а второй
отрицательную. Мнимая часть отвечает за поглощение и в случае отрицательно корня волна распространяясь
вглубь кристалла экспоненциально затухает. Поэтому будем выбирать всегда корень с отрицательной мнимой частью
$Im(\varepsilon)>0$.

Амплитудный коэффициент отражение
\begin{equation}
    R = \frac{E_0}{E_h} = \frac{\delta_0}{C\chi_{-h}} = \frac{2\varepsilon\gamma_0-\chi_0}{C\chi_{-h}}
\end{equation}

Кривая диффракционного отражения (КДО)
\begin{equation}
    \label{eq:KDO_self}
    P (\vartheta) =  |\frac{\gamma_h}{\gamma_0} \cdot R|^2
\end{equation}

    % \subsubsection{Нарушенный слой}
      % \input{broken_layer.tex}
  \subsection{Пьезоэлектрический эффект}
    \label{sec:piezo_theor}
Материалы, в которых  существует линейная связь между механическим напряжением
и электрической поляризаций (прямой пьезоэлектрический эффект) или между
механической деформацией и приложенным электрическим полем
(обратный пьезоэлектрический эффект), называются пьезоэлектриками.
%
% \begin{figure}[H]
%   \centering
%   \subfloat[]{\includegraphics[width=0.4\textwidth]{images/piezo_ther_1.png}}
%   \hfill
%   \subfloat[]{\includegraphics[width=0.45\textwidth]{images/piezo_ther_2.png}}
%   \caption{Свойство пьезоэлектрического кристалла в его простейшем виде для (a)
%    прямого и (b) обратного пьезоэлектрического эффекта}
%   \label{ris:piezo_is}
% \end{figure}

Согласно определению обратного пьезоэлектрического эффекта,
приложенное внешнее электрическое поле $\vec{E}$ является причиной возникновения
в кристаллическом материале деформаций $r_i$. Вектор деформаций пропорционален
величине приложенного напряжения и зависит от пьезоэлектрических свойств материала
в данном направлении. Модуль пьезоэлектрических деформаций $d$ является матрицей,
 с размерностью (3х6) \cite{kedi_1949,Newnham_2005}.

\begin{equation}
  r_j = d_{ij}E_i,
  \label{eq:piezomodule}
\end{equation}
где $i = (1,2,3) = (x,y,z)$, $j = (1,2,3,4,5,6)$, $E_i$ - компонента напряженности электрического поля.

Исходя из уравнения (\ref{eq:piezomodule}) можно судить о том, что поле, приложенное в каком-либо из
направлений, может вызывать деформацию кристалла в любом направлении с коэффициентом пропорциональности $d_{ij}$.
Компоненты деформации  $r_1$, $r_2$ ...$r_6$ можно также обозначать через $x_x$, $y_y$, $z_z$,
$y_z$, $z_x$ и $x_y$ (обозначения Кирхгофа)  \cite{kedi_1949}.

 Например, компонента растяжения/сжатия $r_1= x_x $ соответствует относительному изменения
 длины вдоль данного направления $\frac{\Delta x}{x_0}$ (рис. \ref{ris:deform_piezo}b), а
 компонента сдвиговой деформации соответствует отношению $r_6= x_y = \frac{\Delta x}{y_0}$,
 как показано на рис. \ref{ris:deform_piezo}a.
\begin{figure}[H]
  \centering
  \subfloat[]{\includegraphics[width=0.45\textwidth]{images/x_x_3d.png}}
  \hfill
  \subfloat[]{\includegraphics[width=0.45\textwidth]{images/x_y_3d.png}}
  \caption{Схематичное изображение деформации
  растяжения/сжатия $r_1= x_x = \frac{\Delta x}{x_0}$ (a),
   сдвига  $r_6= x_y = \frac{\Delta x}{y_0} $ (b)  }
  \label{ris:deform_piezo}
\end{figure}

Деформации вида $x_y$ и $y_x$ являются связанными и выражаются друг через друга.
На рис. \ref{ris:2x_y_1} представлен пример преобразования сдвиговой деформации
для $x_y = y_x$, такую деформацию описывают только одной компонентой.
С математической точки зрения отождествление $x_y$ с $y_x$
уменьшает число независимых компонент общего тенора деформаций.

\begin{figure}[H]
  \centering
  \subfloat[]{\includegraphics[width=0.45\textwidth]{images/2x_y_1.png}}
  \hfill
  \subfloat[]{\includegraphics[width=0.45\textwidth]{images/2x_y_2.png}}
  \caption{Переход от двух равных между собой сдвиговых деформации
  $x_y = y_x$ (a) к одной $x_y=\frac{2\Delta x}{y_0}$ (b) с учетом поворота
  поворота образца}
  \label{ris:2x_y_1}
\end{figure}

В развернутой форме выражение (\ref{eq:piezomodule}) выглядит следующим образом:

\begin{equation}
  \begin{pmatrix}
  r_1 \\
  r_2 \\
  r_3 \\
  r_4 \\
  r_5 \\
  r_6
  \end{pmatrix}
   = \begin{pmatrix}
  d_{11} & d_{21}  & d_{31} \\
  d_{12} & d_{22}  & d_{32} \\
  d_{13} & d_{23}  & d_{33} \\
  d_{14} & d_{24}  & d_{34} \\
  d_{15} & d_{25}  & d_{35} \\
  d_{16} & d_{26}  & d_{36}
  \end{pmatrix}
  \begin{pmatrix}
  E_1 \\
  E_2 \\
  E_3
  \end{pmatrix}
  \label{eq:piezomodule_matrica}
\end{equation}

В общем случае все 18 пьезомодулей не зависимы друг от друга. Однако,
под действием операции симметрии кристалл должен полностью совместиться с
самим собой и это касается не только его строения, но и любого физического свойства.
Исходя из принципа Неймана \cite{Shaskolska_1984} физические свойства
 по кристаллографически эквивалентным направлениям должны быть одинаковыми.
Таким образом, пьезоэлектрический эффект может возникнуть только в кристаллах, лишенных центра
симметрии. В 11 из 32 классах точеной группы симметрии нет полярных направлений,
а значит в кристаллах этих классов не может возникать пьезоэффект. Для остальных
классов некоторые пьезомодули могут обратиться в нуль из-за наличия симметрии.
При этом, чем выше симметрия, тем меньше число независимых пьезомодулей.


% ---------------------section 2 -----------------------
\newpage
\section{Оборудование и методы}
  \subsection{Оборудование. Трехкристальный рентгеновский спектрометр}
    Экспериментальная апробация разработанных алгоритмов производилась на лабораторном источнике
рентгеновского излучения - трехкристальном рентгеновском спектрометре (ТРС) (рис. \ref{ris:trs}).
ТРС включает в себя рентгеновский источник, спектр которого является характеристическим и
определяется материалом анода, в случае настоящей работы молибденом. Рентгеновские
лучи от неподвижного источника, падают на кристалл-монохроматор, где происходит
пространственное разделение спектра и монохроматизация пучка. Коллимационная щель №1
вырезает необходимую спектральную составляющую, которая затем отражается от исследуемого кристалла-образца.

\begin{figure}[H]
  \centering
  \includegraphics[width=1\textwidth]{images/trs.png}
  \caption{ Трехкристальный рентгеновский спектрометр. Лаборатория рентгеновских
  методов анализа и синхратронного излучения, ФНИЦ "Кристаллография и фотоника"}
  \label{ris:trs}
\end{figure}

Конструкция ТРС предусматривает возможность работать как в режиме двухкристального эксперимента,
когда перед детектором устанавливается коллимационная щель №2,
так и в режиме трехкристального эксперимента, когда на место
 перед детектором устанавливается кристалл-анализатор и отраженный от анализатора луч
фиксируется детектором.

Оптическое расстояние от источника до коллимационной щели №1 и №2 составляет 570 мм
и 1005 мм соответсвенно. Кристалл-образец устанавливается на многокружный гониометр,
ось которого расположена на расстоянии до детектора, равном 210 мм,
позволяющим осуществлять позиционирование образца и детекторов
 с точностью 0.5 угл. сек. Также ТРС оснащен двумя сцинтилляционнымы детекторами для
проведения экспериментов методами многоволновой и квазимноговолновой дифракции.

  \subsection{Исследуемые образцы}
        \label{sec:piezo_matrix}

  \subsubsection*{ Кристалл Si }
  Для проведения экспериментов, необходимых для апробации разрабатываемых алгоритмов,
  был использован монокристалл кремния (Si). Данный
  кристалл характеризуется идеальной кривой собственного отражения, поэтому
  был использован в качестве модельного образца. Решетка Si имеет кубическую симметрию
  с параметрами элементарной ячейки  $a = b = c = 5.4310 \angstrom$.

  \subsubsection*{ Кристалл LGT }
  Кристаллы семейства лантан-галлиевого силиката ($La_3Ga_5SiO_{14}$ - LGS и $La_3Ga_{5.5}Ta_{0.5}O_{14}$ - LGT)
  обладают пьезоэлектрическими свойствами со стабильной температурной зависимостью
  даже при высоких температурах. Пьзоэлектрический модуль $d_{11}$ остается
  постоянным в диапазоне температур до $600^o$С (изменение не более 5 $\%$ \cite{LGS58}).
  В таких кристаллах отсутствует фазовый переход вплоть до температур плавления \cite{LGS57},
   а также не имеется пироэлектрического эффекта.

  LGT кристалл имеет точечную группу симметрии $32$ и гексагональную сингонию.
  Матрица пьезоэлектрических модулей выглядит следующим образом:
  \begin{equation}
    \begin{pmatrix}
    d_{11} & -d_{11} & 0 & d_{14} & 0 & 0 \\
    0 & 0 & 0 & 0 & -d_{14} & 2d_{11} \\
    0 & 0 & 0 & 0 & 0 & 0
    \end{pmatrix}.
    \label{eq:piezomodule_lgt_matrica}
  \end{equation}
  Параметры ячейки: $a = b = 8.228 \angstrom$, $c = 5.124 \angstrom$ \cite{marchenkov2014}.
  %
  % \subsection*{ Кристалл $TeO_2$ }
  % ...
  % \subsection*{ Кристалл $Li_2B_4O_7$ }
  % ...

  \subsection{Алгоритмы расчетов и методики измерений}
    Стоит отметить, что представление параметров рентгеновского пучка
 в виде двумерных спектрально-угловых карт (рис. \ref{ris:source_distrubition})
 будет использовано далее во всей работе. Это позволяет не только наглядно отслеживать
характеристики излучения после прохождения каждого элемента рентгенооптической
схемы, но и проводить моделирование как любых типов источников, так
и для разных элементов схемы. Такой подход основывается на представлении впервые предложенным
ДюМондом \cite{Dumond1937}.

    \subsubsection{Функция источника}
      \label{sec:source_section}
Спектр рентгеновской трубки является характеристическим, спектральная часть
 которого достаточно хорошо описывается двумя функциями Лоренца взятыми с
 весовыми коэффициентами (\ref{eq:source_spectral}):

 \begin{equation} \label{eq:source_spectral}
   g_{\lambda} (\lambda) = \frac{2\pi}{3}  \left \{ \frac{\delta\lambda_1}{(\lambda - \lambda_1)^2+
   (\delta \lambda_1)^2} + \frac{1}{2} \frac{\delta\lambda_2}{(\lambda-\lambda_1)^2+(\delta\lambda_1)^2} \right \}
  \end{equation}

  Плотность распределения количества потока электромагнитного излучения в зависимости от угла
  отстройки относительно прямолинейного распределения задается функцией Гаусса (\ref{eq:source_angle}):
\begin{equation} \label{eq:source_angle}
  g_{\vartheta} (\vartheta) = \frac{1}{\sigma \sqrt{ 2\pi}} exp  ( -\frac{\vartheta^2}{2\sigma^2} )
 \end{equation}
\noindent
где $\sigma$ - параметр, который характеризует ширину углового распределения на половине высоты.

\begin{figure}[H]
  \centering
  \includegraphics[width=0.6\textwidth]{images/source_distrubition.png}
  \caption{Спектрально – угловое распределение лабораторного источника рентгеновского
   излучения с молибденовым анодом, угловая полуширина распределения составляет $\sigma = 600$ угл. сек. }
  \label{ris:source_distrubition}
\end{figure}

    \subsubsection{Функция щелевых коллиматоров}
      \label{sec:slits_section}
Другой составляющей аппаратной функции является функция углового распределения излучения в
экспериментальной схеме, определяемая геометрическими особенностями (размерами щелевых
коллиматоров и длинами оптических путей). Рассмотрим преобразование пучка рентгеновского
 излучения проходящего через систему щелевых коллиматоров.
\begin{figure}[H]
  \centering
  \includegraphics[width=0.6\textwidth]{images/for_slits.png}
  \caption{Схематичное изображение распространение рентгеновского пучка в
  системе с протяженным источником и двумя щелевыми коллиматорами}
  \label{ris:for_slits}
\end{figure}

На начальном этапе рассматривалась модель точечного источника излучения
 (протяженность источника $\delta = 0$).
В таком случае, интенсивность проходящего излучения будет определятся
одним щелевым коллиматором, которое является более узким из двух при пересчете в угловые
координаты. Например, для фиксированных расстояний между элементами $L_1 = 570$ мм, $L_2 = 1005$ мм,
 в случае одинаковых линейных размеров щелей и точечного
источника, интенсивность будет определяться более удаленным щелевым коллиматором, и
распределение интенсивности принимает вид представленный на рис. \ref{ris:sourc_map_a}. Если источник является
протяженным ($\delta \neq 0$), то угловое распределение интенсивности принимает более сложный вид,
 как показано на рис. \ref{ris:sourc_map_b}.


\begin{figure}[H]
  \centering
  \subfloat[]{\includegraphics[width=0.5\textwidth]{images/point_sourc_map.png}\label{ris:sourc_map_a}}
  \hfill
  \subfloat[]{\includegraphics[width=0.5\textwidth]{images/wide_sourc_map.png}\label{ris:sourc_map_b}}
  \caption{Спектрально-угловое распределение излучения в системе двух щелей для различных типов источника: точечного (a)
  и протяженного ($\delta = 0.2$ мм) (b)}
  \label{ris:sourc_map}
\end{figure}

Необходимо отметить, что для описания дифракционного эксперимента имеет значение именно
спектрально-угловое распределение излучения, т.е. количество и энергия квантов, падающих под тем
или иным углом на кристалл. Данное распределение определяется соотношение площадей параллелограммов,
угол между боковой стороной и основанием которых соответсвует углу распространения излучения
(рис. \ref{ris:how_many_quants_use_parallelogr}).

\begin{figure}[H]
  \centering
  \subfloat[]{\includegraphics[width=0.45\textwidth]{images/how_many_quants_use_parallelogr_1.png}}
  \hfill
  \subfloat[]{\includegraphics[width=0.45\textwidth]{images/how_many_quants_use_parallelogr_2.png}}
  \caption{Схематичное представление углового распределения излучения после
  прохождения системы щелевых коллиматоров. Пропускная способность системы
  щелей в определенном угловом направлении соответствующего параллелограмма (a).
   Интенсивность на экране, установленном после системы щелей для
   протяженности источника $\delta = 0.2$ мм (b)}
  \label{ris:how_many_quants_use_parallelogr}
\end{figure}
Более подробный расчет $g_S(\vartheta)$ представлен в Приложении 3.
На рис. \ref{ris:calc_slits_ability_res} представлены результаты расчета пропускной способности
системы двух щелей для некоторых параметров ренгенооптической схемы в приближении
 точечного источника ($\delta = 0$), в сравнении с протяженным ($\delta \neq 0$).

\begin{figure}[H]
 \centering
 \subfloat[]{\includegraphics[height=6em]{images/calc_slits_ability_res_1.png}}
 \hfill
 \subfloat[]{\includegraphics[height=6em]{images/calc_slits_ability_res_2.png}}
 \hfill
 \subfloat[]{\includegraphics[height=6em]{images/calc_slits_ability_res_3.png}}
 \hfill
 \subfloat[]{\includegraphics[height=6em]{images/calc_slits_ability_res_4.png}}
 \caption{Пропускная способность щелевых устройств в зависимости от угла распространения
 рентгеновского излучения. Расстояние до первой щели $L_1 = 570$ мм, до второй - $L_2 = 1005 $ мм.
 Размеры щелевых коллиматоров и протяженность источника:
   $S_1 = S_2 = 50$ мкм; $\delta = 0.2$ мм (a),
   $S_1 = 20$ мкм; $S_2 = 40$ мкм; $\delta = 0.2$ мм (b),
   $S_1 = 200$ мкм; $S_2 = 400$ мкм; $\delta = 0.2$ мм (c),
   $S_1 = 200$ мкм; $S_2 = 400$ мкм; $\delta = 0.1$ мм (d)}
 \label{ris:calc_slits_ability_res}
\end{figure}

Анализ рис. \ref{ris:calc_slits_ability_res} показывает, что перегиб  возникает вследствие переходного
процесса от точечного источника к бесконечному, т.е. на меньших углах плотность
излучения определяется ближайшей, а начиная с некоторого угла ограничивать пучок начинает
более удаленная щель.
 %\textcolor{mygreen}{НАРИСОВАТЬ РИСУНОК}(рис. ).

    % \subsubsubsection{Дифракция на щели}  Cowley1979ru на странице 47
    \subsubsection{Собственная кривая отражения}
      \label{sec:rocking_curve_section}

Определяемая формулой динамической дифракции, форма кривой дифракционного отражения
представляет собой узкую линию c полушириной порядка нескольких угловых секунд
(рис. \ref{ris:darwin_methody}).


\begin{figure}[H]
  \centering
  \subfloat[]{\includegraphics[width=0.45\textwidth]{images/darwin_lgt.png}\label{ris:darwin_lgt}}
  \hfill
  \subfloat[]{\includegraphics[width=0.45\textwidth]{images/darwin_si.png}\label{ris:darwin_si}}
  \caption{Собственная кривая отражения от кристалла Si(220) и
  кристалла LGT(220) для $MoK_{\alpha 1}$ - излучения}
  \label{ris:darwin_methody}
\end{figure}

В дальнейшем рассмотрении на спектрально-угловой карте будет присутствовать
полоса отражения от кристалла, которая представляет из себя набор кривых с
разными углами Брэгга (рис. \ref{ris:darwin_lambda}).
Ширина полос на двумерной карте определяется полушириной (FWHM)
собственной кривой отражения.

\begin{figure}[H]
\centering
\includegraphics[width=0.7\textwidth]{images/darwin_lambda.png}
\caption{Полоса собственной КДО на спектрально-угловом распределение для кристалла
кремния  Si(220) для $MoK_{\alpha}$ - излучения }
\label{ris:darwin_lambda}
\end{figure}

Весьма наглядной иллюстрацией влияния асимметрии являются собственные
кривые отражения от Si(440) рассчитанные при
трех разных углах падения и соответсвенно имеющие разный коэффициент асимметрии. Угол
Брэгга для такой плоскости отражения составляет $\theta_B = 21.68^o$, угол наклона поверхности
составляет $\varphi = 20^o 53^{'}$.

\begin{figure}[H]
\centering
\includegraphics[width=0.99\textwidth]{images/rocking_curve_assym_3.png}
\caption{Кривые отражения от Si(440) $MoK_{\alpha 1}$ - излучения, полученные при разных углах падения(для разных b)}
\label{ris:rocking_curve_assym_3}
\end{figure}
Сдвиг центра кривой происходит из-за наличия преломления на величину 0.5 и 16.5 угловых секунд.
%
% Варьируя угол между поверхностью кристалла и отражающей плоскостью (например, с помощью шлифовки),
% можно существенно изменить ширину рентгеновского пучка (рис. ~\ref{ris:assym_width_beam}).
% \begin{figure}[H]
%  \centering
%  \includegraphics[width=0.4\textwidth]{images/assym_width_beam.png}
%  \caption{Кристалл с асимметричным отражением по Брэггу}
%  \label{ris:assym_width_beam}
% \end{figure}

    \subsubsection{Отражение от одного кристалла}
      \label{sec:single_crystal_section}
  Постепенно будем наполнять схему и внесем один идеальный кристаллический элемент.
  Кристалл регламентируется уже не только угловой составляющей пучка, но и берет в учет энергию.

  Спектрально-угловое распределение после отражающего кристаллического элемента задается выражением
  \begin{equation} \label{eq:monochromator_spectra}
    P(\vartheta,\lambda) = g_{\lambda}(\lambda)g_{\vartheta}(\vartheta) P(\vartheta - \frac{\lambda - \lambda_1}{\lambda_1}\tan(\theta_B))
   \end{equation}
где $P$ - соответствует (\ref{eq:KDO_self}), $\lambda_1$ - длина волны излучения от которой ведется отсчет углов $\vartheta$.
\begin{figure}[H]
  \centering
  \subfloat[]{\includegraphics[width=0.45\textwidth]{images/single_crystal_schem.png}\label{ris:single_crystal_schem_lamtet_a}}
  \hfill
  \subfloat[]{\includegraphics[width=0.45\textwidth]{images/single_crystal_schem_lamtet.png}}

  \caption{(a) Схема однокристального эксперимента. (b) Спектрально угловое распределение после
   отражения расходящегося, полихромотического пучка от кристалла Si(220),
   положение щелевых устройств обозначено синей линией вблизи $\pm 20$ угл.сек.}
  \label{ris:single_crystal_schem_lamtet}
\end{figure}

На рис. \ref{ris:single_crystal_schem_lamtet}, по своей сути, изображен принцип работы монохроматора,
когда после взаимодействия с кристаллом, разные длины волн отражаются под разными углами в
соответсвии с законом Брэгга.

Кривая отражения в однокристальном эксперименте (рис. \ref{ris:single_crystal_schem_lamtet_a}), в котором сканирование осуществляется
 с помощью детектора, жестко связанного с щелевым устройством, линейного размера S, находящегося на расстоянии $L$ от источника, задается следующим образом

\begin{equation} \label{eq:p_single_crystal}
  P_{single}(\theta) = \sum_{\lambda = -\infty}^{\infty}g_{\lambda}(\lambda) \cdot \sum_{\vartheta = \vartheta_{s1}}^{\vartheta_{s2}}
  g_{\vartheta}(\vartheta) P_M(\vartheta - \frac{\lambda - \lambda_1}{\lambda_1}\tan(\theta_B))
 \end{equation}
\noindent
где $\vartheta$ - угол падения излучения на кристалл, в случае не расходящегося пучка $\vartheta = 0$, в
случае, например, синхротронного источника $\vartheta \in (-6^o; 6^o) $; $g_{\lambda}(\lambda)$
- спектральная плотность распределения пучка (\ref{eq:source_spectral}); $g_{\vartheta}(\vartheta)$ - угловая плотность
распределения пучка (\ref{eq:source_angle}); $P_M$ - коэффициент отражения от неподвижного кристалла, далее мы будем его называть монохроматором,
слагаемое $\frac{\lambda - \lambda_1}{\lambda_1}\tan(\theta_B)$ -
возникает из условия Брэгга и говорит о том, что разные длины волн отражаются под разными углами.
 Суммирование проводится, во-первых, вдоль угловой апертуры детектора, которая задается размером
 щелевого коллиматора перед ним, а пределы определяются исходя из ее углового положения $\theta$ относительно
 оптической оси (зеркально отраженного луча) $\vartheta_{s1} = \theta - \frac{S}{2L}$, $\vartheta_{s2} = \theta + \frac{S}{2L}$,
 $S $ - линейный размер щелевого устройства, $L$ - расстояние от источника до щели.
 Во-вторых, суммирование осуществляется по всем $\lambda$, из-за свойства детектора не различать разные длины волн.
На рис. \ref{ris:zero_exp} приведен результат сканирования расходящегося пучка от рентгеновской
трубки после отражения от неподвижного кристалла кремния Si(220) для разных размеров щелевого коллиматора
в сравнении с расчетными.

    \subsubsection{Методика моделирования двухкристальных кривых дифракционного отражения}
      Метод анализа КДО по прежнему являются одним из основных инструментов диагностики не только совершенства
кристаллических материалов \cite{sov_1} - \cite{sov_5}, в частности, объемных и поверхностных дефектов в
монокристаллах, тонких пленках, а также многослойных кристаллических структурах, но и для анализа физических
процессов происходящих в кристаллах, таких как воздействие внешнего электрического поля \cite{piezo102} (пьезоэлектрический эффект),
 температуры \cite{temp} или влияние магнитного поля.

Измерение кривой дифракционного отражения в двухкристальной схеме представляет
собой измерение зависимости отраженного образцом рентгеновского излучения при
пошаговом повороте исследуемого кристалла относительно падающего на него
излучения в окрестности точного значения угла Брэгга.
Существует несколько схем измерения кривых отражения рентгеновского излучения.

\subsubsection*{$\omega$ - сканирование}
В данном типе сканирования кривая отражения измеряется путем поворота образца
относительно падающего пучка в плоскости дифракции. При таком сканировании
угол между падающим и дифрагированным пучками (угол рассеяния) остается постоянным
(рис. \ref{ris:omega_scan}). Получаемая в результате кривая носит название кривой качания.


\begin{figure}[H]
  \centering
  \includegraphics[width=1\textwidth]{images/omega_scan.png}
  \caption{Схема реализации $\omega $ - сканирования}
  \label{ris:omega_scan}
\end{figure}

\subsubsection*{$\vartheta - 2\vartheta$ - сканирование}
В отличие от предыдущего, данный метод сканирования соответствует изменению
 модуля вектора рассеяния при неизменном его угловом положении
 (рис. \ref{ris:theta_2theta_scan}). Угловое положение падающего пучка и
 детектора изменяется синхронно и симметрично относительно используемой системы
 атомных плоскостей, а установленная перед детектором апертурная щель вырезает
  только зеркально отраженную часть пучка. Именно поэтому при построении карт
   пространственного распределения спектра полосы щелей на этих картах остаются
   неподвижными (т.к. несмотря на движение щели  $S_2$ в процессе
    $\vartheta - 2\vartheta$ -  сканирования ее отстройка от зеркального
    положения всегда равна 0).

 \begin{figure}[H]
   \centering
   \includegraphics[width=1\textwidth]{images/theta_2theta_scan.png}
   \caption{Схема реализации $\vartheta - 2\vartheta$ - сканирования}
   \label{ris:theta_2theta_scan}
 \end{figure}
Кроме того, используемый подход основанный на спектрально-угловом представлении
для данного типа сканирования, наглядно демонстрирует интересный эффект.
 Независимо от ширины входной и приемной щелей характеристическая линия спектра
 трубки $k_{\alpha 2}$ всегда вносит вклад в КДО, проявляясь в виде дополнительного
 пика на ее хвосте, что будет показано далее.

    \subsubsubsection*{Выражение для расчета двухкристальных КДО}
      
Для того, чтобы разобраться в том, как формируются экспериментальные
двухкристальные КДО, нам необходимо построить спектрально-угловое распределение
в соответствии со схемой эксперимента (рисунок \ref{ris:double_crystal_schem_lamtet_a}).

\begin{eqnarray} \label{eq:doudle_spectra_angle_map}
  P(\theta,\vartheta,\lambda) = g_{\lambda}(\lambda)g_{\vartheta}(\vartheta) P_M \left(\vartheta - \frac{\lambda - \lambda_1}{\lambda_1}\tan(\theta_B) \right) \cdot \nonumber \\
   P_S \left(\theta + \vartheta - \frac{\lambda - \lambda_1}{\lambda_1}\tan(\theta_B)\right)
 \end{eqnarray}

Выражение (\ref{eq:doudle_spectra_angle_map}) определяет спектрально угловое распределение после прохождения двух кристаллов с
коэффициентами отражения  $P_M$ (монохроматор) и $P_S$ (образец), причем последний принимает во внимание положение угла отстройки $\theta$ относительно
точного Брегга (рисунок \ref{ris:double_crystal_schem_lamtet_b}).

\begin{figure}[H]
  \centering
  \subfloat[Схема двухкристального эксперимента]{\includegraphics[width=0.5\textwidth]{images/double_crystal_schem.png}\label{ris:double_crystal_schem_lamtet_a}}
  \hfill
  \subfloat[Спектрально угловое распределение. Положение щелевых устройств обозначено синей и белой линиями вблизи $\vartheta = 0$ угл.сек. Кристалл-образец
  выведен из точного Брегговского положения на 300 угл. сек.
   ]{\includegraphics[width=0.45\textwidth]{images/double_crystal_schem_lamtet.png} \label{ris:double_crystal_schem_lamtet_b}}

  \caption{Схема и спектрально-угловое распределение после отражения расходящегося, полихромотического пучка. Несмотря на то, что в
  экспериментальной схеме детектор со щелью не стоят на месте, на карте обе щели $S_1$ и $S_2$ - неподвижны. }
  \label{ris:double_crystal_schem_lamtet}
\end{figure}

Выражение (\ref{eq:doudle_spectra_angle_map}) не учитывает особенности особенности влияния щелевых коллиматоров, о которым мы говорили в
(раздел \ref{sec:slits_section}), а так же тот факт, что детектор не разделят энергетическую составляющую пучка.


\begin{eqnarray} \label{eq:doudle_spectra_angle_map_on_detector}
  P_{double}(\theta) = \sum_{\lambda = -\infty}^{\infty}g_{\lambda}(\lambda)\cdot
  \sum_{\vartheta = \vartheta_{s1}}^{\vartheta_{s2}} g_{\vartheta}(\vartheta) g_{S}(\vartheta) \cdot \nonumber \\
   P_M \left(\vartheta - \frac{\lambda - \lambda_1}{\lambda_1}\tan(\theta_B) \right) \cdot \nonumber \\
   P_S \left(\theta + \vartheta - \frac{\lambda - \lambda_1}{\lambda_1}\tan(\theta_B)\right)
 \end{eqnarray}
 где пределы суммирования определяются как $\vartheta_{s2} = - \vartheta_{s1} = \frac{\delta+S_1}{2L_1}$.

 \begin{figure}[H]
   \centering
   \includegraphics[width=1\textwidth]{images/double_crystal_form_kdo.png}
   \caption{Формирование двухкристальной кривой дифракционного отражения ($\theta - 2\theta$ - cканирование),
   $\theta_B^M = \theta_B^S = 10.6^o$  }
   \label{ris:double_crystal_form_kdo}
 \end{figure}

В том случае, если схема дисперсионная т.е. углол Брегга кристалла - образца отличен от угла Брега кристалла-монохроматора,
наблюдается уширения двухкристальных кривых (рисунок \ref{ris:double_crystal_form_kdo_dissp}).
 \begin{figure}[H]
   \centering
   \includegraphics[width=1\textwidth]{images/double_crystal_form_kdo_dissp.png}
   \caption{Формирование двухкристальной кривой дифракционного отражения ($\theta - 2\theta$ - cканирование) в случае
   наличия дисперсии, $\theta_B^M = 10.6^o$, $\theta_B^S = 21.6^o$ }
   \label{ris:double_crystal_form_kdo_dissp}
 \end{figure}

    \subsubsection{Методика моделирования трехкристальных кривых дифракционного отражения}
      \subsubsubsection{Карта рассеяния в прямом пространстве}
        
\subsubsection{Карта рассеяния в прямом пространстве}

\begin{figure}[H]
  \centering
  \includegraphics[width=0.8\textwidth]{images/triple_crystal_schem.png}
  \caption{Схема трехкристального эксперимента, $\theta$ - отстройка образца от точного угла Брегга,
  $\epsilon$ - угол анализатора относительно зеркально отраженного пучка образцом}
  \label{ris:}
\end{figure}

\begin{eqnarray} \label{eq:doudle_spectra_angle_map}
  P_{triple}(\theta,\varepsilon) = \sum_{\lambda = -\infty}^{\infty}g_{\lambda}(\lambda)\cdot
  \sum_{\vartheta = \vartheta_{s1}}^{\vartheta_{s2}} \Bigg[ g_{\vartheta}(\vartheta) g_{S}(\vartheta) \cdot \nonumber \\
    P_M \left(\vartheta - \frac{\lambda - \lambda_1}{\lambda_1}\tan(\theta_B) \right) \cdot \nonumber \\
   P_S \left(\theta + \vartheta - \frac{\lambda - \lambda_1}{\lambda_1}\tan(\theta_B)\right)  \cdot  \nonumber \\
   P_A \left(2\theta - \varepsilon + \vartheta - \frac{\lambda - \lambda_1}{\lambda_1}\tan(\theta_B)\right) \Bigg]
 \end{eqnarray}

      \subsubsubsection{Карта рассеяния в обратном пространстве}
        Удобным для интерпретации является построение трехкристальных карт в обратном пространстве.
Переход в обратное пространство позволяет исключить из рассмотрения особенности
конструкции дифрактометра и типов сканирований, проводимых в эксперименте.
 Угловые положения падающего и дифрагированного пучков
определяют вектор рассеяния $\vec{q}$. Такой вектор можно разложить на составляющие:
$q_z$ - вертикальную составляющую,
направленную перпендикулярно к отражающей атомной плоскости и $q_x$ - горизонтальную составляющую,
лежащую в отражающей плоскости (рис. \ref{ris:q_vector_reciprocal_space}).

\begin{figure}[H]
  \centering
  \subfloat[]{\includegraphics[width=0.3\textwidth]{images/q_vector/0.png}}
  \hfill
  \subfloat[]{\includegraphics[width=0.3\textwidth]{images/q_vector/_1.png}}
  \hfill
  \subfloat[]{\includegraphics[width=0.3\textwidth]{images/q_vector/1.png}}
  \caption{Отклонение вектора обратной решетки от соответствующего идеальному кристаллу (a)
  при деформации кристаллической решетки (b) и угловой разориентации отражающих плоскостей (с) }
  \label{ris:q_vector_reciprocal_space}
\end{figure}

Для симметричного отражения параметры $q_x$ и $q_z$ связаны с отклонением образца $\theta$ и
анализатора $\varepsilon$ от точного брэгговского положения следующими
уравнениями \cite{Tanner_1998}:

\begin{equation}
  q_x = \frac{\varepsilon}{|\vec{k}_0|} \cos \theta_B,
  \label{eq:qx_eqn}
\end{equation}

\begin{equation}
  q_z = \frac{2\theta - \varepsilon}{|\vec{k}_0|} \sin \theta_B.
  \label{eq:qz_eqn}
\end{equation}

Таким образом, сканирование образцом ($\omega$ - сканирование) влияет только на $q_x$, а
сканирование анализатором ($2\theta$ - сканирование) влияет на обе компоненты,
 изменение только одного $q_z$ достигается за счет $\theta-2\theta$ сканирования.

\begin{figure}[H]
  \centering
  \includegraphics[width=0.6\textwidth]{images/triple_map_reciprocal_space.png}
  \caption{Карта рассеяния в обратном пространстве}
  \label{ris:triple_map_reciprocal_space}
\end{figure}

Углы между ППА, ГП и ППМ определяются исходя из соотношений (\ref{eq:qx_eqn}, \ref{eq:qz_eqn}) и равны углу Брэгга образца:
% только ли образца

\begin{equation}
  \frac{q_y}{q_z} = \frac{2\theta - \varepsilon}{\varepsilon} \cdot \tan (\theta_B) = \pm \tan (\theta_B).
  \label{eq:qz_eqn}
\end{equation}

    \subsubsection{Методика расчета пьзоэлектрических констант по данным дифракции}
      \label{sec:pieao_method}
Согласно сказанному в (\ref{sec:piezo_theor}), в определенных кристаллографических направлениях в
условия воздействия внешнего электрического поля, будет возникать деформация
сжатия или растяжения. Этим деформациям соответсвует изменение межплоскостных
 расстояний, которое может быть измерено с помощью дифракции рентгеновского
 излучения по изменению угла брэгговского пика \cite{marchenkov2014}.

 \begin{figure}[H]
   \centering
   \includegraphics[width=.4\textwidth]{images/piezo.png}
   \caption{Приложенное электрическое поле к $x$ - срезу образца}
   \label{ris:x_cut}
 \end{figure}

Исходя из закона Вульфа - Брэгга, если межплоскостное расстояние получило приращение
$\Delta d$, тогда изменение угла отражения $\Delta \theta$ составит:

$$ \Delta d = \frac{\lambda}{2}\left( \frac{1}{\sin(\theta_B + \Delta \theta) } - \frac{1}{\sin \theta_B } \right) $$

\begin{equation}
   \Delta \theta =-  \frac{\tan \theta_B}{\frac{d}{\Delta d}+1}  = -  \frac{\Delta d }{d}  \tan \theta_B
\end{equation}
\noindent
где $\Delta d/d = r$ - изменение межплоскостного расстояния. Таким образом, учитывая связь с
(\ref{eq:piezomodule}),  в кристалле толщиной $L$ и разностью потенциалов на его
гранях $V$ напряженность электрического поля составляет $E = \frac{V}{L}$, а модуль
 рассчитывается исходя из следующего выражения:
 \begin{equation}
    \frac{\Delta d}{d}  = -\frac{\Delta \theta \cdot L}{V \tan \theta_B}
    \label{eq:piezomodule_l}
 \end{equation}

Выражение (\ref{eq:piezomodule_l}) было использовано в следующих работах
\cite{kibalin2015, marchenkov2014,piezo101,piezo102} для
пересчета отстройки брэгговского максимума в величину пьезоэлектрического модуля.
Следует отметить, такой подход не является общим и имеет существенные ограничения при
измерении, например, сдвиговых пьезомодулей, а также в том случае если параметр решетки
в направлении деформации определяется величиной более чем одного пьезомодуля $d_{ij}$.

На рис. \ref{ris:piezo_deformation_general} изображена элементарная ячейка
на которой обозначены кристаллографические направления по отношению к декартовой
системе координат.

\begin{figure}[H]
  \centering
  \includegraphics[width=.6\textwidth]{images/piezo_deformation_general.png}
  \caption{Триклинная ячейка в декартовой система координат}
  \label{ris:piezo_deformation_general}
\label{ris:}
\end{figure}



Рассмотрим деформационное поведение элементарной ячейки при работе пьезоэлектрического
эффект в частном случае для кристалла LGT, в котором поле приложенное вдоль направления X
вызывает деформации в направлении действия модулей $d_{11}$, $d_{12}$ и $d_{14}$ (\ref{sec:piezo_matrix}).

\begin{figure}[H]
  \centering
  \includegraphics[width=.5\textwidth]{images/d11.png}
  \caption{К объяснению действия модуля $d_{11}$ на деформацию ячейки}
  \label{ris:d11}
\end{figure}

Для модуля $d_{11}=dx/x$ имеется самая простейшая ситуация, деформация вдоль оси $x$
соответсвует изменению параметра решетки $a$ на величину $da = d_{11}\cdot a$ в расчете
на метр/вольт, если перпендикулярно оси $х$ вырезать пластинку толщиной $L$
(рис. \ref{ris:x_cut}), и на обкладки такого конденсатора подать напряжение
$V$, то "новый" параметр решетки $a{'}$ будет равен:
\begin{equation}
   a{'}  = a \left(1+\frac{d_{11}\cdot V }{L}\right)
   \label{eq:a_deformed}
\end{equation}

Для модуля $d_{12} = dy/y$, вследствие приложенного электрического поля вдоль направления $x$,
ячейка деформируется по нескольким параметрам одновременно. Происходит не только увеличение
параметра $b$, но и изменение угла $\gamma$ (рис. \ref{ris:d12}).

\begin{figure}[H]
  \centering
  \includegraphics[width=.6\textwidth]{images/d12.png}
  \caption{Вклад пьезоэлектрического модуля $d_{12}$ в деформационное поведение элементарной ячейки, вид сверху ($XY$)}
  \label{ris:d12}
\end{figure}

Исходя из теоремы косинусов получим измененный параметр $b$:
\begin{equation}
   b'^2=dy^2+b^2-2 \cdot dy \cdot b \cdot \cos(270-\gamma) \nonumber
   \label{eq:b_formed_1}
\end{equation}
\begin{equation}
   b' = b \sqrt{ 2\sin^2 \gamma \cdot d_{12}+1}
   \label{eq:b_formed_2}
\end{equation}

Из теоремы синусов следует изменение параметра $\gamma$
\begin{equation}
   \frac{\sin(270-\gamma)}{b^{'}} = \frac{\sin (\Delta \gamma)}{dy} \nonumber
   \label{eq:b_formed_3}
\end{equation}
\begin{equation}
   \gamma^{'} = \gamma + \frac{\cos\gamma \cdot \sin \gamma \cdot d_{12}}{ \sqrt{ \sin^2 \gamma  (d_{12}^2 + 2d_{12})+1}}
   \label{eq:b_formed_4}
\end{equation}


Модуль $d_{14} = dy/z$ является сдвиговым, его действие обуславливается не только
изменением угла $\beta$, но параметра $c$ (рис. \ref{ris:d14}).
\begin{figure}[H]
  \centering
  \includegraphics[width=.6\textwidth]{images/d14.png}
  \caption{К объяснению действия модуля $d_{14}$ на деформацию ячейки, вид ($YZ$)}
  \label{ris:d14}
\end{figure}
из теоремы косинусов получим зависимость величины сдвига от параметра решетки $с$,
который находится под наклоном к плоскости $YZ$ на угол $\alpha$.
$$
    dy^2 = A^2 + A{'}^2 - 2 A{'}A \cdot \cos \Delta \beta
$$
сделаем приближение по малому углу $\Delta \beta$, тогда величина сдвига будет
определятся разностью длин сторон треугольника
$$
  dy = A^{'} - A
$$
из теоремы синусов приращение угла $\beta$ задается выражением,
$$
\Delta \beta = \frac{dy \cdot\sin \beta}{ A+dy }
$$
\noindent
где
$$
dy = d_{14} \cdot c \cdot \sin \alpha \cdot \sin \beta
$$
тогда конечные выражения для измененных параметров выглядят следующим образом
\begin{equation}
   \Delta \beta = \frac{d_{14} \cdot \sin^2\beta}{1+d_{14}\sin \beta}
   \label{eq:b_formed_5}
\end{equation}
\begin{equation}
   c{'} = c(1+d_{14}\sin\beta)
   \label{eq:b_formed_6}
\end{equation}

Для того, чтобы получить значение угла смещения брэгговского максимума,
необходимо рассчитать межплоскостное расстояние для выбранных индексов отражения
до (недеформированная ячейка) и после (деформированная) приложения электрического
поля.

      \subsubsection{Методика экспериментального определения
      пльзоэлектрических констант по данным рентгеновской дифракции}
        \subsubsubsection{Статический метод}
            Метод двухкристальной дифрактометрии широко распространен для исследования пьезоэлектрических свойств \cite{piezo51} - \cite{piezo54}.
Для того, чтобы зафиксировать смещение пика двухкристальной КДО, необходимо проснять кривую дифракционного отражения
до и после приложенного напряжения (рисунок \ref{ris:piezo_classic}).
\begin{figure}[H]
  \centering
  \subfloat[В отсутствии поля]{\includegraphics[width=0.33\textwidth]{images/piezo_classic_1.png}\label{ris:piezo_classic_1}}
  \hfill
  \subfloat[Под действие элекрического поля]{\includegraphics[width=0.33\textwidth]{images/piezo_classic_2.png}\label{ris:piezo_classic_2}}
  \hfill
  \subfloat[Изменение положения максимума КДО]{\includegraphics[width=0.2\textwidth]{images/piezo_classic_3.png}\label{ris:piezo_classic_3}}

  \caption{Схематичное представление методики измерения сдвига брегговского максимума}
  \label{ris:piezo_classic}
\end{figure}
  Такой метод не позволяет отследить динамику какую - либо динамику в момент приложения электрического поля,
  т.к. время за которое получается КДО составляет десятки секунд на лабораорном источнике.

        \subsubsubsection{Времяразрешающий метод}
            Другой метод предложенный авторами работы \cite{piezo50} заключается в измерении интенсивности для разной
отстройки от точного бреговского угла кристалла образца в двухкристальной схеме дифрактометра.
Необходимо  встать в произвольную точку на кривой дифракционного отражения,
другими словами выведем интенсивность детектора из максимума отражения в точку на склоне
 кривой (Рис.~\ref{ris:kdopiez}).

\begin{figure}[H]
\centering
\includegraphics[width=1\linewidth]{images/kdopiez.eps}
\caption{Выбор точки на КДО(справа), интенсивность сигнала детектора(слева)}
\label{ris:kdopiez}
\end{figure}

При включении электрического поля (рисунок \ref{ris:princip}В) для разных направлений наблюдаем изменение интенсивности
на детекторе (рисунок \ref{ris:princip}А). Изменение интенсивности характеризует динамику смещения
двухкристальной КДО (рисунок \ref{ris:princip}C) из-за изменения межплоскостного расстояния в образце.

\begin{figure}[H]
\centering
\includegraphics[width=0.8\linewidth]{images/princip.eps}
\caption{(A) Интенсивность сигнала на детекторе; (В) величина  приложенного напряжения к
поверхности кристалла; (С) восстановленное положение КДО  }
\label{ris:princip}
\end{figure}

Данный метод является наиболее быстрым, т.к. изменение интенсивности на детекторе происходит сразу,
из рисунка видно, что время за которое деформируется кристалл много меньше разрешающей способности метода.
Но дальше будет показано, что наряду с пьезоэлектрическим эффектом могут присутствовать и сопровождающие
процессы, которые ведут к не мгновенному смещению, а протекают за конечное время.

  Горфман \cite{piezo51} усовершенствовал данную методику, добавив схему совпадения, которая позволяет
  снимать двухкристальную кривую классическим методом с увеличенной временной задержкой в каждой точке отстройки.
  Увеличение времени необходимо для измерения интенсивность до момента подачи поля и после включения
   электрического поля, включение которого согласуется с помощью высокочастотного анализатора с детектором.



% ---------------------section 3 -----------------------
\newpage
\section{Результаты и обсуждения}
  \subsection{Аппаратная функция}
  \subsubsection{Угловая составляющая аппаратной функции}
    Для того, чтобы отдавать себе отчет в том что мы находимся на правильном пути, выбирая
модель для описания источника и щелевых устройств в схеме (раздел \ref{sec:source_section}, \ref{sec:slits_section}),
мы провели ряд экспериментов (рисунок \ref{ris:for_slits_scan}).

\begin{figure}[H]
  \centering
  \subfloat[]{\includegraphics[width=0.52\textwidth]{images/for_slits_scan.png}}
  \hfill
  \subfloat[В случае точеного источника пределы интегрирования определяются из геометрического
   перекрывания щелевых устройств, в случае $\delta \neq 0$ - а и b увеличиваются и определяются из (\ref{sec:calc_slits_ability}).
  ]{\includegraphics[width=0.45\textwidth]{images/for_slits_scan_int.png}}
  \caption{Схема эксперимента для апробации подхода к построению аппаратной функции дифрактометра}
  \label{ris:for_slits_scan}

\end{figure}
В виду отсутствия линейного детектора для прямого наблюдения углового распределения интенсивности рентгеновского
пучка после его прохождения через систему щелевых устройств (рисунок \ref{ris:calc_slits_ability_res}),
мы были вынуждены изменять угловое положение второго щелевого устройства (S2) и измерять суммарную интенсивность
за ним.

\begin{figure}[H]
  \centering
  \subfloat[$S_1 = 20$ мкм; $S_2 = 40$ мкм;]{\includegraphics[width=0.3\textwidth]{images/zero_exp_20_40.png}}
  \hfill
  \subfloat[$S_1 = 40$ мкм; $S_2 = 40$ мкм;]{\includegraphics[width=0.3\textwidth]{images/zero_exp_40_40.png}}
  \hfill
  \subfloat[$S_1 = 50$ мкм; $S_2 = 100$ мкм;]{\includegraphics[width=0.3\textwidth]{images/zero_exp_50_100.png}}
  \hfill
  % \subfloat[$S_1 = 60$ мкм; $S_2 = 40$ мкм;]{\includegraphics[width=0.3\textwidth]{images/zero_exp_60_40.png}}
  % \hfill
  \subfloat[$S_1 = 100$ мкм; $S_2 = 200$ мкм;]{\includegraphics[width=0.3\textwidth]{images/zero_exp_100_200.png}}
  \hfill
  \subfloat[$S_1 = 100$ мкм; $S_2 = 300$ мкм;]{\includegraphics[width=0.3\textwidth]{images/zero_exp_100_300.png}}
  \hfill
  \subfloat[$S_1 = 200$ мкм; $S_2 = 20$ мкм;]{\includegraphics[width=0.3\textwidth]{images/zero_exp_200_20.png}}
  \hfill
  \subfloat[$S_1 = 200$ мкм; $S_2 = 200$ мкм;]{\includegraphics[width=0.3\textwidth]{images/zero_exp_200_200.png}}
  \hfill
  \subfloat[$S_1 = 200$ мкм; $S_2 = 300$ мкм;]{\includegraphics[width=0.3\textwidth]{images/zero_exp_200_300.png}}
  \hfill
  \subfloat[$S_1 = 300$ мкм; $S_2 = 300$ мкм;]{\includegraphics[width=0.3\textwidth]{images/zero_exp_300_300.png}}
  \caption{Нолькристальный эксперимент для разных размеров щелевых устройств; $L_1= 570 $мм,
  $L_2 = 1005$ мм; $\delta = 0.1$ мм; (красная линия) - расчет, (синие точки) - эксперимент.  }
  \label{ris:zero_exp}
\end{figure}

  \subsubsection{Спектральная составляющая аппаратной функции}
      Так как в случае лабораторного источника рентгеновское излучение имеет
некое угловое  (см. \ref{sec:source_section}) и спектральное распределение,
для рентгенодифракционных исследований требуется монохроматор, принцип действия которого
был описан в разделе \ref{sec:single_crystal_section}. Исходный пучок, отражаясь от
кристалла (схема на рис. \ref{ris:single_crystal_schem_lamtet}), разделятся в пространстве
в соответствие с условием Вульфа-Брэгга.
% В рамках последовательно движения от более простого к более сложному мы не оставили без внимание
% получения угловой зависимости интенсивности (рис. \ref{ris:single_crystal_schem_exp}), чтобы соотнести
% выражение для описания спектра трубки (\ref{eq:source_spectral}) с экспериментальным.

\begin{figure}[H]
  \centering
  \includegraphics[width=0.5\textwidth]{images/single_crystal_schem_exp.png}
  \caption{Принцип действия монокристального монохроматора: лучи с разной энергией отражаются под разными углами
  в соответсвии с законон Вульфа-Брэгга}
  \label{ris:single_crystal_schem_exp}
\end{figure}
%
Угловое распределение характеристического излучения после его отражения от монохроматора (по сути,
вид спектра источника излучения) может быть получено в эксперименте с помощью двух видов сканирования:
поворота образца при неподвижном детекторе, либо сканированием разложенного в пространстве
с помощью монохроматора спектра посредством поворота детектора с установленной перед ним узкой щелью.

На рис. \ref{ris:zero_exp} представлен вид спектра рентгеновской трубки с молибденовым анодом,
полученный путем первого вида сканирования. В качестве монохроматора был использован Si(220)
% Интенсивности отражения рентгеновского излучения приведенная на рис. \ref{ris:zero_exp} может быть
% получена в зависимость от угла поворота кристалла или движение детектор с щелевым устройством,
% задающим его апертуру. В качестве кристалла был взят монокристалл кремния Si(220).

\begin{figure}[H]
  \centering
  \subfloat[]{\includegraphics[width=0.45\textwidth]{images/single_cr_exp_s_005mm.png}}
  \hfill
  \subfloat[]{\includegraphics[width=0.45\textwidth]{images/single_cr_exp_s_02mm.png}}
  \caption{Угловая зависимость интенсивности рентгеновского излучения
  после отражения характерестического излучения $MoK_{\alpha}$ от кристалла монохроматора:
  расчет -  (красная линия), эксперимент - (синие точки) для $S = 50$ мкм; полуширина $k_{\alpha 1}$ линии ($\vartheta=0$)
   составляет около 30 угл.сек. (a), $S = 200$ мкм; полуширина $k_{\alpha 1}$ линии ($\vartheta=0$)
   составляет около 50 угл.сек. (b)}
  \label{ris:zero_exp}
\end{figure}

Результат сравнения экспериментальной картины дифракции и моделирования
подтверждает правильность выбора функции спектра рентгеновской трубки (\ref{eq:source_spectral}),
в виде суммы двух функций Лоренца, взятых с весовыми коэффициентами. Также из
сравнения экспериментальных и расчетных данных можно сделать вывод о том, что тормозным
излучением рентгеновской трубки при расчетах можно пренебречь.

  \subsection{Двухкристальные КДО}

    \subsubsection{Бездисперсионная схема}
    Метод анализа КДО по прежнему являются одним из основных инструментов диагностики не только совершенства
кристаллических материалов \cite{sov_1} - \cite{sov_5}, в частности, объемных и поверхностных дефектов в
монокристаллах, тонких пленках, а также многослойных кристаллических структурах, но и для анализа физических
процессов происходящих в кристаллах, таких как воздействие внешнего электрического поля [] (пьезоэлектрический эффект),
 температуры [] или влияние магнитного поля [].

\subsubsection{Вклад соседней характеристической линии в КДО}

\label{sec:non_disspers_KDO_section}
На рисунке \ref{ris:non_disspers_kdo} приведены результаты численного расчета в соответсвии
с выражением (\ref{eq:doudle_spectra_angle_map_on_detector}). В качестве кристалла монохроматора
и образца был выбран монокристалл кремния с отражающей плоскостью (220), эксперимент проводился в
соответсвии со схемой (рисунок \ref{ris:double_crystal_schem_lamtet_a}), материалом источника рентгеновского излучения является молибден.

\begin{figure}[H]
  \centering
  \subfloat[$S_1 = 20 $ мкм; $ S_2 = 40$ мкм;]{\includegraphics[width=0.45\textwidth]{images/non_disspers_20_40.png}\label{fig:f1}}
  \hfill
  \subfloat[$S_1 = 20 $ мкм; $ S_2 = 40$ мкм; ]{\includegraphics[width=0.45\textwidth]{images/non_disspers_20_40_log.png}\label{fig:non_disspers_kdo_1}}
  \hfill
  \subfloat[$S_1 = 300 $ мкм; $ S_2 = 200$ мкм;]{\includegraphics[width=0.45\textwidth]{images/non_disspers_300_200.png}\label{fig:f2}}
  \hfill
  \subfloat[$S_1 = 300 $ мкм; $ S_2 = 200$ мкм;]{\includegraphics[width=0.45\textwidth]{images/non_disspers_300_200_log.png}\label{fig:f2}}
  \caption{Двухкристальная КДО для схемы с кристаллом монохроматором Si(220) и образцом  Si(220); $L_1= 570 $мм,
  $L_2 = 1005$ мм; $\delta = 0.1$ мм; (красная линия) - расчет, (синие точки) - эксперимент.}
  \label{ris:non_disspers_kdo}
\end{figure}

На рисунке \ref{fig:non_disspers_kdo_1} видно, что наряду с главным пиком, соответствующим $k_{\alpha1}$ лиинии
излучения, на которую настроен монохроматор, присутствует вклад от соседней характеристической линии
 $k_{\alpha2}$. Впервые, на это свойство двухкристальных КДО, получаемы в бездисперсионной
схеме, в случае использования рентгеновской трубки было указано авторами работы \cite{chuev2008}

    \subsubsection{Дисперсионная схема}
    \subsubsection{Дисперсионная схема дифракции}
\begin{figure}[H]
  \centering
  \subfloat[Образец Si(440), $S_1 = S_2 = 100$ мкм.]{\includegraphics[width=0.45\textwidth]{images/disspers_220_440_100mcm.png}\label{fig:f1}}
  \hfill
  \subfloat[Образец Si(660), $S_1 = S_2 = 100$ мкм.]{\includegraphics[width=0.45\textwidth]{images/disspers_220_660_100mcm.png}\label{fig:f2}}
  \hfill
  \subfloat[Образец Si(440), $S_1 = S_2 = 300$ мкм.]{\includegraphics[width=0.45\textwidth]{images/disspers_220_440_300mcm.png}\label{fig:f2}}
  \hfill
  \subfloat[Образец Si(660), $S_1 = S_2 = 300$ мкм.]{\includegraphics[width=0.45\textwidth]{images/disspers_220_660_300mcm.png}\label{fig:f2}}
  \caption{Двухкристальная КДО для схемы с кристаллом монохроматором Si(220) для дисперсионного случая}
  \label{ris:disspersion_curves_expantheory}
\end{figure}

    \subsubsection{Учет асимметрии отражения}
      \subsubsection{Асимметричный случай отражения}
  На рисунке \ref{ris:assymetric_exp_50}
  приведены результаты двухкристального эксперимента, где в качестве
  кристалла образца и монохроматора использовался кристалл кремния Si(440). Образец был взят таким
  образом, что плоскость отражения располагалась под углом $\phi = 20.52^o$ к поверхности.

  \begin{figure}[H]
    \centering
    \subfloat[$b = 33.52$, $\varphi$ > 0]{\includegraphics[width=0.45\textwidth]{images/assym-blue-50.png}\label{ris:assymetric_exp_a}}
    \hfill
    \subfloat[$b = 0.03$, $\varphi$ < 0]{\includegraphics[width=0.45\textwidth]{images/assym-red-50.png}\label{ris:assymetric_exp_b}}
    \caption{Двухкристальная КДО для схемы с кристаллом монохроматором Si(440) и асимметричным образцом Si(440),
    угол разориентации поверхности $\varphi = 20^o53^{'}$. Размер щелевых устройств $S_1 = S_2 = 50$ мкм.}
    \label{ris:assymetric_exp_50}
  \end{figure}

  Как было показано в разделе \ref{sec:rocking_curve_section}, чтобы получить
  рентгеновский пучок с очень малой угловой расходимостью необходимо выбирать
  скользящий угол падения к поверхности кристалла \ref{ris:assymetric_exp_a}.

  \subsection{Влияние внешнего электрического поля на двухкристальные КДО}
    % 
--------результаты
цели и выводы

1. Были разработаны алгоритмы вычисления аппаратной функции дифрактометра позволяющие моделировать
двумерное спектрально угловое распределение рентгеновского излучения в экспериментальной схеме
для широкого спектра источников излучения и оптических элементов схемы. Данные алгоритмы позволяют расичтывать
киртины двухкристальной рентгеновской дифракции с учетом ассиметри, дисперсионности. А так же сделаны первые шаги
для расчета идеальных трехкристальных
2. Данный алгоритм был апробирован на всех этапах, что позволило подтвердить их правильности,
а так же определить и уточнить параметры экспериментальной схемы. Такие как полуширина пятна

3. Были разработаны алгоритмы моделирования дифракции в кристаллах подверженых пьезоэффекту а так же обработки
экспериментальных данных
4. были проведены эксперименты и их обработка для измерения пьезоэлектрических констант.
Расширение данных алгоритмов, написание дополнительных модулей в условиях внешних воздейсвий
и апрбация данных модулей в рамках исследования пьезоэлектрических свойств кристалла и
измерения пьезоэлектрических постоянных по данным рентгеновских дефракци



цель, разработка аппаратно програмного комплекса учитывющего все особенности экспериментальной схемы
позволяющего проводить моделирование картины дефракции для реального эксперимента
по исследованию вненего воздействия на кристалл внешнего ээл поля
а) любая реальна схема
б) позволяет учитываются воздействие внешнего поля на кристалл



таким образом в рамках данной рабботы была достигнута поставленная цель
в частности был разработан комплекс программ,
разработка данного комплекса открывает широкие перспективы для дальнейших
научных исследований
учета поляризуемсоти излучения учета деффектности кристаллов

    \subsubsection{Изменение профиля КДО при пьезоэффекте}
      В результате пьезоэлектрического эффекта происходит деформация
кристаллической решетки и данный эффект будет влиять на
КДО, а именно, будет меняться  угловое положение и профиль кривой.
Изменение профиля КДО может быть вызвано исходя из следующих
соображений:

1. Изменение профиля происходит за счет изменения структурного фактора, но
данный эффект не рассматривается в данной работе. На первом этапе ограничимся
случаем однородной деформации решетки кристалла. Такая деформация описывается
 матрицей пьезомодулей, вследствие чего относительное расположение атомов
 внутри элементарной ячейки остается постоянным, как и структурный фактор.

 \begin{figure}[H]
   \centering
   \includegraphics[width=0.3\textwidth]{images/none.png}
   \caption{Изменение профиля собственной КДО кристалла LGT под воздействие электрического поля}
   \label{ris:self_kdo_deformation}
 \end{figure}


2. Профиль кривой также может изменяться изменения дисперсионности схемы.
При пьезоэффекте меняется межплоскостное расстояние, а значит и угол Брэгга, таким
образом может возникать дисперсионное уширение КДО и изменение интегральной интесивности
отраженного пучка (см. \ref{sec:dispersion_cal_an_exp}). Для оценки уширения был
проведен расчет, который заключался в оценки полуширины результирующей двухкристальной КДО
в зависимости от изменения угла Брэгга образца (рис. \ref{ris:FWHM_diference_bragg})
%(отлажить по оси х дельта брэгга)
\begin{figure}[H]
  \centering
  \includegraphics[width=0.95\textwidth]{images/delta_bragg_dispers.png}
  \caption{Зависимость (a) полуширины двухкристальной КДО $f_d$ и (b) амплитуды в максимуме КДО  $P^d_0$
   от разности углов Брэгга $\Delta\theta_B =\theta_B^S-\theta_B^M $ кристаллов
  образца (M) и монохроматора (S). Отсчет ведется от угла Брэгга моноохроматора $\theta_B^M = 21.6785 ^o$ - Si (440),
  в качестве образца был взят кристалл LGT с плоскостью отражения (246) $\theta_B^S = 21.0328 ^o$}
  \label{ris:FWHM_diference_bragg}
\end{figure}

Из результатов можно сделать вывод, как и следовало ожидать минимальная полуширина соответсвует
случаю когда углы Брэгга обоих кристаллов в точности совпадают. С увеличением
дисперсионности изменение полуширины выходит на линейную зависимость, т.е.
имеет монотонный характер. При характерных для пьезоэффекта сдвигах КДО (до 10 угл.сек.)
изменение полуширины за счет изменения дисперсионности схемы составляет
величину меньшую разрешающей способности дифрактометра, т.е полуширина и
амплитуда остается постоянной
 (рис. \ref{ris:FWHM_diference_bragg_KDO}).

\begin{figure}[H]
  \centering
  \includegraphics[width=0.8\textwidth]{images/FWHM_diference_bragg_KDO.png}
  \caption{КДО для разныой степени дисперсионности схемы, $\theta_B^M = 21.6785 ^o$}
  \label{ris:FWHM_diference_bragg_KDO}
\end{figure}

Для того, чтобы добиться существенного изменения изменения кривой нужно
иметь угловую отстройку порядка градуса, такое изменение угла Брэгга
соответсвует изменению межплоскостного расстояния на величину $d/d_0 \simeq 0.03$
процентов, а такие деформации в кристалле не допустимы (при $d/d_0 > 10^{-4}$ происходит разрушение).
Таким образом в результате пьезоэффекта профиль кривой, для рассмотренных нами случаев,
должен оставаться постоянным.

Необходимо отметить, что деформации профиля КДО может происходить вследствие наличия
заряженных дефектов в кристалле, которые подвержены влиянию электрического поля. Но данный механизм
также не рассматривается в данной работе.

Данное заключение позволяется применять рассмотренные методы расчета для определения пьезоэлектрических констант,
а так же использовать времяразрешающий метод исследования (см. \ref{sec:slope_diff_piezo}).

    \subsubsection{Угловой сдвиг КДО при пьезоэффекте}
      Как было рассмотрено ранее, наличие электрического поля, приложенного
к пьезоэлектрическому кристаллу, вызывает изменение межплоскостного расстояния.
Таким образом, был измерен угловой сдвиг КДО рефлексов (рис. \ref{ris:d11_experiment}), и на основании выражения
 (см. \ref{eq:piezomodule_l}) был рассчитан модуль d11 для кристалла LGT.

\begin{figure}[H]
  \centering
  \includegraphics[width=0.7\textwidth]{images/peak_shift_1000v.png}
  \caption{Угловой сдвиг двухкристальной КДО (эксперимент). Кристалл-монохроматор: Si(440),
  кристалл-образец: LGT(440), толщина кристалла $l = 0.27мм$ }
  \label{ris:d11_experiment}
\end{figure}

Экспериментальное определенное изменение брэгговского угла в результате воздействия
электрического поля напряженностью 3,7 кВ/мм составляет 1.89 угл.сек. Таким образом,
наблюдается хорошее соответствие измеренного методом двухкристальной дифрактометрии,
 пьезомодуля $d11 = (6.8 \pm 0.3 ) 10^{-12}$ данным, полученными
  нерентгеновскими методами \cite{LGT_piezo_d11}. В перспективе
  планируется разработка алгоритма восстановления полной матрицы пьезомодулей
  по данным рентгеновской дифрактометрии путем решения системы уравнений, связывающих все $N$
  элементов матрицы пьезомодулей с экспериментально определенными сдвигами КДО для
  такого же количества $N$ рефлексов.

    \newpage

    \section*{ \centering ВЫВОДЫ }
    \addcontentsline{toc}{section}{\protect\numberline{}ВЫВОДЫ}%

      5.7.	Завершающей частью ВКР являются заключение и выводы, которые
содержат обобщение теоретических и практических результатов, 
изложенных в основной части,
и краткое описание основных результатов и выводов работы.
   Объем выводов и заключения не должен превышать 5 - 7 страниц.

      \newpage

      \section*{ \centering ЗАКЛЮЧЕНИЕ }
      \addcontentsline{toc}{section}{\protect\numberline{}ЗАКЛЮЧЕНИЕ}%
      Цель, поставленная в рамках настоящей работы, достигнута в полном объеме.
В частности, были разработаны универсальные алгоритмы расчета,
позволяющие моделировать картину рентгеновской дифракции для различных типов
источников излучения и широкого набора оптических элементов и учитывающие большое
количество эффектов, возникающих в реальной экспериментальной схеме. Данные 
алгоритмы реализованы в виде web-интерфейса, позволяющего широкому кругу
исследователей проводить моделирование дифракционной картины для экспериментальных схем,
собираемых из оптических элементов по модульному принципу. В то же время,
 результаты работы создают предпосылки для дальнейших исследований и расширения
 функционала алгоритмов моделирования, в частности, в направлении учета дефектов
 структуры кристалла, восстановления полной матрицы пьезомодулей по данным
 рентгеновской дифракции, моделирования вклада диффузного рассеяния в картину ТРД.


\newpage
  \begin{center}
  \begin{thebibliography}{99}

\bibitem{International_Tables}
  P. J. Brown, A. G. Fox, E. N. Maslen, M. A. O'Keefe and B. T. M. Willis.
  International Tables for Crystallography (2006). Vol. C, ch. 6.1, pp. 554-595
\bibitem{f0f1f12}
J. Coraux, V. Favre-Nicolin, M. G. Proietti et al. // Phys.Rev. B. – 2007. – 75. – 235312

\bibitem{afanasyev1989}
  А. М. Афанасьев, П. А. Александров, Р. М. Имамов. Ренгеновская диагностика
  субмикронных слоев. - Москва: Наука, 1989 г. - 152 с.

\bibitem{pinsker1982}
  З. Г.  Пинскер. Рентгеновская кристаллооптика. - Москва: Наука, 1982 г. - 292 с.

  \bibitem{iveronova1972}
    В. И.  Иверонова, Г. П. Ревкевич. Теория рассеяния ренгеновских лучей. -
    Москва: Издательство московского университета, 1972 г. - 248 с.
\bibitem{Willis1975}
  Willis, B. T. M. Thermal vibrations in crystallography /
  B. T. M. Willis, A. W. Pryor. — Cambridge University Press, 1975. —P. 279.

\bibitem{kibalin2015}
  Ю. А. Кибалин. Диффракционные исследования атомных колебаний в легкосплавных
  металлах, наноструктуррированных внутри пористых сред. [Текст]: автореф. дис. на соиск.
   учен. степ. канд. физ. - мат. наук (01.04.07) /
   Кибалин Юрий Андреевич; НИЦ "Курчатовский институт". – Москва, 2015. – 99 с.

\bibitem{fetisov2007}
  Г. В. Фетисов. Синхротронное излучение. Методы исследования структуры веществ. -
  Москва: ФИЗМАТЛИТ, 2007 г. - 672 с. ISBN 978-5-9221-0805-8.

\bibitem{landau_8_1992}
 Л.Д. Ландау, Е.М. Лифшиц. Теоретическая физика. том 8 –
 Электродинамика сплошных сред, 2-е изд., Москва: Наука, 1992. - 661 с.
 \bibitem{Bushuev_Oreshko_2002}
 В. А. Бушуев, А. П. Орешко. зеркальное отражение рентгеновских лучей в условиях скользящей дифракции.
 Учебное пособие. Москва: МГУ, физический факультет, 2002. - 57 с.
 \bibitem{Tanner_1998}
 D. Keith Bowen, Brain K. Tanner. High Resolution X-Ray Diffractometry and Topography. - United Kingdom: Taylor and Francis, 1998. - 265 p.

\end{thebibliography}
% \cite{Tanner_1998}

  \end{center}
  \addcontentsline{toc}{section}{\protect\numberline{}СПИСОК ИСПОЛЬЗОВАННЫХ ИСТОЧНИКОВ}%

\newpage
\appendix
  \begin{center}
  \section{ }%Выражение для поляризуемости}
  \label{sec:polarizability}
\end{center}

% ~\ref{sec:wave_equation}
Приведем упрощенный вывод Фурье компонент  $\chi_h$  для рентгеновской
поляризуемости в среде  $\chi(\vec{r})$. Если в какой либо точке находится электрон,
то уравнение его движения под действием электромагнитной волны,
исходя из второго закона Ньютона, запишется в виде \cite{iveronova1972}.
\begin{equation}
  \ddot{x}+ k\dot{x} + \omega_0^2 x = \frac{e}{m}E_0e^{i\omega t}
 \end{equation}
Откуда смещение этого заряда
\begin{equation}
  x = \frac{e}{m} \frac{E_0e^{i\omega t}}{\omega_0^2 - \omega^2+i\omega t}
 \end{equation}
где $\omega_0 $ - собственная частота колебания электрона (частота электронного перехода), $\omega$ - частота рентгеновского излучения.

Поляризация единицы объема в заданной точке пространства $P$ определяется из условия
$P = \frac{\sum e x}{\Delta V}$. Суммирование проводится по всем зарядам в
некотором малом объеме  $\Delta V$.

Для рентгеновских лучей обычно $\omega_0^2 << \omega^2$, поэтому
\begin{equation}
  4\pi P = - \frac{4\pi e^2}{m\omega^2}\frac{\Delta N}{\Delta V} E_0 e^{i\omega t} = -\frac{e^2 \lambda^2}{m \pi c^2} \rho E_0 e^{i\omega t}
 \end{equation}
где $\Delta N$ - число зарядов в объеме $\Delta V$;
$\rho = \frac{\Delta N}{\Delta V}$ - электронная плотность в заданной точке пространства.


\begin{equation}
  \chi_h = -\frac{e^2 \lambda^2}{m \pi c^2}  \rho = -\frac{e^2 \lambda^2}{m \pi c^2} \frac{F_h}{V}
 \end{equation}
 где, $ F_h = \sum_{n} f_n \cdot e^{-2\pi i\vec{h}\cdot \vec{r}_n}= \sum_{n} f_n \cdot e^{- 2 \pi i (hx_n+ky_n+lz_n)}$ -
 структурная амплитуда (раздел ~\ref{sec:structure_factor}),
 коэффициент $h$ в $F_h$ - означает конкретные значения индексов {hkl}; $V$ - объем элементарной ячейки кристалла.

  % 
\newpage
\begin{center}
\section{ }%Волновое уравнение
\label{sec:wave_equation}
\end{center}


  воспользоваться операторным тождеством
 \begin{equation}
   rot \quad rot \vec{E} = grad \quad div \vec{E} - \Laplace \vec{E}
  \end{equation}

  
\newpage
  \begin{center}
  \section{ }%функция кристалла
  \label{sec:sample_functions}
  \end{center}
{\scriptsize
\begin{lstlisting}[language=Python]

#-----------Монохроматор-----------
# статичный элемент
def monohromator(teta, itta, X0, Xh, tetaprmtr_deg, fi):
    tetaprmtr = math.radians(tetaprmtr_deg)
    gamma_0 = math.sin(math.radians(fi) + tetaprmtr)
    gamma_h = math.sin(math.radians(fi) - tetaprmtr)
    b = gamma_0/abs(gamma_h)  # коэффициент ассиметрии
    #брэговского отражения
    C = 1
    monohrom = teta-(itta-1)*math.tan(tetaprmtr)
    # угловая отстройка падающего излучения от угла Брегга
    alfa = -4*math.sin(tetaprmtr) * \
        (math.sin(tetaprmtr+monohrom)-math.sin(tetaprmtr))
    prover = (1/4/gamma_0)* \
        (X0*(1-b)-b*alfa+cmath.sqrt(((X0*(1+b)+b*alfa) * \
        (X0*(1+b)+b*alfa))-4*b*(C*C) * \
        ((Xh.real)*(Xh.real)-(Xh.imag)*(Xh.imag)-2j*Xh.real*Xh.imag)))

    if prover.imag < float(0):
        eps = (1/4/gamma_0)* \
            (X0*(1-b)-b*alfa-cmath.sqrt(((X0*(1+b)+b*alfa) * \
            (X0*(1+b)+b*alfa))-4*b*(C*C) * \
            ((Xh.real)*(Xh.real)-(Xh.imag)*(Xh.imag)-2j*Xh.real*Xh.imag)))
    else:
        eps = prover

    R = (2*eps*gamma_0-X0)/Xh/C
    return (abs(gamma_h)/gamma_0)*abs(R)*abs(R)

#-----------Образец-----------
# подвижный элемент
def sample(dTeta, teta, itta, X0, Xh, tetaprmtr_deg, fi):
    tetaprmtr = math.radians(tetaprmtr_deg)
    gamma_0 = math.sin(math.radians(fi) + tetaprmtr)
    gamma_h = math.sin(math.radians(fi) - tetaprmtr)
    # коэффициент ассиметрии брэговского отражения    # Ignore
    # SpaceConsistencyBear
    b = gamma_0/abs(gamma_h)
    C = 1
    sample = dTeta+teta-(itta-1)*math.tan(tetaprmtr)
    # угловая отстройка падающего излучения от угла Брегга
    alfa = -4*math.sin(tetaprmtr) * \
        (math.sin(tetaprmtr+sample)-math.sin(tetaprmtr))

    prover = (1/4/gamma_0)* \
        (X0*(1-b)-b*alfa+cmath.sqrt(((X0*(1+b)+b*alfa) * \
        (X0*(1+b)+b*alfa))-4*b*(C*C) * \
        ((Xh.real)*(Xh.real)-(Xh.imag)*(Xh.imag)-2j*Xh.real*Xh.imag)))

    if prover.imag < float(0):
        eps = (1/4/gamma_0)* \
            (X0*(1-b)-b*alfa-cmath.sqrt(((X0*(1+b)+b*alfa) * \
            (X0*(1+b)+b*alfa))-4*b*(C*C) * \
            ((Xh.real)*(Xh.real)-(Xh.imag)*(Xh.imag)-2j*Xh.real*Xh.imag)))
    else:
        eps = prover

    R = (2*eps*gamma_0-X0)/Xh/C

    return (abs(gamma_h)/gamma_0)*abs(R)*abs(R)


\end{lstlisting}
}

  \newpage
  \begin{center}
  \section{ }%Как посчитать щели
  \label{sec:calc_slits_ability}
  \end{center}


\begin{figure}[H]
  \centering
  \includegraphics[width=0.8\textwidth]{images/calc_slits_ability.png}
  \caption{Для расчета пропускной способности системы щелевых устройств}
  \label{ris:calc_slits_ability}
\end{figure}

Для того чтобы получить зависимость пропускной способности в зависимости от угла,
под которым распространяется рентгеновский луч, необходимо спроецировать границы
щелей на уровень источника.
$$ S^{(source)}_{1,2} = \pm \frac{S_{1,2}}{2} - \vartheta L_{1,2}$$
 Далее найти минимальное значение проекции верхних (up)
 $$ a^{up} = min [S_1^{up,source},S_2^{up,source},\frac{\delta}{2}]$$
и максимальное значение среди проекции нижних (down)
$$ a^{down} = max [S_1^{down,source},S_2^{down,source},-\frac{\delta}{2}]$$

Следующее условие будет определять величину площади параллелограмма, а соответсвенно
и характеризовать пропускную способность для разных направлений $\vartheta$

\begin{equation}
  g_s(\vartheta) =
 \begin{cases}
   0, \quad \text{если} \quad a^{down} \geq a^{up}
   \\
    (a^{up} - a^{down})L_2,\quad \text{если} \quad a^{down} < a^{up}
 \end{cases}
\end{equation}

  % \newpage
  \begin{center}
  \section{ }%Как посчитать щели
  \label{sec:_}
  \end{center}
Получить формулу


\begin{figure}[H]
  \centering
  \subfloat[f = 3, $\theta_B^M - \theta_B^S = 0$]{\includegraphics[width=0.3\textwidth]{images/220_440_660/220.png}}
  \hfill
  \subfloat[f = 18, $\theta_B^M - \theta_B^S = 11$]{\includegraphics[width=0.3\textwidth]{images/220_440_660/440.png}}
  \hfill
  \subfloat[f = 39, $\theta_B^M - \theta_B^S = 23$]{\includegraphics[width=0.3\textwidth]{images/220_440_660/660.png}}
  \caption{Дисперсия}
  \label{ris:}
\end{figure}

  % \newpage
  \begin{center}
  \section{ }%Как посчитать щели
  \label{sec:_}
  \end{center}
  \begin{figure}[H]
    \centering
    \subfloat[Схема трехкристального эксперимента]{\includegraphics[width=0.65\textwidth]{images/triple_crystal_schem.png}}
    \hfill
    \subfloat[Представление Эвальда]{\includegraphics[width=0.35\textwidth]{images/q_vector/ewald.png}}
    \caption{К выводу зависимости между углами поворота элементов схемы и проекциями волновых векторов}
    \label{ris:}
  \end{figure}




  \subsection*{qx}
  \begin{equation}
\label{eqn: distance}
	 q_x/|\vec{k}_0| = cos(\theta_B + \varepsilon -  \theta) - cos(\theta_B +  \theta)
\end{equation}


$$
	  q_x/|\vec{k}_0| =  cos(\theta_B) \cdot cos(\varepsilon -  \theta) - sin(\theta_B)\cdot sin(\varepsilon -  \theta) -
$$

$$
    - cos(\theta_B)\cdot \cancelto{1}{cos(\theta)} + sin(\theta_B) \cdot \cancelto{\theta}{sin(\theta)} =
$$
$$
	 = cos(\theta_B) \cdot [\cancelto{1}{cos(\theta)} \cdot \cancelto{1}{cos(\varepsilon)} + \cancelto{\theta}{sin(\theta)} \cdot \cancelto{\varepsilon}{sin(\varepsilon)}] -
$$
$$
	 - sin(\theta_B) \cdot [\cancelto{\varepsilon}{sin(\varepsilon)} \cdot
	 \cancelto{1}{cos(\theta)} - \cancelto{1}{cos(\varepsilon)}  \cdot
	 \cancelto{\theta}{sin(\theta)}
	 ]  -
$$
$$
	- cos(\theta_B) +  \theta\cdot sin(\theta_B) =
$$
$$
	= cos(\theta_B) + \cancelto{0}{ \theta \cdot \varepsilon \cdot cos(\theta_B) }
	- \varepsilon \cdot sin(\theta_B) +  \theta\cdot sin(\theta_B) - cos(\theta_B) +  \theta\cdot sin(\theta_B)
$$

\begin{equation}
	q_x/|\vec{k}_0| = (2\theta -\varepsilon) \cdot sin(\theta_B)
\end{equation}




\subsection*{qz}
  \begin{equation}
  	q_z/|\vec{k}_0| = sin(\theta_B + \varepsilon -  \theta )+
  	sin(\theta_B + \theta) - 2 sin(\theta_B)
  \end{equation}

  $$
  	q_z/|\vec{k}_0| =   sin(\theta_B) \cdot cos( \varepsilon  -  \theta )+
  	cos(\theta_B) \cdot sin( \varepsilon  -  \theta )+
  	sin(\theta_B) \cdot \cancelto{1}{cos(\theta)}  +
  $$
  $$
  	+cos(\theta_B) \cdot\cancelto{\theta}{sin(\theta)} - 2sin(\theta_B) =
  $$
  $$
  	=  sin(\theta_B) \cdot [\cancelto{1}{cos(\theta)} \cdot \cancelto{1}{cos(\varepsilon)} +\cancelto{\theta}{sin(\theta)} \cdot \cancelto{\varepsilon}{sin(\varepsilon)}
  	]+
  $$
  $$
  	+cos(\theta_B) \cdot [\cancelto{\varepsilon}{sin(\varepsilon)} \cdot
  	 \cancelto{1}{cos(\theta)} - \cancelto{1}{cos(\varepsilon)}  \cdot
  	 \cancelto{\theta}{sin(\theta)}
  	 ]+
  $$
  $$
  	+sin(\theta_B) + \theta \cdot cos(\theta_B)  - 2sin(\theta_B) =
  $$
  $$
  	= sin(\theta_B) +\cancelto{0}{  \theta \cdot \varepsilon \cdot sin(\theta_B)} +			\varepsilon  \cdot  cos(\theta_B)  - \theta  \cdot  cos(\theta_B) +
  $$
  $$
  	+sin(\theta_B) + \theta  \cdot  cos(\theta_B)  - 2sin(\theta_B)
  $$

  \begin{equation}
  	q_z/|\vec{k}_0| = \varepsilon  \cdot  cos(\theta_B)
  \end{equation}

  % \newpage
  \begin{center}
  \section{ }% матрицы пьезомодулей
  \label{sec:piezo_matrix}
  \end{center}
  \subsection*{ Кристалл LGT }
  Кристаллы семейства лантан-галлиевого силиката ($La_3Ga_5SiO_{14}$ - LGS и $La_3Ga_{5.5}Ta_{0.5}O_{14}$ - LGT)
  обладают пьезоэлектрическими свойствами со стабильной температурной зависимость даже при высоки температурах.
  Пьзоэлектрический модуль $d_{11}$ остается постоянным в диапазоне температур до $600^o$С
  (изменение не более 5 $\%$ \cite{LGS58}). В таких кристаллах отсутствует фазовый переход вплоть
  до температур плавления \cite{LGS57}, а так же не имеется пироэлектрического эффекта.
  Отсутсвует гистерезис физических свойств, в том числе и пьезоэлектрический эффект,
  обладают высоким коэффициентом электромеханической связи (более чем в два раза больше чем у кварца).
  Высокое удельное сопротивление, которое говорит нам об отсутствие дополнительных эффектов,
  которые могли бы вносить свой вклад в картину дифракции при воздействии внешнего электрического
  поля (образование двойного электрического слоя и др.)

  LGT кристалл имеет точечную группой симметрии $32$ и гексагональной сингонию.
  \begin{equation}
    \begin{pmatrix}
    d_{11} & -d_{11} & 0 & d_{14} & 0 & 0 \\
    0 & 0 & 0 & 0 & -d_{14} & 2d_{11} \\
    0 & 0 & 0 & 0 & 0 & 0
    \end{pmatrix}
    \label{eq:piezomodule_lgt_matrica}
  \end{equation}
  Параметры ячейки: $a = b = 8.228 \angstrom$, $c = 5.124 \angstrom$ \cite{marchenkov2014}.

  \subsection*{ Кристалл $TeO_2$ }
  ...
  \subsection*{ Кристалл $Li_2B_4O_7$ }
  ...


\end{document}
