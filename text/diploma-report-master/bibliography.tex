\begin{thebibliography}{99}
\bibitem{iveronova1972}
Иверонова, В. И. Теория рассеяния ренгеновских лучей / В. И. Иверонова, Г. П. Ревкевич.-
Москва: Издательство московского университета, 1972 г. - 248 с.

\bibitem{International_Tables}
Brown, P. J. Intensity of diffracted intensities /
 P. J. Brown,  A. G. Fox, E. N. Maslen, M. A. O'Keefe, B. T. M. Willis .//
International Tables for Crystallography. - 2006. - Vol. C, ch. 6.1. - P. 554-595.

\bibitem{f0f1f12}
Coraux, J. Grazing-incidence diffraction anomalous fine structure: Application to
the structural investigation of group-III nitride quantum dots /
J. Coraux, V. Favre-Nicolin, M. G. Proietti, B. Daudin, and H. Renevier.
// Phys.Rev. B. – 2007. - Vol. 75. - № 23.  - P. 235312.
%https://journals.aps.org/prb/abstract/10.1103/PhysRevB.75.235312

\bibitem{afanasyev1989}
 Афанасьев, А. М. Ренгеновская диагностика субмикронных слоев /
А. М. Афанасьев, П. А. Александров, Р. М. Имамов.  - Москва: Наука, 1989 г. - 152 с.

\bibitem{Willis1975}
Willis, B. T. M. Thermal vibrations in crystallography /
B. T. M. Willis, A. W. Pryor. Cambridge University Press, 1975. — P. 279.

\bibitem{kibalin2015}
Кибалин, Ю. А. Диффракционные исследования атомных колебаний в легкосплавных
металлах, наноструктуррированных внутри пористых сред: автореф. дис. на соиск.
учен. степ. канд. физ. - мат. наук (01.04.07) /
Кибалин Юрий Андреевич; НИЦ "Курчатовский институт". – Москва, 2015. – 99 с.

\bibitem{fetisov2007}
Фетисов, Г. В. Синхротронное излучение. Методы исследования структуры веществ /
 Г. В. Фетисов. - Москва: ФИЗМАТЛИТ, 2007 г. - 672 с.

\bibitem{landau_8_1992}
Ландау, Л.Д. Теоретическая физика: том 8: Электродинамика сплошных сред, 2-е, 1992. - 661 с; /
Л.Д. Ландау, Е.М. Лифшиц.Москва: Наука. - (ISBN 5-9221-0123-4).

\bibitem{Bushuev_Oreshko_2002}
Бушуев, В. А. Зеркальное отражение рентгеновских лучей в условиях скользящей дифракции:
Учебное пособие / В.А. Бушуев, А.П Орешко. Москва: МГУ, физический факультет.  2002. - 56 с.

\bibitem{pinsker1982}
Пинскер, З. Г. Рентгеновская кристаллооптика / З. Г.  Пинскер.  - Москва: Наука,  1982 г. - 292 с.

\bibitem{Tanner_1998}
Bowen, D. K. High Resolution X-Ray Diffractometry and Topography / D. K. Bowen, B. K. Tanner.
  - United Kingdom: Taylor and Francis, 1998. - 265 p.

\bibitem{kedi_1949}
Кэди, У. Пьезоэлектричество и его практические применение / У. Кэди.
 — Москва: Издательство Иностранной литературы, 1949. - 721 p.

\bibitem{Newnham_2005}
Newnham, R. E. Properties of materials. Anisotropy, symmetry, structure /
Robert E. Newnham.  - United Kingdom: Oxford University Press, 2004. -  620 p.

\bibitem{Shaskolska_1984}
Шаскольская, М. П. Кристаллограяфия / М. П. Шаскольская. - Москва: Высшая школа, 1984. - 386 p.

\bibitem{LGS58}
 Bohm, J. Czochralski growth and characterization of piezoelectric single crystals
 with langasite structure: $La_{3}Ga_{5}SiO_{14}$(LGS), $La_{3}Ga_{5.5}Nb_{0.5}O_{14}$ (LGN) and
 $La_{3}Ga_{5.5}Ta_{0.5}O_{14}$ (LGT) II. Piezoelectric and elastic properties /
 J. Bohm, E. Chilla, C. Flannery, et. al. / /Journal of Crystal Growth. - 2000. Vol. - 216. - P. 293-298.

\bibitem{marchenkov2014}
Марченков, Н. В. Рентгенодифракционные исследования пьезоэлектрических кристаллов
при воздействии внешних электрических полей: автореф. дис. на соиск.
учен. степ. канд. физ. - мат. наук (01.04.18) /
Марченков Никита Владимирович; Институт кристаллографии им. А. В. Шубникова РАН  – Москва, - 2014. – 122 с.

\bibitem{trd_Bushuev_1997}
Бушуев, В. А. Особенности формирования спектров трехкристальной рентгеновской дифрактометрии:
Учебное пособие для студентов старших курсов  / В. А. Бушуев, А. П. Петраков. - г. Сыктывкар, - 1997. - 23 с.

\bibitem{sov_1}
Hart, M. Crystal diffraction optics for x-rays and neutrons /
M. Hart. // Lecture Notes in Physics. Imaging Processes and Coherence in Physics. - 1980. - V. 112. - P. 325-335.
% https://link.springer.com/chapter/10.1007/3-540-09727-9_92

\bibitem{sov_2}
Matsushita, T. Sagittally focusing double-crystal monochromator with constant
exit beam height at the photon factory / T. Matsushita, T. Ishikawa, H. Oyanagi.
// Nucl. Instrum. Methods. - 1986. - V. 246. - P. 377.
% http://www.sciencedirect.com/science/article/pii/0168900286901129

\bibitem{sov_3}
Erko A. Modern Developments in X-Ray and Neutron Optics / A. Erko, M. Idir, Th. Krist, G. Michette.
-   Berlin: Springer, 2008. - 541 p.

\bibitem{chuev2008}
Чуев, М.А. Нетривиальная роль аппаратной функции в формировании
 кривых рентгеновской дифракции в двухкристальной схеме /
Чуев, Э.М. Пашаев, В.В. Квардаков, И.А. Субботин. // Кристаллография. -
 2008. - Т.53. - № 5. - С. 780.
% https://elibrary.ru/item.asp?id=11479843

\bibitem{LGS57}
Wang, Z. Piezoelectricity of $A_3BC_3D_2O_{14}$ structure crystals. /
Z. Wang, D. Yuan, L. Pan, X. Cheng. // Appl. Phys. Lett. - 2003. - № 77. -  P. 683–685.
% Z. Wang, D. Yuan, L. Pan, X. Cheng // Appl. Phys. Lett. 2003. No 77. P. 683–685.
% неправильная статья

\bibitem{Dumond1937}
DuMond, J. W. M. Theory of the Use of More Than Two Successive
X-Ray Crystal Reflections to Obtain Increased Resolving Power /
J. W. M. DuMond. // Phys.Rev. – 1937. – V. 52. – P. 872-883.
% https://journals.aps.org/pr/abstract/10.1103/PhysRev.52.872

\bibitem{blohin1957}
Блохин, М. А. Физика рентгеновских лучей / М. А. Блохин: 2 изд. - М.: ГИТТЛ, 1957. - 518 с .

\bibitem{sov_5}
Сутырин, А.Г. Генетический алгоритм решения обратной
задачи в методе высокоразрешающей рентгеновской рефлектометрии /
Сутырин, Д.Ю. Прохоров. // Кристаллография. 2006. -Т. 51. - № 5. - С. 570.
% непонятно
% https://elibrary.ru/item.asp?id=9218763

\bibitem{piezo102}
Paturle, A. Measurement of the piezoelectric tensor of an organic crystal by the x-ray method:
The nonlinear optical crystal 2-methyl 4-nitroaniline / A. Paturle, H. Graafsma,
H.-S. Sheu, P. Coppens, P. Becker. // Phys. Rev. B. – 1991. – V. 43, № 18. – P. 14683-14691.

\bibitem{temp}
Kohra, K. Study on Temperature Effect on X-Ray
Diffraction Curves from Single Crystals by a Triple-Crystal Spectrometer /
K. Kohra, S. Kikuta, S. Annaka, S. Nakano. // J. Phys. Soc. Jpn. -1966. - № 21. - P. 1565-1572.

\bibitem{piezo51}
Gorfman, S.V.X-ray diffraction by a crystal in a permanent external electric field: general considerations
 Gorfman, V.G. Tsirelson, U. Pietsch. // Acta Cryst. - 2005. - V. A61. -  P. 387-396.

\bibitem{piezo52}
Gorfman, S.V.
X-ray diffraction by a crystal in a permanent external electric
field: electric-field-induced structural response in $\alpha$-$GaPO_4$ /
S.V. Gorfman, V.G. Tsirelson, A. Pucher, W. Morgenroth, U. Pietsch. // Acta Cryst. - 2006. - V. A62. - P. 1-10.
% http://journals.iucr.org/a/issues/2006/01/00/

\bibitem{piezo53}
Gorfman, S. X-ray diffraction study of the piezoelectric properties of BiB3O6 single crystals /
S. Gorfman, O. Schmidt, U. Pietsch, P. Becker, L. Bohaty. // Z.Kristallogr. - 2007. - V. 222. - P. 396-401.
% https://www.degruyter.com/view/j/zkri.2007.222.issue-8/zkri.2007.222.8.396/zkri.2007.222.8.396.xml?format=INT

\bibitem{piezo54}
Schmidt,O. Electric-field-induced internal deformation in piezoelectric $BiB_3O_6$ crystals /
 O. Schmidt, S. Gorfman, U. Pietsch. // Cryst. Res. Technol. - 2008. - V. 43. - № 11. - P. 1126-1132.
% http://onlinelibrary.wiley.com/doi/10.1002/crat.200800335/full

\bibitem{piezo50}
Annaka, S. Piezoelectric constants of $\alpha$-quartz determined from dynamical X-ray diffraction curves /
S. Annaka. // J. Appl. Cryst. - 1977. - V. - 10. - P. 354-355.
% http://journals.iucr.org/j/issues/1977/04/00/

\bibitem{piezo101}
Guillot, R. Diffraction study of the piezoelectric properties of low quartz
/ R. Guillot, P. Fertey, N. K. Hansen, P. Alle, E. Elkaim, C. Lecomte. //
Eur. Phys. J. B. – 2004. – V. 42, - Issue 3. – P. 373-380.

\bibitem{LGT_piezo_d11}
N.S. Kozlova, E.V. Zabelina, O.A. Buzanov, V.V.Geraskin. //
Abstract Booklet of 9 th European Conference on Applications of
polar Dielectrics, (ECAPD IX, Roma) Roma, Italy. 2008. P. 247.
\bibitem{lider2009}
В. В. Лидер, Фазочувствительнык рентгеновские методы характеризации
конденсированных сред. Учебное пособие. Москва: Институт Кристаллографии им. А. В. Шубникова, 2009. - 90 с.
\end{thebibliography}
% \cite{lider2009}
