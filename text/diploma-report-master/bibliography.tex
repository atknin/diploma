\begin{thebibliography}{99}

\bibitem{International_Tables}
  P. J. Brown, A. G. Fox, E. N. Maslen, M. A. O'Keefe and B. T. M. Willis.
  International Tables for Crystallography (2006). Vol. C, ch. 6.1, pp. 554-595
\bibitem{f0f1f12}
J. Coraux, V. Favre-Nicolin, M. G. Proietti et al. // Phys.Rev. B. – 2007. – 75. – 235312

\bibitem{afanasyev1989}
  А. М. Афанасьев, П. А. Александров, Р. М. Имамов. Ренгеновская диагностика
  субмикронных слоев. - Москва: Наука, 1989 г. - 152 с.

\bibitem{pinsker1982}
  З. Г.  Пинскер. Рентгеновская кристаллооптика. - Москва: Наука, 1982 г. - 292 с.

  \bibitem{iveronova1972}
    В. И.  Иверонова, Г. П. Ревкевич. Теория рассеяния ренгеновских лучей. -
    Москва: Издательство московского университета, 1972 г. - 248 с.
\bibitem{Willis1975}
  Willis, B. T. M. Thermal vibrations in crystallography /
  B. T. M. Willis, A. W. Pryor. — Cambridge University Press, 1975. —P. 279.

\bibitem{kibalin2015}
  Ю. А. Кибалин. Диффракционные исследования атомных колебаний в легкосплавных
  металлах, наноструктуррированных внутри пористых сред. [Текст]: автореф. дис. на соиск.
   учен. степ. канд. физ. - мат. наук (01.04.07) /
   Кибалин Юрий Андреевич; НИЦ "Курчатовский институт". – Москва, 2015. – 99 с.

\bibitem{fetisov2007}
  Г. В. Фетисов. Синхротронное излучение. Методы исследования структуры веществ. -
  Москва: ФИЗМАТЛИТ, 2007 г. - 672 с. ISBN 978-5-9221-0805-8.

\bibitem{landau_8_1992}
 Л.Д. Ландау, Е.М. Лифшиц. Теоретическая физика. том 8 –
 Электродинамика сплошных сред, 2-е изд., Москва: Наука, 1992. - 661 с.
 \bibitem{Bushuev_Oreshko_2002}
 В. А. Бушуев, А. П. Орешко. зеркальное отражение рентгеновских лучей в условиях скользящей дифракции.
 Учебное пособие. Москва: МГУ, физический факультет, 2002. - 57 с.
 \bibitem{Tanner_1998}
 D. Keith Bowen, Brain K. Tanner. High Resolution X-Ray Diffractometry and Topography. - United Kingdom: Taylor and Francis, 1998. - 265 p.
 \bibitem{kedi_1949}
 У. Кэди. Пьезоэлектричество и его практические применения. — Москва: Издательство Иностранной литературы, 1949. - 721 p.
 \bibitem{Newnham_2005}
  Robert E. Newnham. Properties of materials. Anisotropy, symmetry, structure. - United Kingdom: Oxford University Press, 2004. -  620 p.
  \bibitem{Shaskolska_1984}
  М. П. Шаскольская. Кристаллограяфия. - Москва: Высшая школа, 1984. - 386 p.

  \bibitem{marchenkov2014}
    Н. В. Марченков. Рентгенодифракционные исследования пьезоэлектрических кристаллов
    при воздействии внешних электрических полей. [Текст]: автореф. дис. на соиск.
     учен. степ. канд. физ. - мат. наук (01.04.18) /
     Марченков Никита Владимирович; Институт кристаллографии им А. В. Шубникова РАН  – Москва, 2014. – 122 с.
\bibitem{trd_Bushuev_1997}
     В. А. Бушуев. Особенности формирования спектров трехкристальной рентгеновской дифрактометрии:
    Учебное пособие для студентов старших курсов  / В. А. Бушуев, А. П. Петраков. - г. Сыктывкар, 1997. - 23 с.
\bibitem{sov_1}
      Hart M. // Lecture Notes in Physics. Imaging Processes and Coherence in Physics. 1980. V. 112. P. 325-335.
\bibitem{sov_2}
    Matsushita T., Ishikawa T.,Oyanagi H. // Nucl. Instrum. Methods. 1986. V. A246. P. 377.
\bibitem{sov_3}
      A. Erko, M. Idir, Th. Krist, G. Michette // Modern Developments in X-Ray and Neutron Optics, Eds.: Springer. 2008. 541 p.
\bibitem{chuev2008}
    М.А. Чуев, Э.М. Пашаев, В.В. Квардаков, И.А. Субботин // Кристалло- графия. 2008. Т.53. No 5. С. 780.
\bibitem{sov_5}
А.Г. Сутырин, Д.Ю. Прохоров //Кристаллография. 2006. Т. 51. No 5. С.570. 91. F.N. Chukhovskii //Acta Cryst. 2009. V. A65. P. 39.
\bibitem{LGS57}
  Z. Wang, D. Yuan, L. Pan, X. Cheng // Appl. Phys. Lett. 2003. No 77. P. 683–685.
\bibitem{LGS58}
    J. Bohm, E. Chilla, C. Flannery, H. J. Frohlich, T. Hauke, R. B. Heimann, M. Hengst, U. Straube // Journal of Crystal Growth. 2000. No 216. P. 293-298.
\bibitem{piezo51}
  S.V. Gorfman, V.G. Tsirelson, U. Pietsch. // Acta Cryst. V. A61. P. 387-396.
\bibitem{piezo52}
  S.V. Gorfman, V.G. Tsirelson, A. Pucher, W. Morgenroth, U. Pietsch. // Acta Cryst. V. A62. P. 1-10.
\bibitem{piezo53}
  S. Gorfman, O. Schmidt, U. Pietsch, P. Becker, L. Bohaty // Z.Kristallogr. 2007. V. 222. P. 396-401.
\bibitem{piezo54}
  O. Schmidt, S. Gorfman, U. Pietsch // Cryst. Res. Technol. 2008. V. 43. N. 11. P. 1126-1132.
\bibitem{piezo50}
  S. Annaka // J. Appl. Cryst. 1977. V. 10. P. 354-355.
\end{thebibliography}
% \cite{piezo50}
