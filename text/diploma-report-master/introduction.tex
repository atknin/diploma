
Актуальность:
  В настоящее время одним из основных источников информации об атомной структуре
  вещества являются методы основанные на взаимодействии ренгеновского излучения с веществом.
  Такой инструмент является с одной стороны неразрушающим и высокочувствительным,
   а с другой позволяет исследовать процессы, напрямую связанные со структурными
   свойствами кристаллических материалов. Необходимо отметить, что процесс
   получения информации о взаимодействии рентгеновского излучения с веществом на
   источниках синхротронного излучения, а также на лабораторных источниках, на
   сегодняшний день полностью автоматизирован практически для любой геометрии эксперимента.
    Интерес ученых возрастает к исследованию кристаллической структуры в условиях
     внешнего воздействия, например, температуры, давления, электрического или
     магнитного полей.  Такой подход является следующим этапом развития современных
     теоретических и экспериментальных методов, которые позволят по-новому взглянуть на
     явления, которые до сих пор не до конца понятны на уровне атомных взаимодействий.
      Это, в свою очередь, будет способствовать созданию принципиально новых материалов,
       понять природу которых не представлялось возможным до настоящего времени. Ключевым
       этапом всех рентгенодифракционных экспериментов является обработка экспериментальных
       результатов и решение обратной задачи-определения структурных параметров
       исследуемых образцов по данным рентгеновской дифракции. В этой связи целью
       настоящей работы является:
  разработка расчетно - методической базы для моделирования картины рентгеновской
  дифракции, соответствующей измеренным в реальных экспериментальных схемах
  с учетом их особенностей для кристаллов, в том числе подверженных влиянию
  внешнего электрического поля. Разрабатываемый комплекс вычислительных алгоритмов
  должен быть предназначен
  для моделирования картины дифракции, соответствующей реальному эксперименту,
  и обработки экспериментальных данных.

Для достижения данной цели были поставлены следующие задачи:
\begin{enumerate}
\item Разработать алгоритмы вычисления аппаратной функции дифрактометра, позволяющие моделировать
двумерное спектрально-угловое распределение рентгеновского излучения в экспериментальной схеме
для широкого спектра источников излучения и различных оптических элементов.
Данные алгоритмы должны позволять рассчитывать картину двухкристальной рентгеновской дифракции с учетом
асимметрии рефлексов и дисперсионности оптической схемы.

\item Апробировать разработанный алгоритм на каждом этапе его создания,
проводя сравнение с реальным экспериментом.

\item  Разработать алгоритмы моделирования дифракции в кристаллах, подверженных
влиянию внешнего электрического поля, для исследования пьезоэлектрчиеского эффекта.

\item Провести эксперименты по измерению пьезоэлектрических констант
методом рентгеновской дифракции с целью апробации разработанных алгоритмов
обработки экспериментальных данных
\end{enumerate}

%
% В свете вышесказанного, настоящая работа базируется на исследовании технически
% важных кристаллов
