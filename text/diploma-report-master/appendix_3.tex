\newpage
  \section{ }%функция кристалла
  \label{sec:sample_functions}

Приведен пример функции отражения от кристалла-образца на
языке программирования Python. Функция на вход принимает
параметр углового поворота кристалла, угол характеризующий
составляющую в расходящемся пучке, комплексные величины поляризуемостей
 рентгеновского пучка, а также коэффициент поляризации излучения и угол
 характеризующий асимметричность отражения.

{\scriptsize
\begin{lstlisting}[language=Python]
#--------------Функция кристалла-образца--------------
def sample(dTeta, teta, itta, X0, Xh, tetaprmtr, fi, C):
    # teta - угловое распределение источника (радианы)
    # dTeta - величина отстройки от точного угла Брэгга (радианы)
    # (угол поворота образца)
    # itta - длина волны излучения пересчитанная (радианы)
    # в угловую шкалу в соответсвии с условием Вульфа-Брэгга
    # X0 - комплексная поляризуемость падающего излучения
    # Xh - комплексная поляризуемость для дифрагированного излучения
    # tetaprmtr - угол Брэгга (радианы)
    # C - коэффициент поляризации излучения
    # fi - угол наклона отражающих плоскостей относительно поверхности (градусы)
    gamma_0 = math.sin(math.radians(fi) + tetaprmtr)
    gamma_h = math.sin(math.radians(fi) - tetaprmtr)
    # b - коэффициент асимметрии брэговского отражения
    b = gamma_0/abs(gamma_h)
    sample = dTeta+teta-(itta-1) * math.tan(tetaprmtr)
    # угловая отстройка падающего излучения от угла Брэгга с учетом
    # всех перечисленных факторов
    alfa = -4*math.sin(tetaprmtr) *
              (math.sin(tetaprmtr+sample) - math.sin(tetaprmtr))
    # alfa - в соответсвии с формулой (1.4.5)
    prover = (1/4/gamma_0)*
              (X0*(1-b)-b*alfa+cmath.sqrt(((X0*(1+b)+b*alfa) *
              (X0*(1+b)+b*alfa))-4*b*(C*C) *
              ((Xh.real)*(Xh.real)-(Xh.imag)*(Xh.imag)-2j*Xh.real*Xh.imag)))
    if prover.imag < float(0): #выполнения условия затухания волны,
        eps = (1/4/gamma_0) * # выбор отрицательной мнимой части аккомодации
              (X0 * (1-b)-b*alfa-cmath.sqrt(((X0*(1+b)+b*alfa) *
              (X0 * (1+b)+b * alfa))-4 * b * (C*C) *
              ((Xh.real)*(Xh.real) - (Xh.imag)*(Xh.imag) - 2j*Xh.real*Xh.imag)))
    else:
        eps = prover
    R = (2 * eps*gamma_0 - X0)/Xh/C # коэффициент отражения собственной КДО
    return (abs(gamma_h)/gamma_0) * abs(R) * abs(R)
\end{lstlisting}
}
В результате выполнения, функция возвращает коэффициент отражения собственной КДО в
соответствии с формулой (\ref{eq:KDO_self}). Тестовая версия web - интерфейса
доступна по адресу: http://x-rays.world/ или http://62.109.0.242/.
