
\label{sec:non_disspers_KDO_section}
На рис. \ref{ris:non_disspers_kdo} приведены результаты численного расчета в соответсвии
с выражением (\ref{eq:doudle_spectra_angle_map_on_detector}). В качестве кристалла монохроматора
и образца был выбран монокристалл кремния с отражающей плоскостью (220), эксперимент проводился в
соответсвии со схемой (рис. \ref{ris:double_crystal_schem_lamtet_a}).

\begin{figure}[H]
  \centering
  \subfloat[]{\includegraphics[width=0.45\textwidth]{images/non_disspers_20_40.png}\label{fig:f1}}
  \hfill
  \subfloat[]{\includegraphics[width=0.45\textwidth]{images/non_disspers_20_40_log.png}\label{fig:non_disspers_kdo_1}}
  \hfill
  \subfloat[]{\includegraphics[width=0.45\textwidth]{images/non_disspers_300_200.png}\label{fig:f2}}
  \hfill
  \subfloat[]{\includegraphics[width=0.45\textwidth]{images/non_disspers_300_200_log.png}\label{fig:f2}}
  \caption{Двухкристальная КДО для схемы с установленным кристаллом монохроматором Si(220) и образцом  Si(220). Расстояние до щелевых устройств
  составляет $L_1= 570 $мм, $L_2 = 1005$ мм. Линейный размер источника $\delta = 0.1$ мм. (красная линия) - расчет, (синие точки) - эксперимент
  для размеров щелевых устройств (a) $S_1 = 20 $ мкм; $ S_2 = 40$ мкм, (b) $S_1 = 20 $ мкм; $ S_2 = 40$ мкм,
  (c) $S_1 = 300 $ мкм; $ S_2 = 200$ мкм, (d) $S_1 = 300 $ мкм; $ S_2 = 200$ мкм}
  \label{ris:non_disspers_kdo}
\end{figure}

На рис. \ref{fig:non_disspers_kdo_1} видно, что наряду с главным пиком, соответствующим $k_{\alpha1}$ лиинии
излучения, на которую настроен монохроматор, присутствует вклад от соседней характеристической линии
 $k_{\alpha2}$. Впервые, на это свойство двухкристальных КДО, получаемы в бездисперсионной
схеме, в случае использования рентгеновской трубки было указано авторами работы \cite{chuev2008}.

\begin{figure}[H]
  \centering
  \includegraphics[width=0.8\textwidth]{images/vklad_kalpha2.png}
  \caption{Образование дополнительного пика в спектрально-угловом представлении}
  \label{ris:vklad_kalpha2}
\end{figure}

На рис. \ref{ris:vklad_kalpha2} изображен механизм формирования дополнительного пика,
соответствующего $k\alpha_2$ составляющей спектра. В точке образования пика ($\theta = 230$ угл.сек.), коэффициент
отражения  (см. \ref{eq:doudle_spectra_angle_map_on_detector})
для кристалла образца при длине волны $\lambda_2$ максимален (в случае кристалла Si равен 1). Но отражения
от монохроматора в этой точке является слабым, т.о. интенсивность дополнительного пика на 5 порядков меньше
интенсивности основного. Необходимо отметить, что пик существует вне зависимости от размера щелевых коллиматоров, но
при достаточно большом размере (200 мкм.) становится менее выраженным.
