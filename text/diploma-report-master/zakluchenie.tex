  В рамках настоящей работы получены следующие результаты:
\begin{enumerate}
\item Разработаны алгоритмы вычисления аппаратной функции дифрактометра, позволяющие моделировать
двумерное спектрально-угловое распределение рентгеновского излучения в экспериментальной схеме
для широкого спектра источников излучения и наличия разных оптических элементов.
Данные алгоритмы позволяют рассчитывать картину двухкристальной рентгеновской дифракции с учетом
асимметрии отражений от кристаллов и дисперсионности экспериментальной схемы.
 Также сделаны первые шаги в направлении расчета трехкристальных кривых дифракционного
отражения, на данном этапе только для случая идеальных кристаллов.

\item Разработанные алгоритмы были апробированы на всех уровнях, что позволило подтвердить их адекватность,
а также определить и уточнить параметры экспериментальной схемы, такие как линейный размер пятна
рентгеновской трубки.

\item  Разработаны алгоритмы моделирования картины дифракции рентгеновского излучения в
пьезоэлектрических кристаллах, подверженных
влиянию внешнего электрического поля для исследования пьезоэлектрчиеского эффекта и обработки
экспериментальных данных.

\item Проведены эксперименты по исследованию влияния внешнего электрического поля на структуру
пьезоэлектрических кристаллов и обработке полученных результатов с помощью разработанных алгоритмов
с целью определения пьезокоэффициентов.
\end{enumerate}
