\begin{enumerate}
\item Разработаны алгоритмы вычисления аппаратной функции дифрактометра, позволяющие моделировать
двумерное спектрально-угловое распределение рентгеновского излучения в экспериментальной схеме
для широкого спектра источников излучения и наличия разных оптических элементов.
Данные алгоритмы позволяют рассчитывать картину двухкристальной рентгеновской дифракции с учетом
асимметрии и дисперсионности схемы. А так же сделаны первые шаги для расчета трехкристальных кривых дифракционного
отражения, на данном этапе только для случая идеальных кристаллов в схеме.

\item Данный алгоритм был апробирован на всех этапах, что позволило подтвердить их правильность,
а так же определить и уточнить параметры экспериментальной схемы, такие как линейный размер рентгеновско пятна.

\item  Разработаны алгоритмы моделирования дифракции в кристаллах подверженных
влиянию внешнего электрического поля для исследования пьезоэлектрчиеского эффекта и обработки
экспериментальных данных.

\item Проведены эксперименты и сделана их обработка для измерения пьезоэлектрических констант.
\end{enumerate}
