
--------результаты
цели и выводы

1. Были разработаны алгоритмы вычисления аппаратной функции дифрактометра позволяющие моделировать
двумерное спектрально угловое распределение рентгеновского излучения в экспериментальной схеме
для широкого спектра источников излучения и оптических элементов схемы. Данные алгоритмы позволяют расичтывать
киртины двухкристальной рентгеновской дифракции с учетом ассиметри, дисперсионности. А так же сделаны первые шаги
для расчета идеальных трехкристальных
2. Данный алгоритм был апробирован на всех этапах, что позволило подтвердить их правильности,
а так же определить и уточнить параметры экспериментальной схемы. Такие как полуширина пятна

3. Были разработаны алгоритмы моделирования дифракции в кристаллах подверженых пьезоэффекту а так же обработки
экспериментальных данных
4. были проведены эксперименты и их обработка для измерения пьезоэлектрических констант.
Расширение данных алгоритмов, написание дополнительных модулей в условиях внешних воздейсвий
и апрбация данных модулей в рамках исследования пьезоэлектрических свойств кристалла и
измерения пьезоэлектрических постоянных по данным рентгеновских дефракци



цель, разработка аппаратно програмного комплекса учитывющего все особенности экспериментальной схемы
позволяющего проводить моделирование картины дефракции для реального эксперимента
по исследованию вненего воздействия на кристалл внешнего ээл поля
а) любая реальна схема
б) позволяет учитываются воздействие внешнего поля на кристалл



таким образом в рамках данной рабботы была достигнута поставленная цель
в частности был разработан комплекс программ,
разработка данного комплекса открывает широкие перспективы для дальнейших
научных исследований
учета поляризуемсоти излучения учета деффектности кристаллов
