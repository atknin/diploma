\subsubsection{Рентгеновская поляризуемость в среде}

Вне кристалла, падающая волна описывается в виде совокупности плоских волн с волновым вектором  $\vec{k_0}$.
\begin{equation}
  \vec{E}_0(\vec{r},t) = E_0 e^{i(\vec{k}_0\vec{r}-\omega t)}
 \end{equation}

Падающая волна $vec{E}(\vec{r},t)_0$ порождает волновое поле внутри кристалла, которое характеризуется вектором
электромагнитной индукции $\vec{D}||\vec{E}$
\begin{equation}
 \vec{D}_0(\vec{r},t) = (1+\chi(\vec{r})) E_0 e^{i(\vec{k}_0\vec{r}-\omega t)} = A(r) e^{i(\vec{k}_0\vec{r}-\omega t)}
\end{equation}

где, $\chi$ - поляризуемость среды. Амплитуда волны $A(\vec{r})$ - не зависит
от времени, но зависит от координат, связано
это с тем что электроны колеблются под действие распространяющейся волны, и испускаемые ими
электромагнитные волны интерферируют между собой и с исходной волной. Устанавливается некоторое стабильное
электромагнитное поле с переодически изменяющейся в пространстве амплитудой. Периодичность эта должна быть
той же, что и периодичность решетки. Таким образом, в силу трехмерной периодичности $\chi(\vec{r}+\vec{h}) = \chi(\vec{r})$
, функцию $\chi(\vec{r})$ можно разложить в ряд Фурье и представить в виде

\begin{equation}
\chi(\vec{r}) = \sum_{h}\chi_h e^{i\vec{h}\vec{r}}
\end{equation}
Подробный вывод выражений для Фурье компонент $\chi_h$ представлен в (~\ref{sec:polarizability}), получим


\begin{equation}
\chi_h = -\frac{e^2 \lambda^2}{m \pi c^2} \frac{1}{V} \sum_{n} f_n \cdot e^{-2\pi i\vec{h}\cdot \vec{r}_n}
\end{equation}
где, $h$ - соответсвует какому - либо направлению вектора обратной решетки, для конкретных {hkl}.

На данном этапе мы не рассматриваем возможность распространение в кристалле большого количества
волн (многоголовой случай), а рассмотрим только два узла обратной решетки h - [000] и h - [hkl].
Тогда поляризуемость примет конечный вид

\begin{equation}
\chi(\vec{r}) = \chi_0 + \chi_h e^{i\vec{h}\vec{r}} + \chi_{-h} e^{-i\vec{h}\vec{r}}
\end{equation}

\textcolor{mygreen}{Если будет время, нарисовать распределение волнового поля для всех волн и их суммы}
