Книги
Однотомное издание
Автор. Заглавие: сведения, относящиеся к заглавию (см. на титуле) / сведения об
ответственности (авторы); последующие сведения об ответственности (редакторы,
переводчики, коллективы). – Сведения об издании (информация о переиздании, номер
издания). – Место издания: Издательство, Год издания. – Объем. – (Серия).
Примеры:
---------------------------------------------------------------------------------------------------
Если у издания один автор, то описание начинается с фамилии и инициалов автора.
Далее через точку «.» пишется заглавие. За косой чертой «/» после заглавия имя
автора повторяется, как сведение об ответственности.

Лукаш, Ю.А. Индивидуальный предприниматель без образования юридического лица [Текст]
/ Ю.А. Лукаш.    – Москва: Книжный мир, 2002. – 457 с.
--------------------------------------------------------------------------------

Если у издания два автора, то описание начинается с фамилии и инициалов первого автора.
За косой чертой «/» после заглавия сначала указывается первый автор, а потом через запятую – второй автор.

Бычкова, С.М. Планирование в аудите [Текст]/ С.М. Бычкова, А.В. Газорян.-Москва:  Финансы и статистика, 2001. – 263 с.
------------------------------------------------------------------------------------------------
Если у издания три автора, то описание начинается с фамилии и инициалов первого автора.
 За косой чертой «/» после заглавия сначала указывается первый автор, а потом через запятую – второй и третий авторы.

Краснова, Л.П. Бухгалтерский учет [Текст]: учебник для вузов /Л.П. Краснова, Н.Т. Шалашова,  Н.М. Ярцева. – Москва: Юристъ, 2001. – 550 с.
----------------------------------------------------------------
Если у издания четыре автора, то описание начинается с заглавия. За косой чертой указываются все авторы.
Лесоводство [Текст]: учебное пособие к курсовому проектированию/З.В. Ерохина, Н.П. Гордина, Н.Г. Спицына, В.Г. Атрохин.  –   Красноярск: Изд-во СибГТУ, 2000. - 175 с.
----------------------------------------------------------------
Если у издания  пять авторов и более, то описание начинается с заглавия. За косой чертой указываются  три автора и др.
Логика [Текст]: учебное пособие для 10-11 классов / А.Д. Гетманова, А.Л. Никифоров, М.И. Панов и др. – Москва: Дрофа, 1995. – 156 с.
----------------------------------------------------------------
Если у издания есть один или несколько авторов, и также указаны редакторы, составители, переводчики и т.п., то информация о них указывается в сведении об отвественности, после всех авторов перед точкой с запятой «;».
Ашервуд Б. Азбука общения [Текст]  / Б. Ашервуд; пер. с анг. И.Ю.Багровой и Р.З. Пановой, науч. ред. Л.М. Иньковой. – Москва: Либерея, 1995. – 175 с.
----------------------------------------------------------------
Если у издания нет автора, но указаны редакторы, составители, переводчики и т.п., то описание начинается с заглавия. За косой чертой после заглавия сразу пишутся редакторы, составители и т.п. с указанием функции.
Логопедия [Текст]: учебник для студ. дефектолог. фак. пед. вузов / ред. Л.С. Волкова, С.Н. Шаховская. – 3-е изд., перераб. и доп. – Москва: Гуманит. изд. центр. ВЛАДОС, 2002. – 680 с.

Если у издания нет автора, редакторов и т.п., то после заглавия сразу идет информация об издании после точки и тире «. -  ».
Иллюстрированный словарь английского и русского языка с указателями [Текст].  – Москва: Живой язык, 2003. – 1000 с.
