\newpage
  \begin{center}
  \section{ }% матрицы пьезомодулей
  \label{sec:piezo_matrix}
  \end{center}
  \subsection*{ Кристалл LGT }
  Кристаллы семейства лантан-галлиевого силиката ($La_3Ga_5SiO_{14}$ - LGS и $La_3Ga_{5.5}Ta_{0.5}O_{14}$ - LGT)
  обладают пьезоэлектрическими свойствами со стабильной температурной зависимость даже при высоки температурах.
  Пьзоэлектрический модуль $d_{11}$ остается постоянным в диапазоне температур до $600^o$С
  (изменение не более 5 $\%$ \cite{LGS58}). В таких кристаллах отсутствует фазовый переход вплоть
  до температур плавления \cite{LGS57}, а так же не имеется пироэлектрического эффекта.
  Отсутсвует гистерезис физических свойств, в том числе и пьезоэлектрический эффект,
  обладают высоким коэффициентом электромеханической связи (более чем в два раза больше чем у кварца).
  Высокое удельное сопротивление, которое говорит нам об отсутствие дополнительных эффектов,
  которые могли бы вносить свой вклад в картину дифракции при воздействии внешнего электрического
  поля (образование двойного электрического слоя и др.)

  LGT кристалл имеет точечную группой симметрии $32$ и гексагональной сингонию.
  \begin{equation}
    \begin{pmatrix}
    d_{11} & -d_{11} & 0 & d_{14} & 0 & 0 \\
    0 & 0 & 0 & 0 & -d_{14} & 2d_{11} \\
    0 & 0 & 0 & 0 & 0 & 0
    \end{pmatrix}
    \label{eq:piezomodule_lgt_matrica}
  \end{equation}
  Параметры ячейки: $a = b = 8.228 \angstrom$, $c = 5.124 \angstrom$ \cite{marchenkov2014}.

  \subsection*{ Кристалл $TeO_2$ }
  ...
  \subsection*{ Кристалл $Li_2B_4O_7$ }
  ...
