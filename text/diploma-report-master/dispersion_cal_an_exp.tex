\label{sec:dispersion_cal_an_exp}
  Дисперсия возникает когда есть некое спектральное распределение источника и
   угол Брэгга монохроматора отличается от угла Брэгга исследуемого кристалла-образца
   (рис. \ref{fig:double_crystal_schem_disp_a}).
  \begin{figure}[H]
    \centering
    \subfloat[$\theta_B^M \neq \theta_B^S$]{\includegraphics[width=0.6\textwidth]{images/double_crystal_schem_disp.png}\label{fig:double_crystal_schem_disp_a}}
    \hfill
    \subfloat[Спектральное-угловое распределение]{\includegraphics[width=0.35\textwidth]{images/double_crystal_lamtet_disp.png}\label{fig:double_crystal_schem_disp_b}}
    \caption{Дисперсионная схема дифракции}
    \label{ris:double_crystal_schem_disp}
  \end{figure}
  Факт наличия дисперсии можно проанализировать на спектрально-угловом распределении
  (рис. \ref{fig:double_crystal_schem_disp_b}), прямая образца в этом случае не параллельна прямой монохроматора и
  в области, близкой к точному брэгговскому отражению происходит не наложение одной на другую, как в случае отсутствия дисперсии,
  а их пересечение. В точке пересечение коэффициент отражения практически равен единице,
  легко заметить что кривая отражения будет уширенной (рис. \ref{ris:disspersion_curves_expantheory}).
\begin{figure}[H]
  \centering
  \subfloat[]{\includegraphics[width=0.45\textwidth]{images/disspers_220_440_100mcm.png}\label{fig:}}
  \hfill
  \subfloat[]{\includegraphics[width=0.45\textwidth]{images/disspers_220_660_100mcm.png}\label{fig:}}
  \hfill
  \subfloat[]{\includegraphics[width=0.45\textwidth]{images/disspers_220_440_300mcm.png}\label{fig:}}
  \hfill
  \subfloat[]{\includegraphics[width=0.45\textwidth]{images/disspers_220_660_300mcm.png}\label{fig:}}
  \caption{Двухкристальная КДО для схемы с кристаллом монохроматором Si(220) - $\theta_B = 10.6^o$ для дисперсионного случая для разных размеров щелевых устройств:
  (a) образец Si(440) - $\theta_B = 21.7^o$, $S_1 = S_2 = 100$ мкм, (b) образец Si(660) - $\theta_B = 33.7^o$, $S_1 = S_2 = 100$ мкм,
   (c) образец Si(440) - $\theta_B = 21.7^o$, $S_1 = S_2 = 300$ мкм, (d) образец Si(660) - $\theta_B = 33.7^o$, $S_1 = S_2 = 300$ мкм}
  \label{ris:disspersion_curves_expantheory}
\end{figure}

Дисперсия приводит к уширение КДО, величина которого может превышать полуширины бездисперсионных
двухкристальных КДО в несколько раз.В дисперсионное схеме дифракции существенно заметно
влияние размера щелевых устройств на полуширину КДО в отличии от бездисперсионных (раздел \ref{sec:non_disspers_KDO_section}).

\begin{equation}
 f_d^2 = \frac{f_S^2}{b_S}+\frac{f_M^2}{b_M}+\left( \frac{\Delta \lambda}{\lambda} (\tan(\theta_B^S) - \tan(\theta_B^M)) \right)^2
 \label{eq:disspers_width}
\end{equation}

Так же существует формула (\ref{eq:disspers_width}) которая выводится в приближении
гауссовской формы собственной кривой отражения от кристалла и приближенно описывает полуширину двухкристальных КДО
с учетом дисперсии. Дисперсионное слагаемое возникает из условия Брэгга.
Только точный учет дисперсии позволит производить адекватный анализ собственных кривых образца. Безусловно оценка
дисперсионности схемы важна и при исследовании пьезоэффекта.
