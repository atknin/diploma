  \label{sec:piezo_matrix}

  \subsubsection*{ Кристалл Si }
  При проведении экспериментов, необходимых для апробации разрабатываемых алгоритмов,
  был использован монокристалл кремния (Si). Данный
  кристалл характеризуется близкой к теоретической кривой собственного отражения, поэтому
  был использован в качестве модельного образца. Решетка Si имеет кубическую симметрию
  с параметрами элементарной ячейки  $a = b = c = 5.4310 \angstrom$.

  \subsubsection*{ Кристалл LGT }
  Кристаллы семейства лантан-галлиевого силиката ($La_3Ga_5SiO_{14}$ - LGS и $La_3Ga_{5.5}Ta_{0.5}O_{14}$ - LGT)
  обладают пьезоэлектрическими свойствами со стабильной температурной зависимостью
  даже при высоких температурах. Пьзоэлектрический модуль $d_{11}$ остается
  постоянным в диапазоне температур до $600^o$С (изменение не более 5 $\%$ \cite{LGS58}).
  В таких кристаллах отсутствует фазовый переход вплоть до температур плавления \cite{LGS57},
   а также не имеется пироэлектрического эффекта.

  Кристалл LGT имеет точечную группу симметрии $32$ и гексагональную сингонию.
  Матрица пьезоэлектрических модулей выглядит следующим образом:
  \begin{equation}
    \begin{pmatrix}
    d_{11} & -d_{11} & 0 & d_{14} & 0 & 0 \\
    0 & 0 & 0 & 0 & -d_{14} & 2d_{11} \\
    0 & 0 & 0 & 0 & 0 & 0
    \end{pmatrix}.
    \label{eq:piezomodule_lgt_matrica}
  \end{equation}
  
  Параметры ячейки: $a = b = 8.228 \angstrom$, $c = 5.124 \angstrom$ \cite{marchenkov2014}.
  %
  % \subsection*{ Кристалл $TeO_2$ }
  % ...
  % \subsection*{ Кристалл $Li_2B_4O_7$ }
  % ...
