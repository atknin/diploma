
При расчете структурных амплитуд рассеяния необходимо учитывать
тепловые колебания атомов в решетке. Предполагается, что атомы колеблются около положения
равновесия независимо друг от друга. Тогда это эквивалентно увеличению радиуса атома,
что приводит к более быстрому спаду функции атомного рассеяния с ростом угла рассеяния.
С другой стороны эффективное увеличение радиуса атома, очевидно, должно зависеть от
величины среднеквадратичного смещения смещения атома  $<u^2>$ из положения равновесия.
Также для простоты предполагается, что период тепловых колебаний атомов намного больше периода
колебаний падающего излучения, тем самым можно считать атом неподвижным в момент рассеяния,
т.е. пренебречь эффектом Доплера.

Таким образом, структурный фактор усредняется за время наблюдения по всем возможным отклонениям
\begin{equation}
  F_T = \left\langle \sum_{n} f_n \cdot  e^{-i\vec {h} \cdot (\vec{r_n}+ \vec{u}(t))} \right\rangle =  \sum_{n} f_n \cdot  e^{-i\vec {h} \cdot \vec{r_n}}  \left\langle e^{-i \vec{h} \cdot \vec {u}(t)  } \right\rangle,
 \end{equation}
\noindent
где $\vec{u(t)}$ - отклонение атома во времени, $\vec{r_n}$ - положение атома $n$
в идеальной ячейке, $\vec{h}$ - вектор обратной решетки, $|\vec{h}| = 2 \pi / d = $ где $d$ - межплоскостное расстояние,
а суммирование производится, по всем атомам элементарной ячейки.

При разложении экспоненты, содержащей параметр отклонения, в ряд Тейлора, получается следующее выражение:

\begin{equation}
  \left\langle e^{-i \vec{h} \cdot \vec {u}(t)  } \right\rangle = 1 - i  \left\langle \vec{h} \cdot \vec {u} \right\rangle - \frac{1}{2} \left\langle (\vec{h} \cdot \vec {u})^2 \right\rangle+ \ldots .
 \end{equation}
\noindent
 Cреднее значение всех членов нечетной степени будет тождественно равно нулю.
Учитывая,  $ \left\langle (\vec{h} \cdot \vec {u})^2 \right\rangle = q^2 <u^2> <cos(\theta)> = \frac{1}{2}<u^2>h^2$, можно преобразовать:

\begin{equation}
1 - i  \left\langle \vec{h} \cdot \vec {u} \right\rangle - \frac{1}{2} \left\langle (\vec{h} \cdot \vec {u})^2 \right\rangle+ \ldots = e^{-\frac{1}{2} <u^2> h^2},
 \end{equation}
 \begin{equation}
   F_T =  \sum_{n} f_n \cdot  e^{-i\vec {h} \cdot \vec{r_n}}  e^{-B (\frac{sin\theta_B}{\lambda} )^2 },
  \end{equation}
\noindent

 где $B = 8 \pi^2 <u^2>$ - температурный коэффициент Дебая - Валлера,
 $(\frac{h}{4\pi})^2=(\frac{sin\vartheta_B}{\lambda})^2$ -
 вектор обратной решетки или вектор рассеяния. Обычно температурный коэффициент
 находится в пределах от $0.20 \angstrom ^2$ до $3.0 \angstrom ^2$.
 Приведенные выше рассуждения верны в приближении о том, что все колебания в кристалле изотропные
 (изотропное гармоническое приближение). В более общем случае
 температурный коэффициент определяется тензором третьего ранга \cite{Willis1975}.
 В большинстве случаев гармоническое приближение дает адекватное описание, однако при описании
 атомных колебаний в области высоких температур, когда амплитуда колебаний сопоставима с расстоянием
 между соседними атомами, гармоническое приближение некорректно. В этом случае нужно учитывать ангармонические
 поправки:
 \begin{equation}
 <u^2> = <u^2_{harm}> (1+2\gamma \alpha T),
\end{equation}
\noindent
где $\gamma$ - константа Грюнайзена, $\alpha$ - объемный коэффициент теплового расширения, $T$ - температура.
В случае возрастания температуры кристалла, интенсивность брэгговского рефлекса будет уменьшаться,
но угловая полуширина отраженной кривой постоянной останется прежней.

 Кроме теплового фактора Дебая-Валлера (динамического), существует статическая поправка,
 величина которой в первую очередь зависит от концентрации дефектов в образце.
 Такой вклад меньше зависит от температуры, поэтому проведение температурных измерений
 обычно позволяет разделить статический и динамический вклады \cite{kibalin2015}.


% Разложим второе слагаемое в ряд Тейлора, тогда среднее значение всех членов нечетной степени
% будет тождественно равно нулю. Ограничившись вторым порядком разложения, получим:
%
% \begin{equation}
%   F_T = F \cdot  \left(1 - 4 \pi^2 <u^2> \left( \frac{h}{a} + \frac{k}{b} + \frac{l}{c}\right)^2 \right)
%  \end{equation}

% Мерой смещения атомов при тепловых колебаниях служит
% их среднеквадратичная амплитуда:
% \begin{equation}\label{eq:debay}
%   <u^2> = \frac{9\hbar^2 T}{m k_B \Theta_D^2}
%  \end{equation}
% где $\hbar$ - постоянная Планка, $k_B$ - постоянная Больцмана, $\Theta_D$ - температура Дебая.
% Величина $B$ может варьироваться в диапазоне от $1 \angstrom $ до $ 100\angstrom $.
%
% В случае возрастания температуры кристалла, интенсивность брэговского рефлекса будет уменьшаться,
% но угловая полуширина отраженной кривой постоянной останется прежней. Удивительным является то, что
% удается получить достаточно узкие кривые отражения от кристалла в котором атомы случайным
% образом смещены относительно равновесных положений, относительное изменение расстояния
% между соседними атомами может составлять до 10$\%$ при комнатной температуре.
