
 Из системы уравнение Максвелла можно получить следующее волновое уравнение \cite{pinsker1982}:

\begin{equation}
 \Laplace \vec{E} - k_0^2 \vec{D} = \Laplace \vec{E} - k_0^2 (1+\chi)\vec{E} = 0.
 \label{eq:wave_maxwel}
\end{equation}

Как было упомянуто выше, в кристалле распространяются две волны:
\begin{equation}
 \begin{cases}
   \vec{E}_0 = \vec{e}_0 E_0 e^{i\vec{q}_0\vec{r}}
   \\
   \vec{E}_h = \vec{e}_h E_h e^{i\vec{q}_h\vec{r}},
 \end{cases}
 \label{eq:E_0_E_h}
\end{equation}
\noindent
где произведение единичных векторов $\vec{e}_0$ и $\vec{e}_h$ определяется следующим образом:
\begin{equation}
\vec{e}_0 \cdot \vec{e}_h = C
 \begin{cases}
   1, \quad \quad \quad \quad  \sigma    - \text{поляризация}\\
   \cos(2\theta_B), \quad   \pi - \text{поляризация}.
 \end{cases}
\end{equation}

\begin{figure}[H]
  \centering
  \includegraphics[width=0.7\textwidth]{images/polarize_E.png}
  \caption{ Колебание вектора напряженности электрического поля для разных типов
  линейной поляризации рентгеновского излучения $\sigma$ (слева) и $\pi$ (справа)}
  \label{ris:polarize_E}
\end{figure}

Система динамических уравнений получается путем подстановки (\ref{eq:E_0_E_h}) в (\ref{eq:wave_maxwel}),
которая имеет следующий вид:

\begin{equation}
 \begin{cases}
   \delta_0 E_0 - C\chi_{-h}E_h=0
   \\
   \delta_h E_h - C\chi_{h}E_0=0,
 \end{cases}
\end{equation}
\noindent
где
\begin{equation}
   \delta_{(0,h)} = \frac{q_{(0,h)}^2}{k_0^2}-1-\chi_0.
\end{equation}

Приравняв детерминант системы к 0, дисперсионное уравнение имеет следующий вид:

\begin{equation}
   \delta_0 \delta_h -C^2 \chi_h \chi_{-h} = 0.
\end{equation}

Исходя из равенства тангенциальных компонент волнового вектора при переходе
 между средами (\ref{eq:k_h_squred}, \ref{eq:k_0_squred}),
необходимо отметить $k_h = q_h$, т.к при выходе излучения из среды происходит
лишь преломление, а суммарная интенсивность остается прежней.
\begin{equation}
 \begin{cases}
   \delta_0 = \frac{q_0^2 - k_0^2}{k_0^2} - \chi_0 = 2\varepsilon\gamma_0 - \chi_0
   \\
   \delta_h = \frac{q_h^2 - k_0^2}{k_0^2} - \chi_0 = 2\varepsilon\gamma_h - \alpha \chi_0,
 \end{cases}
\end{equation}
\noindent
где $\alpha$ соответствует выражению (\ref{eq:alpha}). Дисперсионное уравнение
с учетом граничных условий имеет следущий вид:

\begin{equation}
   (2\varepsilon \gamma_0 - \chi_0)(2\varepsilon \gamma_h - \alpha - \chi_0) - C^2 \chi_h\chi_{-h} = 0.
   \label{eq:dissperionnoe_eq}
\end{equation}

Решение уравнения (\ref{eq:dissperionnoe_eq}) относительно параметра аккомодации $\varepsilon$ имеет два корня:

\begin{equation}
   \varepsilon_{1,2} = \frac{1}{4\gamma_0} \left( \chi_0 (1-b) - b\alpha \pm \left( [\chi_0(1+b)+b\alpha]^2 - 4bC^2 \cdot \chi_{h}\chi_{-h} \right)^{1/2} \right),
\end{equation}
\noindent
где $b$ - соответсвует (\ref{eq:koef_b}),  а произведение коэффициентов
 поляризуемости определяется следующим образом:

 $$\chi_{h} \cdot \chi_{-h} = Re(\chi_{h})^2-Im(\chi_{h})^2 - 2i \cdot Re(\chi_{h}) \cdot Im(\chi_{h}).$$

Наличие двух решений говорит о том, что в кристалле распространяется две проходящие и две дифрагированные волны, но
анализ полученного решения $ \varepsilon_{1,2}$ показывает, что один корень имеет положительную мнимую часть, а второй -
отрицательную. Мнимая часть отвечает за поглощение, и в случае отрицательно корня волна, распространяясь
вглубь кристалла, экспоненциально затухает. Поэтому необходимо выбирать всегда корень с отрицательной мнимой частью
$Im(\varepsilon)<0$.

Амплитудный коэффициент отражения:
\begin{equation}
    R = \frac{E_0}{E_h} = \frac{\delta_0}{C\chi_{-h}} = \frac{2\varepsilon\gamma_0-\chi_0}{C\chi_{-h}}.
\end{equation}

Вид кривой диффракционного отражения (КДО) описывается следующим выражением \cite{Bushuev_Oreshko_2002}:
\begin{equation}
    \label{eq:KDO_self}
    P (\vartheta) =  |\frac{\gamma_h}{\gamma_0} \cdot R|^2.
\end{equation}

\begin{figure}[H]
  \centering
  \includegraphics[width=0.5\textwidth]{images/typical_rocking_curve.png}
  \caption{КДО от кристалла кремния Si(220) $CuK_{\alpha}$ - излучения для различных коэффициентов ассиметрии
  отражения: $b = 0.1$ (1); $b = 1$ (2); $b = 10$ (3)}
  \label{ris:typical_rocking_curve}
\end{figure}
На рис. \ref{ris:typical_rocking_curve} приведен типичный пример собственной кривой дифракционного отражения
для разных коэффициентов асимметрии \cite{Bushuev_Oreshko_2002}.
