\subsection{Температурный фактор рассеяния}
При определении положений атомов следует также
учитывать их тепловые колебания около равновесных
 положений, нарушающих «совершенность» решетки.
 Мерой смещения атомов при тепловых колебаниях служит
 их среднеквадратичная амплитуда $u^2$.
 Структурный фактор рассеяния для колеблющихся атомов имеет вид:

 \begin{equation}
   F_T = F\cdot e^{-B(\frac{sin\vartheta_B}{\lambda})^2}
  \end{equation}
где $B = 8 \pi^2 u^2$ - температурный фактор (фактор Дебая - Валлера).
Величина $B$ может варьироваться в диапазоне от $1 \angstrom $ до $ 100\angstrom $.
