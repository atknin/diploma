\subsection{Температурный фактор рассеяния}
При определении положений атомов следует также
учитывать их тепловые колебания около равновесных
 положений, нарушающих «совершенность» решетки. Будем считать, что амплитуда
 тепловых колебаний подчинятся нормальному закону. Наличие же  колебаний
 решётки приводит к уменьшению интенсивности рассеяния.
Структурный фактор рассеяния в изотропном $p$ приближении для колеблющихся атомов
имеет вид:

 \begin{equation}
   F_T = F\cdot e^{-B(T)(\frac{sin\vartheta_B}{\lambda})^2} = F\cdot e^{-2W_p}
  \end{equation}

где $B(T) = 8 \pi^2 <u^2>$ - температурный коэффициент Дебая - Валлера, $(\frac{sin\vartheta_B}{\lambda})^2$ -
вектор вектор обратной решетки или вектор рассеяния.
% (\ref{eq:debay})
Мерой смещения атомов при тепловых колебаниях служит
их среднеквадратичная амплитуда:
\begin{equation}\label{eq:debay}
  <u^2> = \frac{9\hbar^2 T}{m k_B \Theta_D^2}
 \end{equation}
где $\hbar$ - постоянная Планка, $k_B$ - постоянная Больцмана, $\Theta_D$ - температура Дебая.
Величина $B$ может варьироваться в диапазоне от $1 \angstrom $ до $ 100\angstrom $.

В случае возрастания температуры кристалла, интенсивность Бреговского рефлекса будет уменьшаться,
но угловая полуширина отраженной кривой будет постоянной. Удивительным является то, что
удается получить достаточно узкие кривые отражения от кристалла в котором атомы случайным
образом смещены относительно равновесных положений, относительное изменение расстояния
между соседними атомами может составлять до 10$\%$ при комнатной температуре.
