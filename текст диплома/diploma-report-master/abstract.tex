\section*{Реферат}
\thispagestyle{empty}

Дипломный проект выполнен на 6 листах формата А1 с пояснительной запиской на 67 страницах (без приложений справочного или информационного характера). Пояснительная записка включает 6 глав, 18 рисунков, 22 литературных источников.

\emph{Ключевые слова}: графические структуры, дескриптор признаков, shape context, адаптивное усиление классификаторов, adaboost, оператор обнаружения краев, canny.

Целью данного дипломного проекта является разработка приложения для захвата людей на изображении. ПО может быть внедрено в другие проекты с целью получения информации об объекте на изображении. Пояснительная записка состоит из следующих частей:

\emph{Введение}: описана актуальность проблемы в области компьютерного зрения; предлагается решение проблемы на основе графических структур;

\emph{Глава 1}: описан оператор обнаружения краев Канни;

\emph{Глава 2}: рассмотрен метод усиления простых классификаторов; описан алгоритм AdaBoost; изложена его математическая часть;

\emph{Глава 3}: описан дескриптор признаков Shape Context; изложен необходимый математический аппарат; описаны возможные области применения;

\emph{Глава 4}: описание графических структур; приведен метод захвата людей на изображении; синтез изложенного в предыдущих разделах в единый фреймворк;

\emph{Глава 5}: рассмотрена реализация пространственно\hyp{}антропометрической эргономической совместимости работника и технического средства при организации рабочего места;

\emph{Глава 6}: приводится расчет себестоимости разработки, дается расчет экономического эффекта от использования программного средства;

\emph{Заключение}: содержит краткие выводы по дипломному проекту.

\newpage
