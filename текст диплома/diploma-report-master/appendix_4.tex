\newpage
  \begin{center}
  \section{ }%Как посчитать щели
  \label{sec:calc_slits_ability}
  \end{center}


\begin{figure}[H]
  \centering
  \includegraphics[width=0.8\textwidth]{images/calc_slits_ability.png}
  \caption{Для расчета пропускной способности системы щелевых устройств}
  \label{ris:calc_slits_ability}
\end{figure}

Для того чтобы получить зависимость пропускной способности в зависимости от угла,
под которым распространяется рентгеновский луч, необходимо спроецировать границы
щелей на уровень источника.
$$ S^{(source)}_{1,2} = \pm \frac{S_{1,2}}{2} - \vartheta L_{1,2}$$
 Далее найти минимальное значение проекции верхних (up)
 $$ a^{up} = min [S_1^{up,source},S_2^{up,source},\frac{\delta}{2}]$$
и максимальное значение среди проекции нижних (down)
$$ a^{down} = max [S_1^{down,source},S_2^{down,source},-\frac{\delta}{2}]$$

Следующее условие будет определять величину площади параллелограмма, а соответсвенно
и характеризовать пропускную способность для разных направлений $\vartheta$

\begin{equation}
  g_s(\vartheta) =
 \begin{cases}
   0, \quad \text{если} \quad a^{down} \geq a^{up}
   \\
    (a^{up} - a^{down})L_2,\quad \text{если} \quad a^{down} < a^{up}
 \end{cases}
\end{equation}
