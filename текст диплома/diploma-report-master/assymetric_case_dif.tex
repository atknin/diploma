\subsubsection{Асимметричная схема дифракции}
В том случае если рентгеновское излучение отражается от атомных плоскостей не
 параллельных поверхности, в таком случае говорят об асимметрии отражения (рисунок ~\ref{ris:assymetric_brag}).

\begin{figure}[h]
  \centering
  \subfloat[$b >> 1$, $\varphi$ > 0]{\includegraphics[width=0.45\textwidth]{images/assym2.png}\label{fig:f1}}
  \hfill
  \subfloat[$b << 1$, $\varphi$ < 0]{\includegraphics[width=0.45\textwidth]{images/assym1.png}\label{fig:f2}}
  \caption{Схема Брегговской дифракции для асимметричного отражения}
  \label{ris:assymetric_brag}
\end{figure}

Для того чтобы охарактеризовать степень асимметрии, введем коэффициент $b$:

\begin{equation}
  b = \frac {\gamma_0}{|\gamma_h|}
 \end{equation}
где, $\gamma_0 = cos \psi_0 = sin ( \varphi + \theta_B)$, $\gamma_h = cos \psi_h = sin ( \varphi - \theta_B)$,
$\varphi$ - угол между плоскостью отражения и поверхностью образца.


Весьма наглядной иллюстрацией являются собственные кривые отражения от Si(440) рассчитанные при
трех разных углах падения и соответсвенно имею разный коэффициент асимметрии. Угол
Брегга для такой плоскости отражения составляет $\theta_B = 21.68^o$, угол наклона поверхности
составляет $\varphi = 20^o 53^{'}$.

\begin{figure}[H]
\centering
\includegraphics[width=0.99\textwidth]{images/rocking_curve_assym_3.png}
\caption{Кривые отражения 440 $MoK_{\alpha 1}$ от Si, полученные при разных углах падения(для разных b)}
\label{ris:rocking_curve_assym_3}
\end{figure}
Сдвиг центра кривой происходит из-за наличия преломления на величину 0.5 и 16.5 угловых секунд.

Варьируя угол между поверхностью кристалла и отражающей плоскостью (например, с помощью шлифовки),
можно существенно изменить ширину рентгеновского пучка (рисунок ~\ref{ris:assym_width_beam}).
\begin{figure}[H]
 \centering
 \includegraphics[width=0.4\textwidth]{images/assym_width_beam.png}
 \caption{Кристалл с асимметричным отражением по Бреггу}
 \label{ris:assym_width_beam}
\end{figure}
