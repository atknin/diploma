\section{Расчеты и эксперименты}
\subsection{Собственная кривая отражения}
\subsubsection{Симметричная}

\subsubsection{Асимметричная}
Весьма наглядной иллюстрацией являются собственные кривые отражения от Si(440) рассчитанные при
трех разных углах падения и соответсвенно имею разный коэффициент асимметрии. Угол
Брегга для такой плоскости отражения составляет $\theta_B = 21.68^o$, угол наклона поверхности
составляет $\varphi = 20^o 53^{'}$.

\begin{figure}[H]
\centering
\includegraphics[width=0.99\textwidth]{images/rocking_curve_assym_3.png}
\caption{Кривые отражения 440 $MoK_{\alpha 1}$ от Si, полученные при разных углах падения(для разных b)}
\label{ris:rocking_curve_assym_3}
\end{figure}
Сдвиг центра кривой происходит из-за наличия преломления на величину 0.5 и 16.5 угловых секунд.

Варьируя угол между поверхностью кристалла и отражающей плоскостью (например, с помощью шлифовки),
можно существенно изменить ширину рентгеновского пучка (рисунок ~\ref{ris:assym_width_beam}).
\begin{figure}[H]
 \centering
 \includegraphics[width=0.4\textwidth]{images/assym_width_beam.png}
 \caption{Кристалл с асимметричным отражением по Бреггу}
 \label{ris:assym_width_beam}
\end{figure}


\subsection{Нолькристальный эксперимент}
\begin{figure}[H]
  \centering
  \subfloat[$S_1 = 20$ мкм; $S_2 = 40$ мкм;]{\includegraphics[width=0.3\textwidth]{images/zero_exp_20_40.png}}
  \hfill
  \subfloat[$S_1 = 40$ мкм; $S_2 = 40$ мкм;]{\includegraphics[width=0.3\textwidth]{images/zero_exp_40_40.png}}
  \hfill
  \subfloat[$S_1 = 50$ мкм; $S_2 = 100$ мкм;]{\includegraphics[width=0.3\textwidth]{images/zero_exp_50_100.png}}
  \hfill
  % \subfloat[$S_1 = 60$ мкм; $S_2 = 40$ мкм;]{\includegraphics[width=0.3\textwidth]{images/zero_exp_60_40.png}}
  % \hfill
  \subfloat[$S_1 = 100$ мкм; $S_2 = 200$ мкм;]{\includegraphics[width=0.3\textwidth]{images/zero_exp_100_200.png}}
  \hfill
  \subfloat[$S_1 = 100$ мкм; $S_2 = 300$ мкм;]{\includegraphics[width=0.3\textwidth]{images/zero_exp_100_300.png}}
  \hfill
  \subfloat[$S_1 = 200$ мкм; $S_2 = 20$ мкм;]{\includegraphics[width=0.3\textwidth]{images/zero_exp_200_20.png}}
  \hfill
  \subfloat[$S_1 = 200$ мкм; $S_2 = 200$ мкм;]{\includegraphics[width=0.3\textwidth]{images/zero_exp_200_200.png}}
  \hfill
  \subfloat[$S_1 = 200$ мкм; $S_2 = 300$ мкм;]{\includegraphics[width=0.3\textwidth]{images/zero_exp_200_300.png}}
  \hfill
  \subfloat[$S_1 = 300$ мкм; $S_2 = 300$ мкм;]{\includegraphics[width=0.3\textwidth]{images/zero_exp_300_300.png}}
  \caption{Нолькристальный эксперимент}
  \label{ris:zero_exp}
\end{figure}






\subsection{Однокристальный эксперимент}


\begin{figure}[H]
  \centering
  \subfloat[$S = 50$ мкм; ]{\includegraphics[width=0.7\textwidth]{images/single_cr_exp_s_005mm.png}}
  \hfill
  \subfloat[$S = 200$ мкм;]{\includegraphics[width=0.7\textwidth]{images/single_cr_exp_s_02mm.png}}
  \caption{Однокристальный эксперимент}
  \label{ris:zero_exp}
\end{figure}

\subsection{Двухкристальный эксперимент}


  Метод анализа КДО по прежнему являются одним из основных инструментов диагностики не только совершенства
кристаллических материалов \cite{sov_1} - \cite{sov_5}, в частности, объемных и поверхностных дефектов в
монокристаллах, тонких пленках, а также многослойных кристаллических структурах, но и для анализа физических
процессов происходящих в кристаллах, таких как воздействие внешнего электрического поля [] (пьезоэлектрический эффект),
 температуры [] или влияние магнитного поля [].

\subsubsection{Вклад соседней характеристической линии в КДО}

\label{sec:non_disspers_KDO_section}
На рисунке \ref{ris:non_disspers_kdo} приведены результаты численного расчета в соответсвии
с выражением (\ref{eq:doudle_spectra_angle_map_on_detector}). В качестве кристалла монохроматора
и образца был выбран монокристалл кремния с отражающей плоскостью (220), эксперимент проводился в
соответсвии со схемой (рисунок \ref{ris:double_crystal_schem_lamtet_a}), материалом источника рентгеновского излучения является молибден.

\begin{figure}[H]
  \centering
  \subfloat[$S_1 = 20 $ мкм; $ S_2 = 40$ мкм;]{\includegraphics[width=0.45\textwidth]{images/non_disspers_20_40.png}\label{fig:f1}}
  \hfill
  \subfloat[$S_1 = 20 $ мкм; $ S_2 = 40$ мкм; ]{\includegraphics[width=0.45\textwidth]{images/non_disspers_20_40_log.png}\label{fig:non_disspers_kdo_1}}
  \hfill
  \subfloat[$S_1 = 300 $ мкм; $ S_2 = 200$ мкм;]{\includegraphics[width=0.45\textwidth]{images/non_disspers_300_200.png}\label{fig:f2}}
  \hfill
  \subfloat[$S_1 = 300 $ мкм; $ S_2 = 200$ мкм;]{\includegraphics[width=0.45\textwidth]{images/non_disspers_300_200_log.png}\label{fig:f2}}
  \caption{Двухкристальная КДО для схемы с кристаллом монохроматором Si(220) и образцом  Si(220); $L_1= 570 $мм,
  $L_2 = 1005$ мм; $\delta = 0.1$ мм; (красная линия) - расчет, (синие точки) - эксперимент.}
  \label{ris:non_disspers_kdo}
\end{figure}

На рисунке \ref{fig:non_disspers_kdo_1} видно, что наряду с главным пиком, соответствующим $k_{\alpha1}$ лиинии
излучения, на которую настроен монохроматор, присутствует вклад от соседней характеристической линии
 $k_{\alpha2}$. Впервые, на это свойство двухкристальных КДО, получаемы в бездисперсионной
схеме, в случае использования рентгеновской трубки было указано авторами работы \cite{chuev2008}

  \subsubsection{Дисперсионная схема дифракции}
\begin{figure}[H]
  \centering
  \subfloat[Образец Si(440), $S_1 = S_2 = 100$ мкм.]{\includegraphics[width=0.45\textwidth]{images/disspers_220_440_100mcm.png}\label{fig:f1}}
  \hfill
  \subfloat[Образец Si(660), $S_1 = S_2 = 100$ мкм.]{\includegraphics[width=0.45\textwidth]{images/disspers_220_660_100mcm.png}\label{fig:f2}}
  \hfill
  \subfloat[Образец Si(440), $S_1 = S_2 = 300$ мкм.]{\includegraphics[width=0.45\textwidth]{images/disspers_220_440_300mcm.png}\label{fig:f2}}
  \hfill
  \subfloat[Образец Si(660), $S_1 = S_2 = 300$ мкм.]{\includegraphics[width=0.45\textwidth]{images/disspers_220_660_300mcm.png}\label{fig:f2}}
  \caption{Двухкристальная КДО для схемы с кристаллом монохроматором Si(220) для дисперсионного случая}
  \label{ris:disspersion_curves_expantheory}
\end{figure}

    \subsubsection{Асимметричный случай отражения}
  На рисунке \ref{ris:assymetric_exp_50}
  приведены результаты двухкристального эксперимента, где в качестве
  кристалла образца и монохроматора использовался кристалл кремния Si(440). Образец был взят таким
  образом, что плоскость отражения располагалась под углом $\phi = 20.52^o$ к поверхности.

  \begin{figure}[H]
    \centering
    \subfloat[$b = 33.52$, $\varphi$ > 0]{\includegraphics[width=0.45\textwidth]{images/assym-blue-50.png}\label{ris:assymetric_exp_a}}
    \hfill
    \subfloat[$b = 0.03$, $\varphi$ < 0]{\includegraphics[width=0.45\textwidth]{images/assym-red-50.png}\label{ris:assymetric_exp_b}}
    \caption{Двухкристальная КДО для схемы с кристаллом монохроматором Si(440) и асимметричным образцом Si(440),
    угол разориентации поверхности $\varphi = 20^o53^{'}$. Размер щелевых устройств $S_1 = S_2 = 50$ мкм.}
    \label{ris:assymetric_exp_50}
  \end{figure}

  Как было показано в разделе \ref{sec:rocking_curve_section}, чтобы получить
  рентгеновский пучок с очень малой угловой расходимостью необходимо выбирать
  скользящий угол падения к поверхности кристалла \ref{ris:assymetric_exp_a}.

